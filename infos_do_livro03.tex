\documentclass{a4paper, 12pts}{book}
\usepackage[utf8]{inputenc}
\usepackage[brazil]{babel}
\usepackage{blindtext}
\begin{document}

Começamos o capitulo 3, sobre design de página.
Espera-se aprender:
\begin{itemize}
    \item Ajuste de margens;
    \item Mudança no espaçamento das linhas;
    \item Seccionar o documento;
    \item Criação de sumário;
    \item Design de cabeçalhos e rodapés;
    \item Controles de quebra de página;
    \item Definir notas de rodapé e modificar sua aparência.
\end{itemize}

\chapter{Exploring the page layout}
In this chapter we will study the layout of pages.

\section{Some filler text}
\blindtext
\section{A lot more filler text}
More dummy text will follow.
\subsection{Plenty of filler text}
\blindtext[10]

O pacote blindtext é usado para obter texto de preenchimento.
O pacote Babel altera o idioma desre texto filler.

O comando \emph{chapter} irá sempre começar uma nova página.
Para gerar texto filler, existe o pacote ipsum, que gera o famoso loremipsum.

O comando \emph{section} é similar, e já vimos sobre ele. Ele não começa uma nova página. Será que existe o comando \emph{chapter} para o tipo article?

Como livros são textos com páginas à esquerda e à direita, as margens ganham o seguinte significado:

\begin{itemize}
    \item A margem direita é a margem externa, já que capítulos começam na página direita.
    \item A margem esquerda é a margem interna, sempre maior por causa da colagem das folhas.
    \item A margem inferior contém a numeração de páginas.
    \item A margem superior possui um espaçamento maior em páginas de início de capítulo.
\end{itemize}

Para definir as margens fora do padrão \LaTeX\ utiliza-se o pacote \emph{geometry}.

\backslash usepackage\[a4paper, inner=1.5cm, outer=3cm, top=2cm, bottom=3cm, bindingoffset=1cm\]\{geometry\}

O pacote \emph{geometry} nasceu por conta de, nos primeiros dias do \LaTeX\, realizar-se a diagramação à mão.
Muitas vezes o tamanho margem esquerda + margem direita + comprimento do texto 
simplesmente não era compatível com o tamanho da página.

O \emph{geometry} compreende o formato chave-valor para seus parâmetros.
Se você utilizá-lo sem argumentos, pode executá-lo com \backslash geometry\{lista de argumentos\}.
Imagine a necessidade de se alterar a geometria a cada página.

Hora de listar o poder do geometry:

Pra começar, as opções do papel:
\begin{itemize}
    \item \emph{paper=name} - letterpaper, executivepaper, legalpaper, a0paper, \ldots ,a6paper, b0paper, \ldots ,b6paper;
    \item \emph{paperwidth} - paperwidth=10in;
    \item \emph{paperheight} - paperheight=7in;
    \item \emph{papersize=\{width, height\}} - papersize=\{7in,10in\};
    \item \emph{landscape} - ;
    \item \emph{portrait} - modo padrão;
\end{itemize}

\begin{itemize}
    \item \emph{textwidth} Delimita a largura que o texto ocupará na página, por exemplo \emph{textwidth=140mm};
    \item \emph{textheight} Delimita a altura que o texto ocupará na página, por exemplo \emph{textheight=180mm} ;
    \item \emph{lines} Outra forma de delimitar a altura do texto, por exemplo \emph{lines=25} ;
    \item \emph{includehead} Faz com que o cabeçalho seja incorporado como área de texto. A opção padrão é \emph{false} ;
    \item \emph{includefoot} Faz com que o rodapé seja incorporado como área de texto. A opção padrão é \emph{false} ;
    \item \emph{left ou right} Delimita o tamanho da margem, por exemplo \emph{left=3cm}. Usa-se no caso de documentos de uma página por folha;
    \item \emph{inner ou outer} Delimita o tamanho da margem interna ou externa, por exemplo \emph{inner=2cm}. Usa-se para documentos de duas páginas por folha;
    \item \emph{top, bottom} Especifica o tamanho das margens verticais como em \emph{top=25mm};
    \item \emph{twosides} Força o documento a adotar página esquerda e direita. Nas páginas verso os valores das margens esquerda e direita serão invertidos;
    \item \emph{bindingoffset} Separa um espaço para a união de folhas. Será aplicado à margem interna;
\end{itemize}

Para se alterar a geometria do documento, mesmo já tendo definido no preâmbulo,
usar o comando \backslash newgeometry\{argument list\},
e para retornar à geometria definida inicialmente \backslash restoregeometry.

Para encontrar documentação por tópicos, caso nem o nome se saiba:
\url{http://texcatalogue.sarovar.org/bytopic.html}

O livro sugere pesquisar sobre a documentação do pacote typearea, que utiliza a abordagem de delimitar a geometria
de forma mais semelhante com os padrões tipográficos (design de página).
Esse pacote é explicado no manual KOMA-Script.

Alterando o espaçamento das linhas:
Usa-se o pacote setspace que possui as seguintes opções:
\begin{itemize}
    \item \backslash usepackage[onehalfspacing]\{setspace\}
    \item \backslash usepackage[singlespacing]\{setspace\}- Opção padrão.
    \item \backslash usepackage[doublespacing]\{setspace\}
\end{itemize}

Normalmente definimos o espaçamento ebtre linhas para todo o documento.
Mas caso seja necessário, pode-se usar o ambiente spacing para alterar um trecho do texto.

\begin{spacing}{2.4}
    Este texto está esticado em uma razão de 2.4
\end{spacing}

Alguns pacotes não possuem documentação separada de seus arquivos fonte.
Esse é o caso de setspace.
Rodar depois o comando kpsewich setspace.sty
Ele irá mostrar onde está o arquivo do pacote.

Essa ferramenta pertence ã biblioteca kpathsea, focada em caminhos de arquivos.

Criando um documento com duas colunas e orientação paisagem:

\backslash documentclass[a4paper, 12pt, landscape, twocolumn]\{book\}

Verificar pelo comando \backslash twocolumn[Opening text]
e pelo pacote multicols

As classes base do LaTeX são book, article, report, slides, e letter.
Existe uma classe de documento melhor que letter, scrlttr2. Slides é velho e ruim, usar beamer ou powerdot.


Vamos verificar as opções de documentclass:
\begin{itemize}
    \item a4paper, b2paper, letterpaper, legalpaper, executivepaper.
    \item 10pt, 11pt ou 12pt. 
    \item landscape.
    \item onecolumn ou twocolumn.
    \item oneside ou twoside. Não se usa twoside com slides e letter.
    \item 
    \item 
\end{itemize}



\end{document}
