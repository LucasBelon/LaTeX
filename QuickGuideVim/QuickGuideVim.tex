\documentclass[a4paper, 12pt]{article}

%---------Pacotes Usados-----------------+
\usepackage{amsmath,                    %|
            amssymb,                    %|
            amsfonts,                   %|
            latexsym,                   %|
            mathrsfs,                   %|
            amsthm,                     %|
            amstext,                    %|
            bezier,                     %|
            amscd}                      %|
                                        %|
\usepackage[top=2cm,                    %|
            bottom=2cm,                 %|
            left=2.5cm,                 %|
            right=2.5cm]                %|
            {geometry}                  %|
                                        %|
\usepackage[utf8]{inputenc}             %|
                                        %|
\usepackage{indentfirst}                %|
                                        %|
\usepackage{graphicx}                   %|
%----------------------------------------+

\begin{document}

%---------Infos MakeTitle----------------+
\title{                                 %|
    \textbf{                            %|
    Anotações Definitivas               %|
    }                                   %|
    \break                              %|
    Aprendendo a usar VIM               %|
}                                       %|
                                        %|
\author{                                %|
    Lucas Gouveia Belon                 %|
    \\                                  %|
    lucasgouveiabelon@usp.br            %|   
}                                       %|
\date{\vspace{2.2cm}São Paulo\\2023}    %|
                                        %|
\maketitle                              %|
\newpage                                %|
%----------------------------------------+

%---------Sumário------------------------+
\tableofcontents                        %|
\newpage                                %|
%----------------------------------------+ 

% TODO: Encontrar uma imagem para adicionar na capa.
%\thispagestyle{empty}
%\setcounter{page}{-1}
%\thispagestyle{empty}

\section{Motivação}
Minha ideia original era aprender a utilizar o latex.
No entanto, é difícil aprender ambas as ferramentas.
Por conta disso, vou reescrever meu pequeno manual e fazer uma extensão com o que aprendi até então.
O uso de um editor específico com certeza simplifica muito as coisas, seja para edição de textos, seja para programação.
Porém, não é sempre que temos máquinas potentes o suficiente para isso.
O pacote de ferramentas que vem com o \LaTeX\ é gigantesco em variedade e em peso.

O pacote de ferramentas que acompanham qualquer outro software de edição de texto o são.
São alguns gigabytes de arquivos, muitas fontes, e diferentes compiladores.
Assustador, mas vamos com calma.
Este pequeno projeto é uma forma de eu me ambientar com o vimtex, e se ajudar alguém, é um ponto extra.

\subsection{Minha Motivação}
Quando comecei a faculdade havia a matéria de introdução à computação.
Nessa matéria sugeriu-se que se instalasse um linux para fazer os exercícios e entender melhor sobre computação e afins.
De fato funcionou.
Comecei a aprender, e procurar, e aos poucos fui criando uma trilha de aprendizado.
Nessa trilha me encontrei com o terminal, ou prompt de comando, ou shell.
No linux, é a forma mais eficiente de se executar uma tarefa.
Mesmo que muitas vezes não seja exatamente fácil, é possível salvar em um arquivo de script uma sequência de ações para que seja difícil uma única vez.
Isso me encantou.

Não bastasse isso, eu curso engenharia, e existem contas para se fazer, procedimentos de cálculo, aproximações e avaliações que podem ser facilitadas com computadores.
Existem diferentes tipos de programas, e todos possuem essa natureza estranha, a existência dentro de computadores.

E para finalizar, todas as distribuições Linux conseguem rodar ao menos a linha de comando e um editor de textos.
E o VIM é um editor e tanto.

\subsection{Movitação Para Você}
Se você não dispõe de uma máquina potente, e precisa começar com programação, talvez seja um caminho muito mais difícil o que te espera.
Mas é no caminho mais difícil que temos a evolução mais acentuada.
Aprendi a programar em 2 linguagens de programação, das 3 que conheço, no celular.

Me bastou ter um aplicativo de emulação de terminal.
Tendo no terminal um editor e um compilador, aprendi C, uma das linguagens mais velhas que se tem por aí.
Além disso, fazer exercícios de programação mantém a mente afiada, e exige criatividade.
Com o básico bem estabelecido se chega realmente muito longe.

Se você possui um computador fraco, pode ser que instalar uma distribuição Linux alivie o peso.
Perde-se muitos aplicativos nativos do Windows, e muitas vezes suas tarefas dependem específicamente do Microsoft Office, ou do AutoCad, ou da suíte Adobe.
Mas também existe a possibilidade de se utilizar programas alternativos.

Falando de ir além do básico, configurações em servidores são geralmente realizados sem interface gráfica.
Se trata de utilizar um computador distante, especializado em alguma tarefa.
E como eu disse antes, pode ser que pareça difícil, mas basta ser difícil uma única vez.

\subsection{Componentes}
O VIM possui pedaços.
E por ele começar tão simples que é uma ótima ferramenta para se evoluir.
Começamos com quase nada, e adicionamos camadas.

Outros programas de edição possuem os componentes que podem ser adicionados ao nosso editor favorito.
Outros componentes simplesmente não fazem sentido juntos dessa forma.
Pela versatilidade, e capacidade de se ter numa única ferramenta diversas configurações, decidi me dedicar a aprender.

\subsubsection{Editor de Texto}
O editor de texto é a parte núcleo, que quase todo servidor linux possui.
Tem-se as funcionalidades básicas: Leitura; Escrita; Busca; Substituição.
E é aí que começa a ficar divertido.
Existe o que se chama de syntax highlighting, a capacidade de exibir o texto com cores diferentes baseado na sintaxe, no sentido lógico usado.

Além disso, existem formas de se escrever, formas de se pensar, formas de se mover o cursor pelo texto, que só fazem sentido dentro do vim.
Acontece que essa forma de se pensar é benéfica de tal forma que editores de código especializados permitem integrar essa forma de se mover.

Temos também a gravação de macros, que são sequências de ações que podem ser tomadas e repetidas dentro de um texto.
Temos abreviações, que te impedem de escrever a mesma palavra errado e errado novamente, bastando configurar.
Também pode-se adicionar pequenos trechos de texto que são recorrentes no que se desenvolve.

Pode-se editar arquivos remotamente, pode-se utilizar abas, e divisões de janela para visualizar mais de um arquivo ao mesmo tempo.
E pode-se invocar um terminal para continuar executando comandos do lado de fora, enquanto anota-se do lado de dentro.
É muita versatilidade.

\subsubsection{Configurações}
Arquivos de configuração fazem loucuras.
Existem templates de configuração que transformam e aproximam o vim de editores especializados para usos específicos.
Existem configurações que facilitam o uso, e outras que são impossíveis de se desacostumar.
Nos arquivos de configuração, tornam-se persistentes as alterações que também podem ser feitas com o editor aberto.

As configurações permitem alterar atalhos, e existe uma infinidade deles.
Permitem salvar macros, abreviações, mudar esquemas de cores, aparência e ergonomia do uso.
Basicamente, os arquivos de configuração fazem com que o editor fique com a sua cara.
No extremo, chega-se a um ponto em que alguém que também seja conhecido do vim não consiga usar o seu vim. Nos arquivos de configuração também é onde se deixam plugins ativos.

\subsubsection{Plugins}
Para diversos usos, o vim é uma ferramenta tosca.
Não que seja impossível realizar qualquer tarefa com o básico do básico.
Mas se trata de uma ferramenta que começou com funcionalidades limitadas.
E programadores externos ao desenvolvimento possuiam formas de adicionar formas mais fáceis de se fazer uma tarefa.
Pensaram novas utilidades nunca antes pensadas, e extensões que se conectam com outros programas.

Talvez seja nos plugins que o vim deixa de ser uma ferramenta ruim para ser sincero.
Vê-se muitos vídeos, tantos vídeos, e blogs, e sites, sobre plugins que fazem isso e aquilo, que faz parecer que a única função do vim é agregar todos os plugins num único ponto de encontro.

Mas não é bem assim. Existem muitas ferramentas já integradas, e hoje em dia não é tão ruim assim.
Vamos avançar a maior parte do livro sem plugins.

\subsection{Configurando Máquinas Remotas}
Servidores de dados, de e-mail, servidores web e o mundo na nuvem.
Todos são parte da mesma família de tecnologias.
Aprender uma ferramenta irá te ajudar a entender a próxima, e conseguir aprender mais rápido é motivador.

Essas máquinas remotas podem ser tão simples quanto um computador de 10 anos atrás, como podem ser super computadores, e ainda assim, são máquinas especializadas.
Não adianta querer jogar nessas máquinas.
Assim como não adianta querer fazer manobras de skate com uma tábua.
É parecido, mas se trata de outra coisa.

Se você chegou até aqui e aturou minha enrolação, então não tem motivo para não ir até o final.
Tentarei ser o mais prático possível.
Planejo fazer de tal forma que você possa abandonar o livro pela metade e se virar muito bem.
Mas muitas funcionalidades interessantes demoraram para serem colhidas e entendidas por esse cérebro de geléia que está escrevendo. Então, vamos lá.

\newpage

\section{Introdução}
Quando se trata de aprender a utilizar as funcionalidades avançadas de um editor de texto muito antigo precisamos avançar um pequeno trecho por vez.
Para tecnologias antigas, existe na internet livremente todo tipo de dica, todo tipo de vídeo, e até mesmo todo tipo de texto.
Mas um dos trabalhos deste documento é organizar para tornar mais orgânica a absorção.

O vim possui um tutorial, acessível pela linha de comando.
Basta escrever "\$ vimtutor" que o tutorial começa.
% TODO: Dar um jeito de adicionar notação de linha de comando.
Assim como quase tudo na vida, você primeiro lê, depois tenta tomar alguma ação.
Pretendo seguir mais ou menos a mesma ordem do tutorial, mas com mais explicações.

Uma espécie de piada que aparece em certos meios é sobre não se saber sair do vim.
Podemos começar por aí.
Podemos iniciar o vim com ou sem um nome de arquivo especificado.
Digamos que entramos sem especificar.
Assim que se abre o vim entra-se no modo "normal".
Através de teclas de atalho podemos entrar em outros modos.
Existem poucas formas de intercambiar entre modos, sendo geralmente mais simples voltar para o modo normal antes de seguir para um outro.
Para voltar para o modo normal, pressione esc repetidamente, ou mesmo a combinação Ctrl+C, que irá desativar o modo atual.
Se você não fez nenhuma alteração no texto que está na tela, basta apertar \textbf{"\$:"} para entrar no modo "ex".
O modo ex serve para inserção de comandos.
Inseriremos o comando "quit".
% TODO: Criar um estilo para os comandos internos do vim

É parte da cultura unix criar comandos como abreviações de nomes maiores, como cat que abrevia con\textbf{cat}enate.
Você pode realizar o procedimento anterior usando apenas ":q".

Novamente com o vim aberto, pode-se escolher salvar o arquivo, definindo o nome, com ":write [nome do arquivo]".
Atenção. Os comandos diferenciam minúsculas de maiúsculas.
Caso você esteja no modo normal, pode apertar "ZZ" para salvar e sair. Um atalho para ":wq".


\subsection{Movimentação}
No modo normal fazemos as movimentações de cursor mais avançadas.
Dependendo da sua versão de vim, você pode usar as setas do teclado para se mover, e pode realizar essa movimentação mesmo dentro do modo de inserção de texto.
Vamos ver o básico de movimentação:

% Arranjar um jeito de deixar aquela imagem maneira aqui
%%   ^
%%   k
%%<-h l->
%%   j
%%   v
\subsubsection{Lógica Número-Ação}
Em muitos casos podemos pedir que o vim interprete nossos comandos com um multiplicador.
Isso significa que, ao invés de pressionar "l" para andar para a direita 10 vezes, podemos escrever 10l, e a mágica acontece.
Essa lógica vai nos acompanhar, e inclusive vai ficar mais rebuscada à medida que começamos a misturar conhecimentos.

% Continuar com...
...
\subsection{Modo Visual}
\subsubsection{Modo Visual Normal}
...
\subsubsection{Modo Visual Linha}
...
\subsubsection{Modo Visual Bloco}
...
\subsection{Trocando Caracteres (Replace Mode)}



\subsection{Marks}
...

\newpage
\section{Recursos avançados}
...
\subsection{Registers}
...

\subsection{MultiWindows}
...

\subsection{Sessions}
...
\newpage

\section{Settings}
...
\newpage

\section{Plugins Built-in}
...
\newpage

\section{Escolhendo e gerenciando Plugins}
...
\newpage

\section{Usando o Manual}
...
\newpage

\end{document}
