\chapter{Usando o Manual}
Não fosse eu estar escrevendo em uma ordem bonita, com um jeito de falar cativante,
e com muitas imagens, certamente não valeria a pena ler.
O manual possui tudo o que você precisa.
O manual é a primeira coisa que você encontra ao abrir o vim pela primeira vez.

\insertfigure{scale=1.20}{manual/manualsplashscreen.jpg}{Você já tinha visto, ou até mesmo já abriu o manual.}

O manual te ensina a navegar pelo próprio manual, e inclusive ajuda a fazer buscas mais rebuscadas.
Será que você está procurando um comando que se realiza no modo normal, no modo de inserção, no modo visual?
Chamando pelo manual com \vimcommand{:help}, entramos na seguinte tela:

\insertfigure{scale=0.80}{manual/manualfirstglance.jpg}{O manual mostra inclusive um pouco do básico do vim.}

Se seu terminal estiver habilitado para cores, certamente você perceberá que o arquivo é colorido.
Não é à toa, talvez você se lembre que aprendemos a nos locomover por tags.
Posicionando o mouse em certos termos conseguimos seguir a tag até uma parte específica do manual.

Além de ter tudo o que já falamos neste livro, o manual possui minha página favorita.
Nesta página existe o santo graal.
Não estou brincando.
Use \vimcommand{:help index} e então pesquise por "graal" ou "grail".

Esta última página é especificamente importante por trazer absolutamente todos os atalhos que fazem
parte do vim completo, mas sem plugins.
Ou seja, se você pretende aprender a mexer como um mago supremo, é aqui que vale a pena procurar.
Para cada atalho, certamente existe um pedaço do manual que explica um pouco melhor sobre as ações possíveis de se tomar.

Falta apenas uma coisa. Recomendo sempre aplicar alguma alteração básica ao manual, como a numeração de linhas.
Essa adição é particularmente útil para verificar onde de qual página do manual você está.
E também, caso você seja jogado para a linha 100000 de um documento, você corre o risco de se perder.
São muitas páginas sobre muitos assuntos, todos muito legais e interessantes.
Mas ninguém tem tempo infinito.

No entanto, caso você seja alguém extremamente curioso, veja os arquivos existentes em:
\insertfigure{scale=0.70}{manual/doc.jpg}{Existem muitos, mas muitos arquivos de documentação do vim.}

Para não deixar de citar:
Muitos plugins possuem páginas de manual em seus arquivos.
Certamente existirá um diretório "doc" dentro das pastas \vimkeys{~/.vim/bundle/[plugin]}.
Se algo não estiver certo, ou se precisar de ajuda, procure nestas pastas.
Certamente sua resposta estará lá.

Este capítulo foi super curto, mas serve para deixar claro:
Este não é um guia de referência. é apenas uma introdução avançada.
Isso significa que você deve ser, a essa altura, tão capaz quanto eu mesmo,
a continuar procurando as funcionalidades que fazem sentido para você.
E como não pode faltar, o maior guia do vim é e sempre será seu manual.

\newpage
