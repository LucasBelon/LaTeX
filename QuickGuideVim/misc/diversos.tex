\chapter{Usos Diversos}

\section{Editando Arquivos Remotamente Com Meu Vim}
Se você criou um arquivo de configuração rebuscado, cheio de estilizações de aparência, e deseja aproveitar essa aparência
ao editar arquivos remotamente, saiba que pode utilizar o seu vim para navegar e editar código remotamente através de uma
conexão ssh, usando o comando \vimkeys{\$ vim scp://192.168.0.55/home/user/file.txt}.
Se você pretende navegar e abrir outros arquivos, é recomendável utilizar autenticação via chave ssh.
Se esta não for configurada, a conexão irá pedir a senha a cada interação com os arquivos.
Você não deixará de se autenticar nas vezes que o sistema ssh requisitar, mas ao menos será
automático e praticamente invisível.

Verifique no manual com \vimcommand{:help ssh}.
Também é aconselhável verificar o manual para detalhes mais interessantes sobre o netrw. \vimcommand{:help netrw}.

\section{PairProgramming Com Tmux}
A prática de PairProgramming é o ato de dois desenvolvedores se revezarem na escrita do código.
Teoricamente, haveria um entendimento maior do que está acontecendo, podendo um ajudar o outro,
pesquisando, escrevendo, alternando entre duas ações de duas tarefas diferentes, ou qualquer coisa assim.

Pensando de uma forma mais imediata, bastaria ambos estarem juntos para fazer acontecer.
Mas estamos num mundo conectado.
Se dois desenvolvedores souberem utilizar o vim, pode-se praticar o pair programming a partir de uma conexão ssh.
Utilizando o tmux obtém-se o efeito de ambos controlarem o teclado ao mesmo tempo.

A ferramenta que possibilita essa edição simultânea, dividindo o teclado, é o tmux.
Há quem o utilize para dividir a tela, e até sobrescrevem os atalhos do tmux para que as ações de criar novas janelas
seja semelhante ao do vim.
E ainda há quem diga que o setup de terminal supremo seja o uso do vim com o tmux.

Para nós que sabemos como utilizar o vim para invocar janelas ao nosso gosto e prazer, não precisamos
das funcionalidades avançadas do tmux.
Mas pense bem.
Se alguém começar a usar essas funcionalidades,
e você estiver no contexto do pair-programming, não vai querer ficar para trás.
Aprender coisas novas nunca é um desperdício.

Uma explicação rápida:
O tmux é um multiplexador de terminais. O que significa que a partir de uma única sessão
podemos invocar diversos terminais individuais.
A partir de cada um destes terminais podemos executar diferentes tarefas.
Fazendo desta forma acabamos por criar uma instância de terminal que é resistente à quedas de conexão,
uma vez que estes terminais não são vinculados ao terminal.
Esta característica é o que permite duas pessoas utilizarem, virtualmente, o mesmo terminal ou terminais.

\section{Escrevendo Um Tutorial}
Bom, não deve ser surpresa, mas eu usei o vim para escrever esse pequeno manuél (quem fala manual não é maneiro).
Eu utilizei uma linguagem de marcação, o \LaTeX\ para poder formatar e gerar este pdf.
Em suma, esse texto inteiro foi escrito dentro de um aplicativo de terminal, o vim, com o plugin vimtex.
Se eu fosse experiente poderia ter escrito tudo sem precisar compilar, mas como estou aprendendo,
utilizei um pseudo-terminal, numa interface gráfica, para poder gerar o pdf incrementalmente.
Creio que posso dizer que foi um sucesso.

Como estou chegando ao fim, vou deixar aqui minhas impressões.
Das vantagens de ter utilizado um editor baseado em terminal, se destacou o controle de organização dos arquivos.
Eu gosto de me manter leve, o que significa que eu gosto de guardar apenas os arquivos que são extritamente necessários.
A organização de diretórios facilitou a navegação.
A navegação acelerou as diversas correções que foram necessárias.
E nem a navegação, nem as correções, seriam assim tão fáceis sem o vim.


