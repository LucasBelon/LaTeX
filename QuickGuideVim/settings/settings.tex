\chapter{Settings}
Apesar de termos criado um arquivo de configuração mínimo,
e termos a capacidade de nos manter com um único arquivo para tudo, não é a melhor das ideias.
O vim carrega diversos arquivos de configuração, e termos adicionado um arquivo chamado .vimrc no começo apenas
sobrescreve outras configurações, e adiciona novas.

A lógica das configurações podem variar de acordo com um tipo de arquivo,
com o momento de ser carregado, e podemos criar diversos arquivos para que cada
um seja responsável por um aspecto das alterações que fizermos.

Alguns plugins precisam definir como lidar com a alteração de cor baseada na sintaxe,
para adequar um tema para este arquvo, destacar palavras chave sempre com um certo padrão,
e estruturas de texto concisas em sua apresentação.
Certas configurações podem acabar por tornar nosso editor lento caso sejam implementadas de qualquer forma.
Basicamente, não queremos que um conjunto de configurações que são otimizadas para escrever texto seja usada 
quando estamos editando código.
Manter-se leve também significa manter o programa rodando com boa velocidade e tempo de resposta baixo.

Apenas para recordarmos, estas eram as confiurações definidas até então, para termos o mínimo de ergonomia para o uso simples do vim.

\insertfigure{scale=1.00}{settings/settings_basico.jpg}{Nossa conhecida configuração mínima.}


\section{Onde Estão Minhas Configurações}
O vim percorre o computador atrás de arquivos de configuração.
Ele o faz em uma determinada ordem, a variar também pelo tipo de arquivo.
O vim que estou rodando se trata do vim 8.2, então alguns nomes podem ser diferentes,
além de poder não existirem certos diretórios, apesar de aparecerem como incluídos como adiante.
Podemos ver pela saída de \vimcommand{:echo \&runtimepath}, a partir de um vim com poucas configurações,
quais os diretórios procurados por arquivos de configuração.
\begin{multicols}{2}
\begin{enumerate}
	\item /home/xubu/.vim
	\item /var/lib/vim/addons
	\item /etc/vim
	\item /usr/share/vim/vimfiles
	\item /usr/share/vim/vim82
	\item /usr/share/vim/vimfiles/after
	\item /etc/vim/after
	\item /var/lib/vim/addons/after
	\item /home/xubu/.vim/after
\end{enumerate}
\end{multicols}

\insertfigure{scale=1.30}{settings/etc_vim.jpg}{Os arquivos de configuração iniciais estão nesta pasta.}

Verificando o arquivo \vimkeys{/etc/vim/vimrc} encontramos um arquivo que possuem algumas
recomendações de alterações a serem feitas.
Podemos ver neste arquivo a possibiliade de reverter o modo de compatibilidade com \vimcommand{set nocompatible}.
Ele nos mostra a possibilidade de ligar o \emph{syntax highlighting}, se este puder ser ativado.
Algumas outras opções encontradas são, por exemplo, garantir que o background será sempre de tema escuro,
a opção de abrir o arquivo na mesma posição que da última vez aberto.

\insertfigure{scale=0.90}{settings/explorando_arquivos.jpg}{Existem sugestões de configurações simples nos arquivos padrão de qualquer instalação.}

Logo abaixo destas, vemos as seguintes opções:
\begin{multicols}{3}
\begin{itemize}
	\item \vimcommand{set showcmd}
	\item \vimcommand{set hidden}
	\item \vimcommand{set autowrite}
	\item \vimcommand{set showmatch}
	\item \vimcommand{set ignorecase}
	\item \vimcommand{set smartcase}
	\item \vimcommand{set incsearch}
	\item \vimcommand{set mouse=a}
\end{itemize}
\end{multicols}

\insertfigure{scale=0.90}{settings/outras_configs.jpg}{Os comentários em vimscript são os sinais de $"$}

As duas primeiras opções nós já vimos.
A opção autowrite executa um salvamento automático quando muda-se de buffer,
ou quando se executa um compilador.
Showmatch grifa a busca.
Ignorecase adiciona maiúsculas na busca caso apenas minúsculas forem buscadas,
oferecendo uma busca mais abrangente.
Smartcase desfaz ignorecase caso você utilize maiúsculas e minúsculas em sua busca.
E incsearch realiza a busca a cada inserção de letra, incrementando os resultados à medida que se digita.

Verificando o arquivo \vimkeys{/usr/share/vim/vim82} vemos uma quantidade grande de arquivos de configuração.
Não é mandatório ler todos os arquivos, mas alguns podem chamar a atenção.
Dois conjuntos de arquivos interessantes são, o conjunto dentro do diretório \vimkeys{doc}, que compreende
alguns dos arquivos que são acessados quando pressionamos \vimcommand{:help}.
O outro conjunto interessante reside em \vimkeys{ftplugin}.
Trata-se dos arquivos que configuram o vim de acordo com determinados tipos de arquivos.

\insertfigure{scale=0.90}{settings/runtimepathfiles.jpg}{Existem mais arquivos nos subdiretórios, muitos mais.}

\section{Abreviações}
Abreviações funcionam da seguinte forma.
Você digita uma sequência de texto, e se a sequência de texto bater com alguma abreviação definida, ocorre uma substituição.
Podemos utilizar essa substituição para impedir que escrevamos palavras de forma incorreta, ou podemos definir uma sequência de texto
para disparar outra sequência de texto.
Pense nisso como a capacidade de criar um template.

Essa funcionalidade existe em muitos editores de código, e são chamados snippets.
Trechos de código que são úteis de se poder repetir de forma simplificada.
Criar um arquivo de abreviações que de fato de atenda leva tempo, por isso esperei que conhecêssemos uma forma
de salvar nossas configurações num arquivo.
Agora faz sentido, sem trabalho sendo jogado fora.

Em modo normal, \vimcommand{:abbreviate agua água} irá fazer com que, quando você digitar agua, sua palavra seja imediatamente substituida.
Quando usamos essa funcionalidade no vimrc, geralmente prefere-se utilizar \vimcommand{iabbrev agua água}
para que a abreviação se limite ao modo de inserção.
Temos também o \vimcommand{cabbrev} para modo de comando (ou modo normal), e \vimcommand{!abbrev} para ambos.

Sinceramente, existem formas infinitas de se brincar com essa funcionalidade.
Por conta disso, vou apenas deixar recomandações:
\begin{itemize}
	\item Pesquisar por "abbreviate vim files github".
	\item Procurar pelo plugin abolish, do desenvolvedor tpope.
	\item Pesquisar por formas interessantes de se usar abreviações. Em certos sites existem tutoriais incríveis
\end{itemize}

\section{Salvando Macros}
Falamos de macros, e dissemos que elas poderiam ser salvas em um arquivo de configuração.
Pois falamos do arquivo, e agora basta aprender a sintaxe para inserir corretamente e salvar nossas complexas e poderosas macros.

Abra seu arquivo vimrc, seja ele \vimkeys{~/.vim/vimrc} ou \vimkeys{~/.vimrc}, e insira o seguinte:
\vimcommand{let @A=''}.
Dentro das aspas cole através do registro o que sua macro tiver.
Ou seja, utilize, nesse caso \vimcommand{$"$aP}.

Caso você não tenha criatividade e queira ideias, vou compartilhar as minhas.
Eu mantenho uma macro para abrir, à esquerda,
o navegador de arquivos, e na parte inferior direita, uma instância do terminal.
Para trocar o estilo do contador de linhas, eu mantenho uma macro que salva o comando
\vimcommand{:set relativenumber!}, para que o comando seja executado com apenas duas teclas.
No meio tempo que aprendi a programar em C, usava muito o template padrão do arquivo main.
Por conta disso, gravei uma macro que escreve o template padrão.
Hoje percebo que salvar como abreviação teria sido mais efetivo.

\insertfigure{scale=0.85}{settings/minhas_macros.jpg}{Entender apenas lendo é realmente difícil.}

\section{Olhando um arquivo vimrc pré-pronto}
Um arquivo vimrc é um arquivo que precisa ser alimentado, e mantido.
Isso significa reescrever do zero de vez em quando.
Dicas importantes é deixar anotado o que cada comando faz.
Fazendo dessa forma você evita de se esquecer o motivo de um comando estar onde está.

Eu possuo dois tipos de vimrc.
Um que irá funcionar mais ou menos bem com o vim.tiny, e outro que inclui plugins.
Já deixei claro que nem sempre poderemos utilizar plugins.
Mas no caso de estarmos em nosso computador pessoal, com liberdade para instalar qualquer arquivo,
de qualquer fonte, então apoio completamente essas configurações.
Vamos verificar o que existe no meu vimrc sem plugins.

\insertfigure{scale=0.85}{settings/vimrc.jpg}{Vamos começar a moldar nosso vim.}

\subsection{Configurações set}

A primeira parte força o vim à procurar pelo próprio arquivo,
uma medida de segurança.
Em seguida adiciono o processamento baseado em syntaxe.
Faço com que o vim seja capaz de lidar com arquivos de configuração por tipo de arquivo.
E adiciono a possibilidade de plugins sobrescreverem essas configurações.

\insertfigure{scale=0.85}{settings/set01.jpg}{Tomando contato com o arquivo de configuração.}

A depender da versão de vim podemos ativar o uso do mouse.
Além disso, existem versões que permitem que um registro do vim interaja com o clipboard do sistema.

\insertfigure{scale=0.95}{settings/set02.jpg}{Existem condicionais no vimscript.}

Em sequência, ativo a barra inferior do vim como vimos no capítulo sobre qualidade de vida.
Além de ativar, faço uma pequena estilização.

\insertfigure{scale=0.90}{settings/set03.jpg}{Essas opções possuem sintaxe própria.}

Então adiciono uma configuração para garantir que a tecla backspace irá se comportar como o esperado.
Adiciono a opção number, nocompatible, ignorecase, smarcase, showmatch, hlsearch e incsearch.

\insertfigure{scale=0.85}{settings/set04.jpg}{Lembrando que configuramos e esquecemos. Por isso comentamos.}

Permito que linhas sejam dobradas, limitando o número de níveis que a dobra pode ter.
Também mantenho alguns comandos inativos.
Manter comandos inativos pode ser útil para caso você não seja bom de memória.
É melhor o ter no arquivo, desativado, do que ficar pesquisando por aí.

\insertfigure{scale=0.83}{settings/set05.jpg}{Como esquecemos, é melhor ter a configuração desativada do que não a ter.}

Altero a quantidade de comandos que são salvos na memória do vim, para que eu possa entrar no modo comando e pressionar cima até encontrar um comando interessante.
Em seguida aciono o showcmd e o wildmenu para ter qualidade de vida.

A opção lazyredraw faz com que o vim ganhe um pouco de performance caso exista algo tentando alterar o layout muitas vezes por segundo.
A opção noerrorbells inibe o som que o vim faz quando avançamos ao fim da linha, entre outras ações.
O som é realmente muito chato.

\insertfigure{scale=0.80}{settings/set06.jpg}{Haja configuração.}

Na sequência temos as opções que alteram o caractere tab.
Hoje prefiro que plugins, ou o próprio vim, cuide disso de acordo com cada arquivo.
Portanto, mantenho apenas smartindent ativo.

Na sequência, desativo arquivos swap, que são criados, mas raramente apagados, gerando uma bagunça nos diretórios.
Seleciono uma pasta para serem armazenados arquivos de backup e arquivos que salvam as ações de desfazer do vim.

Então adiciono todos os subdiretórios ao caminho de busca do vim.
Essa opção faz com que eu tenha algumas sugestões mais interessantes quando teclo tab.
Além disso, permito que o plugin built-in do vim se ative.
Ele também funciona sugerindo palavras já utilizadas nos arquivos dos buffers ativos, e outros arquivos.

Matenho algumas configurações desativadas.
Perceba que todas as opções possuem uma breve explicação, mesmo as desativadas.
A partir de então começo a modificar os atalhos do vim para o meu gosto.

Adiciono à variável \vimkeys{path} todos os subdiretórios.
Essa adição pode tornar o vim lento, mas traz uma melhoria no autocomplete.
Ele irá também adicionar certos arquivos ao completamento por \vimkeys{$<$TAB$>$} no modo comando.

Por fim, altero o esquema de cores para um esquema que é comum em qualquer vim.
Não é exatamente o tema mais bonito, mas dos que estão pré-instalados, é o que mais me agrada.
Pesquise por aí sobre temas, existem muitos, e certamente um vai te agradar mais do que o default.

\insertfigure{scale=0.98}{settings/set07.jpg}{Terminamos. Mas agora que está pronto teremos uma ótima experiência.}

\subsection{Criando e substituindo atalhos}
Existem atalhos que funcionam no modo normal, no modo de inserção, e no modo ex, ou modo de \emph{comando}.
Há como tentar mapear uma tecla, mas não concluir a ação caso houver algum mapeamento pré-existente.
E, por fim, há como mapear combinações de teclas para praticamente qualquer coisa.

Em meu vimrc eu apenas remapeio um básico de funções, para facilitar meus movimentos,
para redimensionar as janelas, e para lidar com busca e as famosas folds.

\insertfigure{scale=0.80}{settings/key_binding01.jpg}{Criar atalhos é o que faz o vim ficar na ponta dos seus dedos.}

Meu primeiro mapeamento \vimcommand{nnoremap $<$space$>$ za} me permite
abrir e fechar uma dobra qualquer.
Basicamente é útil quando acontece uma dobra por acidente, para eu não precisar me preocupar com
qual era o comando para fechar.
Na sequência o comando para tirar o grifado que a busca deixa nas palavras.

Quando eu abro múltiplas janelas e quero ser capaz de redimensioná-las, não fico usando comandos.
Deixo quatro atalhos para alterar, em uma unidade por vez, cada uma das dimensões da janela.
Diria que ter algum mapeamento para estas funções seja quase obrigatório.

\insertfigure{scale=0.80}{settings/key_binding02.jpg}{Atalhos úteis aumentam o amor pela ferramenta.}

Ao final, mapeamentos para navegar por abas.
Como existem muitos comandos úteis para mexer com tabs, eu deixo muitos
atalhos configurados.

\newpage
