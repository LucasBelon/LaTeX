\chapter{Settings}
Apesar de termos criado um arquivo de configuração mínimo,
e termos a capacidade de nos manter com um único arquivo para tudo, não é a melhor das ideias.
O vim carrega diversos arquivos de configuração, e termos adicionado um arquivo chamado .vimrc no começo apenas
sobrescreve outras configurações, e adiciona novas.

A lógica das configurações podem variar de acordo com um tipo de arquivo,
com o momento de ser carregado, e podemos criar diversos arquivos para que cada
um seja responsável por um aspecto das alterações que fizermos.

Alguns plugins precisam definir como lidar com a alteração de cor baseada na sintaxe,
para termos temas, palavras chave sempre com um certo padrão, e estruturas de texto
concisas em sua apresentação.

E certas configurações podem acabar por tornar nosso editor lento caso sejam implementadas
de qualquer forma. Basicamente, não queremos que um conjunto de configurações que são otimizadas
para escrever texto seja usada quando estamos editando código.
Manter-se leve também significa manter o programa rodando com boa velocidade e tempo de resposta baixo.

\begin{itemize}
	\item /home/lucas/.vim
	\item /home/lucas/.vim/bundle/Vundle.vim
	\item /home/lucas/.vim/bundle/vim-lsp-settings
	\item /home/lucas/.vim/bundle/vim-lsp
	\item /home/lucas/.vim/bundle/asyncomplete.vim
	\item /home/lucas/.vim/bundle/asyncomplete-lsp.vim
	\item /home/lucas/.vim/bundle/vim-vsnip
	\item /home/lucas/.vim/bundle/vim-vsnip-integ
	\item /home/lucas/.vim/bundle/friendly-snippets
	\item /home/lucas/.vim/bundle/vimtex
	\item /home/lucas/.vim/bundle/vim-spell-pt-br
	\item /usr/share/vim/vimfiles
	\item /usr/share/vim/vim90
	\item /usr/share/vim/vimfiles/after
	\item /home/lucas/.vim/after
	\item /home/lucas/.vim/bundle/Vundle.vim
	\item /home/lucas/.vim/bundle/Vundle.vim/after
	\item /home/lucas/.vim/bundle/vim-lsp-settings/after
	\item /home/lucas/.vim/bundle/vim-lsp/after
	\item /home/lucas/.vim/bundle/asyncomplete.vim/after
	\item /home/lucas/.vim/bundle/asyncomplete-lsp.vim/after
	\item /home/lucas/.vim/bundle/vim-vsnip/after
	\item /home/lucas/.vim/bundle/vim-vsnip-integ/after
	\item /home/lucas/.vim/bundle/friendly-snippets/after
	\item /home/lucas/.vim/bundle/vimtex/after
	\item /home/lucas/.vim/bundle/vim-spell-pt-br/after
\end{itemize}


\section{Onde Estão Minhas Configurações}
\section{Abreviações}
\section{Salvando Macros}
\section{Olhando um arquivo vimrc pré-pronto}
Adicionar aqui questões do omnicomplete.
\newpage

