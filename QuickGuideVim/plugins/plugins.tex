\chapter{Escolhendo e gerenciando Plugins}
Um gerenciador de plugins não é necessário, já que o vim possui uma forma de adicioná-los manualmente.
No entanto, já estamos utilizando um vim completo, com modificações, com atualizações,
além de que não faz sentido esforçar-se para fazer algo que um programa faz por você.

Temos plugins que gerenciam plugins.
É um meta-plugin, que irá tirar de você a responsabilidade por manter os plugins atualizados,
as pastas organizadas, e eventualmente irá disponibilizar um arquivo para verificar quais foram
as mudanças entre uma atualização ou outra.
\section{O Caso do desenvolvedor de software}
Hoje em dia, para desenvolvimento temos sugestões geradas por inteligência artificial,
auxílio de compiladores mais inteligentes,
e ecossistemas de desenvolvimento podendo ser criados e destruídos de forma automática.
Deixar de utilizar ferramentas de auxílio para desenvolvimento é quase o equivalente a se tornar
improdutivo e ineficiente.

O mínimo que um programador moderno irá utilizar é uma IDE. Esta conta com sugestões para complementação
de escrita e informações sobre o que funções fazem, o que objetos e estruturas de dados carregam, além
de programas que sugerem estilística.
A máxima é: você não escreve código para funcionar, você escreve código para que outros entendam,
para que outros editem, melhorem, e não fiquem perdendo tempo decifrando o que foi feito.
Ferramentas são adicionadas geralmente buscando esses princípios.

Antes de haver o modelo de uma espécie de editor de código genérico, como é o vim e o vscode,
em que você adiciona módulos para trabalhar com sua linguagem, todas as ferramentas eram
inclusas no editor principal.
Hoje sabemos que precisamos de um protocolo de servidor de linguagem para que nosso editor possa
interagir com uma espécie de programa que sugere edições de forma inteligente.
Esse servidor de linguagem pode te trazer informações sobre trechos obscuros, identificar erros grosseiros
ou mesmo mostrar qual é o padrão adotado em determinado projeto.
Não é um programa que está dentro do seu editor, mas que o complementa,
e te ajuda a concluir sua tarefa mais rapidamente.

Outro tipo de programa que é útil se especializa na estilística.
Os chamados linters irão verificar inconsistências de uso de nomenclatura e
estrutura geral de código. Podemos citar como exemplos a aplicação de quebras de linha, uso de parênteses,
até mesmo a existência de comentários dentro das funções.
De certa forma, é um programa que lê seu código, aponta onde podem ocorrer melhorias, e que segue apitando
caso você não realize correções.
Esta padronização pode acabar sendo uma dor de cabeça, mas é importante para que se tenha possa trabalhar em equipe,
alcançando um padrão que facilite o entendimento rápido do código, que por sua vez acelera a implementação de melhorias
contínuas, e evolução de projetos de software.
\newpage
