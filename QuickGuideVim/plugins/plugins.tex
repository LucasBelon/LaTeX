% Capítulo 05
\chapter{Plugins, Extensão Infinita}
Um gerenciador de plugins não é necessário, já que o vim possui uma forma de adicioná-los manualmente.
No entanto, já estamos utilizando um vim completo, com modificações, com atualizações,
além de que não faz sentido esforçar-se para fazer algo que um programa faz por você.

\insertfigure{scale=1.20}{plugins/vundle_github.jpg}{A maior parte dos plugins estão alojados no github.}

Temos plugins que gerenciam plugins.
É um meta-plugin, que irá tirar de você a responsabilidade por manter os plugins atualizados,
as pastas organizadas, e eventualmente irá disponibilizar um arquivo para verificar quais foram
as mudanças entre uma atualização ou outra.

\insertfigure{scale=1.20}{plugins/vundle_in_vimrc.jpg}{O Vundle atualiza e organiza os plugins.}

Temos dentre nossas opções:
\begin{multicols}{2}
\begin{itemize}
	\item \emph{vim-plug};
	\item \emph{pathogen};
	\item \emph{dein};
	\item outros.
\end{itemize}
\end{multicols}
Neste tutorial vamos utilizar o \emph{vundle}.

\insertfigure{scale=1.00}{plugins/vimfiles.jpg}{No diretório Bundle estão os plugins do autor.}

Bastou adicionar os repositórios dos plugins no arquivo vimrc com a sintaxe certa que o próprio gerenciador
realizou o download, a atualização, e organizou na pasta correta.
Cada gerenciador possui uma sintaxe própria.

\section{O Caso Do Desenvolvedor De Software}
Hoje em dia, para desenvolvimento, temos sugestões geradas por inteligência artificial,
auxílio de compiladores mais inteligentes,
e ecossistemas de desenvolvimento podendo ser criados e destruídos de forma automática.
Deixar de utilizar ferramentas de auxílio para desenvolvimento é quase o equivalente a se tornar
improdutivo e ineficiente.

O mínimo que um programador moderno irá utilizar é uma IDE.
Esta conta com sugestões para complementação de escrita e informações sobre o
que funções fazem, o que objetos e estruturas de dados carregam, além de
programas que sugerem estilística.
A máxima é: você não escreve código para funcionar, você escreve código para que outros entendam,
editem, melhorem, e não fiquem perdendo tempo decifrando o que foi feito.
Se seu código não funciona você ainda pode deixar na mão de um colega que irá ajudar,
mas um código mal formatado acaba tomando tempo, e ninguém quer ser responsável por tomar
tempo de um colega.
Ferramentas são adicionadas geralmente buscando esses princípios.

Para nosso exemplo, precisaremos de um plugin que simplifica as configurações dos tais servidores.
Na página de cada plugin existem sugestões de configurações, e até mesmo
algum manual ou tutorial. 

\insertfigure{scale=0.75}{plugins/lsp_vimrc.jpg}{Adição de dois plugins e definição de uma função que remapeia atalhos.}

Perceba que existem remapeamento de teclas para que o plugin tome conta de tarefas no lugar do próprio vim.
Como dito anteriormente, é um programa separado, mas integrado ao nosso editor.

Existe um contraste entre um editor genérico, como é o vim e o vscode,
em que você adiciona módulos para trabalhar com sua linguagem, e IDE's,
nas quais todas as ferramentas de auxílio são inclusas no editor principal.

Hoje sabemos que precisamos de um protocolo de servidor de linguagem \vimkeys{(language server protocol - lsp)}
para que nosso editor possa
interagir com uma espécie de programa que sugere edições de forma inteligente.
Esse servidor de linguagem pode te trazer informações sobre trechos obscuros, identificar erros grosseiros
ou mesmo mostrar qual é o padrão adotado em determinado projeto.
Não é um programa que está dentro do seu editor, mas que o complementa,
e te ajuda a concluir sua tarefa mais rapidamente.

Outros plugins auxiliam nosso lsp. Um plugin cria uma janela em pop-up para escolhermos dentre sugestões geradas pelo lsp,
outro traz sugestões pré-definidas genéricas, e outro pode trazer sugestões de acordo com o tipo de arquivo.
Modularidade é a palavra-chave.

\insertfigure{scale=0.75}{plugins/modulos_lsp.jpg}{Podemos adicionar diversos componentes com diversas finalidades.}

Outro tipo de programa que é útil se especializa na estilística.
Os chamados linters irão verificar inconsistências de uso de nomenclatura e
estrutura geral de código. Podemos citar como exemplos a aplicação de quebras de linha, uso de parênteses,
até mesmo a existência de comentários dentro das funções.
De certa forma, é um programa que lê seu código, aponta onde podem ocorrer melhorias, e que segue apitando
caso você não realize correções.
Esta padronização pode acabar sendo uma dor de cabeça, mas é importante para que se tenha possa trabalhar em equipe,
alcançando um padrão que facilite o entendimento rápido do código, que por sua vez acelera a implementação de melhorias
contínuas, e evolução de projetos de software.

\subsection{Utilizando Um LSP Para Python}
Por se tratar de uma linguagem muito simples do ponto de vista da escrita e leitura,
vamos ter nossos exemplos simples em python.
Python é principalmente composto por módulos e bibliotecas.
Para quase tudo que se deseja realizar, há uma biblioteca que faça o serviço.
Vamos explorar como um \vimkeys{lsp} pode nos ajudar a utilizar algum módulo simples.

\insertfigure{scale=0.85}{plugins/python_popups.jpg}{O plugin nos traz um popup com a docstring da função chamada.}

\subsection{Um Exemplo De Linter Para Python}
Python também é um bom exemplo por haver uma espécie de etiqueta para a escrita.
Recomenda-se uma extensão máxima para itens, além de a linguagem se basear na identação
para que sua sintaxe seja corretamente interpretada pelo interpretador.
Portanto, um linter é mais que bem vindo.

\insertfigure{scale=1.25}{plugins/python_teste.jpg}{Temos nossos plugins funcionando já apontando erros de estilo.}

Explicando melhor, um linter é um dedo-duro que analisa o código estático (de certa forma, o texto do código),
à procura de bugs, construções suspeitas, entre outros erros.

\newpage
