\documentclass[a4paper, 12pt]{article}
\usepackage[utf8]{inputenc}
\usepackage[brazil]{babel}
\usepackage{amsmath,
            amssymb,
            amsfonts,
            latexsym,
            mathrsfs,
            amsthm,
            amstext,
            bezier,
            amscd}

\begin{document}

Escrevendo caracteres especiais;
\begin{itemize}
        \item \#
        \item \%
        \item \$
        \item \_ (underscore)
        \item \&
        \item \{
        \item \}
        \item \textbackslash

\end{itemize}

O texto pode ser \emph{enfatizado}
Pode também ser \textit{italico} ou \textbf{Negrito}
E mesmo ser encadeado \textbf{\textit{Italico e Negrito}}
\emph{Algumas formatações possuem \emph{comportamento específico} quando encadeados}
Os comandos de formatação de texto obedecem a sintaxe: 

$\backslash$text**. Os dois asteriscos são referentes aos efeitos:
Vou tentar aqui um negócio. $\backslash$ tentativa
% Backslash é um elemento que precisa estar entre cifrões

\begin{itemize}
    \item{text it: italic - itálico}
        \textit{Exemplo}
    \item{text bf: bold face - negrito}
        \textbf{Exemplo}
    \item{text sl: slanted - xxxxxx}
        \textsl{Exemplo}
\end{itemize}

**Verificar depois a diferença entre vimtex e vim-latex.

% Usando comandos para mudar a fonte de trechos do texto.
\section{\textsf{Fontes sobre \LaTeX na internet.}}

% \textsf Sans-seriff
Um dos melhores lugares para encontrar conteúdo sobre LaTeX é no site CTAN:
\texttt{https://www.ctan.org}
%\texttt{} typewriter text
%\textrm{} texto romano = texto com serifa. Fonte padrão.



Mudando a fonte e família usando declarações.

\section{\sffamily\LaTeX\ resources in the internet}

The best place for downloading LaTeX related software is CTAN.

Its address is \ttfamily http://www.ctan.org\rmfamily.


\begin{tabular}{ c c c }
    Comando         & Declaração  & Significado       \\
    \textrm{...}    & \rmfamily   & roman family      \\
    \textsf{...}    & \sffamily   & sans-serif family \\
    \texttt{...}    & \ttfamily   & tipewriter family \\
    \textbf{...}    & \bfseries   & bold-face         \\
    \textmd{...}    & \mdseries   & medium            \\
    \textit{...}    & \itshape    & italic shape      \\
    \textsl{...}    & \slshape    & slanted shape     \\
    \textsc{...}    & \scshape    & Small Caps Shape  \\
    \textup{...}    & \upshape    & Upright Shape     \\
    \textnormal{...}& \normalfont & Default font      \\
\end{tabular}

Podemos usar chaves paea delimtar o escopo de ação
de uma declaração:

{\sffamily

Text can be {\em emphasized}.

Besides being {\itshape italic} words could be {\bfseries bold},

{\slshape slanted} or typeset in {\scshape Small Caps}.

Such commands can be {\itshape\bfseries nested}.}

{\em See how {\em emphasizing} looks when nested.}

Alterações em tamanho de fonte são usadas em macros.
Mas vamos aplicar no corpo do texto para exemplificar.

\noindent \tiny we \scriptsize start 
\footnotesize \small small
\normalsize get
\large big 
\Large and
\LARGE bigger,
\huge huge and
\Huge gigantic!


\normalsize
O tamanho das fontes sendo alteradas são referentes à fonte padrão do documento.

É possível criar ambientes (environments) para que trechos 
sejam contemplados pelas alterações ditadas.
Se trata de uma envoltória estilizada.

\begin{huge}
    \bfseries
    Um pequeno exemplo

\end{huge}

\bigskip

Este é outro exemplo ilustrativo.


Criamos uma envoltória em que \huge "\normalsize "foi aplicado.
A declaração de negrito acaba delimitada pelo escopo.
Posso aplicar alterações de fonte, família e tamanho com a mesma ferramenta?

A linha vazia antes de \textbackslash end \{huge\} denota fim do parágrafo.
Bigskip nos dá um espaço vertical.
O espaço entre parágrafos é calculado a partir do tamanho 
da fonte, por isso, é recomendável pular parágrafo antes
de trocar o tamanho da fonte.

Podemos criar comandos, exemplo com texto pré definido.
Criar comandos pode simplificar sintaxe e servir para salvar
pre-sets.

\newcommand{\TUG}{TeX Users Group}

\section{The \TUG}

The \TUG\ is an organization for people who are interested in 

\TeX\ or \LaTeX.

\newcommand{\TUG}{\textsc{TeX Users Group}}

O backslash em \TeX\ força o espaço seguinte a aparecer.
Se não, ele será ignorado
(vou ter que ficar conferindo? Meio ruim isso aí).
Posso estilizar o comando também.

Continua no próximo documento. Página 60 do livro.


\end{document}
