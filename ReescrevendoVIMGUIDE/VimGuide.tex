\documentclass[a4paper, 12pt]{article}
\usepackage[utf8]{inputenc}
\usepackage{amsmath,
            amssymb,
            amsfonts,
            latexsym,
            mathrsfs,
            amsthm,
            amstext,
            bezier,
            amscd}
\usepackage[top=2cm,
            bottom=2cm,
            left=2.5cm,
            right=2.5cm]
            {geometry}
\usepackage{indentfirst}
\usepackage{graphicx}
\usepackage[brazil]{babel}


\title{
    \textbf{
    Anotações Definitivas
    }
    \break
    Aprendendo VIM
}
\author{
    Lucas Gouveia Belon
    \\
    lucasgouveiabelon@usp.br
}

\date{\vspace{15cm}São Paulo\\2023}

\begin{document}
\maketitle
%\thispagestyle{empty}
%\setcounter{page}{-1}
\newpage
%\thispagestyle{empty}
\tableofcontents
\newpage
\section{Introdução}
\subsection{Considerações Iniciais}

Para aprender a usar o latex com vim não fica muito claro se é necessário saber antes vim ou tex.
Então eu sugiro aprender vim antes.
Eu mesmo estou aprendendo a mexer no vimtex para poder criar este documento, e não é tarefa exatamente muito fácil.
O uso de uma IDE com certeza simplifica muito as coisas, mas não é sempre que temos máquinas potentes o suficiente para isso.

O pacote de ferramentas que vem com o \LaTeX \hspace{0.1cm} é gigantesco em variedade e em peso.
São alguns gigabytes de arquivos, muitas fontes, e diferentes compiladores.
Até assusta, mas vamos com calma que chegamos lá.

Este pequeno projeto é uma forma de eu me ambientar com o vimtex, e se ajudar alguém, é um ponto extra.
Os textos que já tenho preparados foram preparados em inglês, mas sinto falta de textos em português, para um acesso mais fácil e para disponibilização para amigos e quem mais tenha interesse.

Para exemplificar o VIM tenho uma variedade de arquivos, mas talvez valha a pena começar explicando suas partes componentes.
Dessa forma, pode-se ter uma ideia de quais funcionalidades que a ferramenta tem a oferecer, e quais os atrativos para aprender a usar algo que possui uma curva de aprendizagem tão íngrime.


\subsection{Componentes}
\subsubsection{O Editor}
O Vim é um programa com objetivo de se possibilitar a edição de textos via linha de comando.
Portanto, em sua mais simples natureza, trata-se de uma espécie de bloco de notas.
Mas acontece que esse bloco de notas nasceu nos anos 90, e se não morreu até agora, pelo menos deve ter tomado uns esteróides.


O Vim funciona de forma diferente de outros editores de texto, possuindo modos de execução de comandos, inserção de textos, edição de linhas, de blocos, de caracteres, entre outros modos.
Os modos principais são o modo normal, de inserção, e visual.

\subsubsection{Modos de Operação}
\begin{itemize}
    \item \textbf{Modo Normal:}
        Este modo se trata da disposição relaxada do vim.
        Nele podemos alternar para outros modos, e também se trata do modo em que podemos nos movimentar com o cursor da forma mais eficiente.
        Há quem diga que os comandos de movimentação do vim seja o aprendizado mais importante.
 

        Também pode-se executar edições no próprio vim, mudando seu comportamento padrão, e adicionando funcionalidades malucas, que chega a fazer os olhos brilharem.

    \item \textbf{Modo Visual:}
        Neste, temos como selecionar pedaços e trechos de texto, tanto para edição, como para cópia, para se excluir, ou mesmo para servir de alvo numa determinada sequência de ações.
        Existem na verdade 3 modos visuais.
        O modo caractere, o modo linha e o modo bloco.
        O primeiro seleciona trechos numa linha ou mais.
        O segundo seleciona linhas inteiras.
        O terceiro é ideal para selecionar em tabelas, ou em textos escritos de forma tabular, como é o caso de matrizes.


    \item \textbf{Modo de Inserção:}
        Se você não é programador, e sim um escritor de qualquer natureza, esse será o modo em que passará mais tempo.
        Não há o que explicar, este modo insere texto no arquivo.
        Este modo conta com maneiras de se evitar que se escreva errado, corrigindo enquanto se escreve, sugestões se palavras que já apareceram no texto anteriormente, e até mesmo pode realizar buscas em dicionários.

\end{itemize}

\subsubsection{Configurações}
A força do VIM está em sua alta capacidade de configuração.
Ele pode possuir configurações gerais, configurações que dependem do tipo de arquivo, configurações que dependem de você ativar.
E como todo bom sistema de configuração, pode-se criar configurações em um único arquivo, ou em múltiplos.

Por hora vamos focar no formato de arquivo de configuração única.

\begin{itemize}
    \item \textbf{.vimrc:}
        Uma das opções para configuração é armazenar todas as alterações em relação ao editor padrão num arquivo chamado .vimrc.
        Este arquivo será lido cada vez que o editor for aberto, e pode estar localizado em diferentes locais de sua máquina:
        \subitem{/home/user/.vim/vimrc}: Um diretório oculto no \$HOME do usuário.
        \subitem{/home/user/.vimrc}: Um arquivo no \$HOME do usuário.
        % Abrir o VIMLEEREN para verificar quais são os locais percorridos na busca por arquivos de
        % configuraçao do vim
        \subitem{local3}: 

        Ignorando tudo isso, basta saber que pode-se configurar o editor e levar suas configurações consigo para qualquer lugar.
        Dessa forma, pode-se moldar o programa para funcionar da maneira que se queira, e caso encontre uma máquina desconfigurada, basta carregar seu arquivo.
        Faça do VIM o \textbf{seu} editor de textos.

        % DEU MERDA AQUI NO SÍMBOLO <>. ARRUMAR MAIS TARDE.
    \item \textbf{O Comando ":set <options>":}
        Às vezes você se encontrará usando o editor numa máquina desconfigurada, e provavelmente irá preferir deixar da forma que estiver.
        Mas existem certas configurações que quando inexistentes, farão falta no fluxo de atividade.

    \item \textbf{Plugins:}
        Para muitas coisas que fazemos na vida, sabemos que alguém já fez, já fez melhor, e ainda deixou rastros de como repetir o feito de forma mais eficiente.
        Esse é o caso de arquivos .vimrc que são disponibilizados livremente, mas existem também arquivos de configurações que extendem as funcionalidades do VIM para lugares não antes percorridos.

        Esse é o caso dos plugins. Eles existem em várias formas, tamanhos, pesos, cores, com as mais diversas funcionalidades.
        São de tal forma que existem plugins adicionados ao núcleo do vim, com funcionalidades do tipo busca de arquivos, edição sobre ssh, leitura e escrita de arquivos comprimidos, edição por rede, dentre outroso. 
        Existem plugins para baixar e gerenciar plugins, e este deve ser o primeiro plugin a ser encontrado quando se deparar com a necessidade de adicionar ao seu vim, funcionalidades mais potentes.

        Para concluir o pensamento, existem plugins que fazem com que seu humilde editor de textos se torne tão poderoso quanto um ambiente integrado de desenvolvimento, que é o que muitos programadores fazem.
\end{itemize}

\subsection{Casos De Uso}
Creio que já ficou claro quais são os usos principais do VIM, mas ainda assim, devo citar alguns extras.
A maneira que se utiliza o vim faz mudar a forma de pensamento, te deixando claramente mais metódico, e forçando a pensar antes de clicar enter.

Aprender a usar uma ferramenta que funciona em seu poder máximo dentro da interface de linha de comando te forçará, obviamente, a se manter dentro da linha de comando, e te forçará a aprender mais sobre programação.
Além de te forçar a refletir sobre os conceitos mais básicos que a operação de computadores exige, é possível que seja o ponto de inflexão para perder o receio de se manter em um ambiente outrora hostil.

Por fim, eu estou escrevendo um arquivo em pdf, quase que diretamente, a partir de um editor de texto estilo bloco de notas. 
Ele tem numeração de linha, sugestão de termo, e como se trata de \LaTeX, até mesmo pequenos trechos complexos, os chamados \textit{snippets}.

Não se trata de algo feito para você programar.
Não se trata de algo feito para edição de pdf.
Apenas edição de texto. Mas com funcionalidades gigantescas.
Gosto de dizer que o VIM é famoso forte e feio, muito feio. 
Mas é integrável com outras ferramentas e extensível para obter outras funcionalidades.

\subsection{Outros Tipos de VIM}
\subsubsection{Neovim}
\subsubsection{Astrovim}
\subsubsection{Lazyvim}

% Não perder o foco e não encher muita linguiça.
% Vou mesclar o tutorial de quickguide com o uma explicação mais rigorosa. É literalmente escrever um livro, mas não posso ficar enrolando muito nas explicações. É sempre melhor ser direto ao ponto

\end{document}
