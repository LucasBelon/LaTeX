\documentclass[a4paper, 12pt]{article}
\usepackage[utf8]{inputenc}
\usepackage{amsmath,
            amssymb,
            amsfonts,
            latexsym,
            mathrsfs,
            amsthm,
            amstext,
            bezier,
            amscd}
\usepackage[top=2cm,
            bottom=2cm,
            left=2.5cm,
            right=2.5cm]
            {geometry}
\usepackage{indentfirst}
\usepackage{graphicx}
\usepackage[brazil]{babel}


\title{
    \textbf{
    Anotações Definitivas
    }
    \break
    Aprendendo VIM
}
\author{
    Lucas Gouveia Belon
    \\
    lucasgouveiabelon@usp.br
}

\date{\vspace{15cm}São Paulo\\2023}

\begin{document}
\maketitle
%\thispagestyle{empty}
%\setcounter{page}{-1}
\newpage
%\thispagestyle{empty}
\tableofcontents
\newpage
\section{Introdução}
\subsection{Considerações Iniciais}

Para aprender a usar o latex com vim não fica muito claro se é necessário saber antes vim ou tex.
Então eu sugiro aprender vim antes.
Eu mesmo estou aprendendo a mexer no vimtex para poder criar este documento, e não é tarefa exatamente muito fácil.
O uso de uma IDE com certeza simplifica muito as coisas, mas não é sempre que temos máquinas potentes o suficiente para isso.

O pacote de ferramentas que vem com o \LaTeX \hspace{0.1cm} é gigantesco em variedade e em peso.
São alguns gigabytes de arquivos, muitas fontes, e diferentes compiladores.
Até assusta, mas vamos com calma que chegamos lá.

Este pequeno projeto é uma forma de eu me ambientar com o vimtex, e se ajudar alguém, é um ponto extra.
Os textos que já tenho preparados foram preparados em inglês, mas sinto falta de textos em português, para um acesso mais fácil e para disponibilização para amigos e quem mais tenha interesse.

Para exemplificar o VIM tenho uma variedade de arquivos, mas talvez valha a pena começar explicando suas partes componentes.
Dessa forma, pode-se ter uma ideia de quais funcionalidades que a ferramenta tem a oferecer, e quais os atrativos para aprender a usar algo que possui uma curva de aprendizagem tão íngrime.


\subsection{Componentes}
\subsubsection{O Editor}
O Vim é um programa com objetivo de se possibilitar a edição de textos via linha de comando.
Portanto, em sua mais simples natureza, trata-se de uma espécie de bloco de notas.
Mas acontece que esse bloco de notas nasceu nos anos 90, e se não morreu até agora, pelo menos deve ter tomado uns esteróides.


O Vim funciona de forma diferente de outros editores de texto, possuindo modos de execução de comandos, inserção de textos, edição de linhas, de blocos, de caracteres, entre outros modos.
Os modos principais são o modo normal, de inserção, e visual.

\subsubsection{Modos de Operação}
\begin{itemize}
    \item \textbf{Modo Normal:}
        Este modo se trata da disposição relaxada do vim.
        Nele podemos alternar para outros modos, e também se trata do modo em que podemos nos movimentar com o cursor da forma mais eficiente.
        Há quem diga que os comandos de movimentação do vim seja o aprendizado mais importante.
 

        Também pode-se executar edições no próprio vim, mudando seu comportamento padrão, e adicionando funcionalidades malucas, que chega a fazer os olhos brilharem.

    \item \textbf{Modo Visual:}
        Neste, temos como selecionar pedaços e trechos de texto, tanto para edição, como para cópia, para se excluir, ou mesmo para servir de alvo numa determinada sequência de ações.
        Existem na verdade 3 modos visuais.
        O modo caractere, o modo linha e o modo bloco.
        O primeiro seleciona trechos numa linha ou mais.
        O segundo seleciona linhas inteiras.
        O terceiro é ideal para selecionar em tabelas, ou em textos escritos de forma tabular, como é o caso de matrizes.


    \item \textbf{Modo de Inserção:}
        Se você não é programador, e sim um escritor de qualquer natureza, esse será o modo em que passará mais tempo.
        Não há o que explicar, este modo insere texto no arquivo.
        Este modo conta com maneiras de se evitar que se escreva errado, corrigindo enquanto se escreve, sugestões se palavras que já apareceram no texto anteriormente, e até mesmo pode realizar buscas em dicionários.

\end{itemize}

\subsubsection{Configurações}
A força do VIM está em sua alta capacidade de configuração.
Ele pode possuir configurações gerais, configurações que dependem do tipo de arquivo, configurações que dependem de você ativar.
E como todo bom sistema de configuração, pode-se criar configurações em um único arquivo, ou em múltiplos.

Por hora vamos focar no formato de arquivo de configuração única.

\begin{itemize}
    \item \textbf{.vimrc:}
        Uma das opções para configuração é armazenar todas as alterações em relação ao editor padrão num arquivo chamado .vimrc.
        Este arquivo será lido cada vez que o editor for aberto, e pode estar localizado em diferentes locais de sua máquina:
        \subitem{/home/user/.vim/vimrc}: Um diretório oculto no \$HOME do usuário.
        \subitem{/home/user/.vimrc}: Um arquivo no \$HOME do usuário.
        % Abrir o VIMLEEREN para verificar quais são os locais percorridos na busca por arquivos de
        % configuraçao do vim
        \subitem{local3}: 

        Ignorando tudo isso, basta saber que pode-se configurar o editor e levar suas configurações consigo para qualquer lugar.
        Dessa forma, pode-se moldar o programa para funcionar da maneira que se queira, e caso encontre uma máquina desconfigurada, basta carregar seu arquivo.
        Faça do VIM o \textbf{seu} editor de textos.

        % DEU MERDA AQUI NO SÍMBOLO <>. ARRUMAR MAIS TARDE.
    \item \textbf{O Comando ":set <options>":}
        Às vezes você se encontrará usando o editor numa máquina desconfigurada, e provavelmente irá preferir deixar da forma que estiver.
        Mas existem certas configurações que quando inexistentes, farão falta no fluxo de atividade.

    \item \textbf{Plugins:}
        Para muitas coisas que fazemos na vida, sabemos que alguém já fez, já fez melhor, e ainda deixou rastros de como repetir o feito de forma mais eficiente.
        Esse é o caso de arquivos .vimrc que são disponibilizados livremente, mas existem também arquivos de configurações que extendem as funcionalidades do VIM para lugares não antes percorridos.

        Esse é o caso dos plugins. Eles existem em várias formas, tamanhos, pesos, cores, com as mais diversas funcionalidades.
        São de tal forma que existem plugins adicionados ao núcleo do vim, com funcionalidades do tipo busca de arquivos, edição sobre ssh, leitura e escrita de arquivos comprimidos, edição por rede, dentre outroso. 
        Existem plugins para baixar e gerenciar plugins, e este deve ser o primeiro plugin a ser encontrado quando se deparar com a necessidade de adicionar ao seu vim, funcionalidades mais potentes.

        Para concluir o pensamento, existem plugins que fazem com que seu humilde editor de textos se torne tão poderoso quanto um ambiente integrado de desenvolvimento, que é o que muitos programadores fazem.
\end{itemize}

\subsubsection{Configurando Máquinas Alheias}
Apesar de não ficar muito claro sem antes se possuir certa experiência, um editor de texto em máquinas remotas é algo extremamente necessário.
Saber utilizar, de forma eficiente, e talvez mais importante, sem adição de grandes configurações, um editor, pode ser o que te difere de quem está desenvolvendo um programa, de quem está inserindo o programa numa cadeia de execução.

Ser aquele que consegue controlar e configurar conexões relacionadas a segurança, à disponibilidade, e ser capaz de visualizar, buscar, filtrar e transformar dados em informação, é uma habilidade interessante para administradores de sistemas.
Saber utilizar o VIM te dá esta capacidade. E saber utilizá-lo sem configurações adicionais te capacita a editar servidores. E no mundo em nuvem, esta deixa de ser um diferencial para se tornar um requisito básico.

Além disso, por se tratar de um programa que existe por padrão em diversas máquinas unix e por ser extremamente leve e veloz no que se propõe, pode-se utilizar virtualmente máquinas com configurações que originalmente seriam impossíveis de se obter qualquer vantagem de desempenho.
Estou falando de utilizar uma máquina velha para aprender a fazer coisas incríveis.
Isso agrega valor, não pela máquina ser potente, mas pelo usuário ser capaz de extrair mais de um mesmo pedaço de metal.
\subsection{Casos De Uso}
Creio que já ficou claro quais são os usos principais do VIM, mas ainda assim, devo citar alguns extras.
A maneira que se utiliza o vim faz mudar a forma de pensamento, te deixando claramente mais metódico, e forçando a pensar antes de clicar enter.

Aprender a usar uma ferramenta que funciona em seu poder máximo dentro da interface de linha de comando te forçará, obviamente, a se manter dentro da linha de comando, e te forçará a aprender mais sobre programação.
Além de te forçar a refletir sobre os conceitos mais básicos que a operação de computadores exige, é possível que seja o ponto de inflexão para perder o receio de se manter em um ambiente outrora hostil.

Também vale salientar que é possível utilizar esta ferramenta em um emulador de terminal que funciona baseado em aplicativos.
Certamente seu celular tem tal aplicativo em sua loja, e caso você tenha vontade de aprender a programar, esta pode se tornar sua ferramenta primária.

Claro, começando de forma humilde, com dificuldades extras, mas falando com a experiência que tive, a atividade de aprender a programar pelo celular, escrevendo numa tela preta, me forçou a refletir muito mais antes de qualquer rajada de teclas que eu poderia dar.
Por isso, acabei por pensar muito mais do que fazer, e fazendo o necessário por completo.
Inserir-se em um contexto desafiador pode acabar desenvolvendo noções e habilidades inesperadas.

Por fim, eu estou escrevendo um arquivo em pdf, quase que diretamente, a partir de um editor de texto estilo bloco de notas. 
Ele tem numeração de linha, sugestão de termo, e como se trata de \LaTeX, até mesmo pequenos trechos complexos, os chamados \textit{snippets}.

Não se trata de algo feito para você programar.
Não se trata de algo feito para edição de pdf.
Apenas edição de texto. Mas com funcionalidades gigantescas.
Gosto de dizer que o VIM é o famoso forte e feio, muito feio. 
Mas é integrável com outras ferramentas e extensível para obter outras funcionalidades.
Ele é extremamente feio, mas à medida que fica feio, fica forte.
E você pode utilizá-lo para fazer a força por você.

\subsection{Outros Tipos de VIM}
Existem programas desenvolvidos baseados no VIM "original", pré-configurados, com adição de plugins e layouts que, para adicionar ao vim, seriam necessários dias de configuração.

Vendo isso como uma demanda, e tendo este como um problema próprio, seja pelo motivo que for, programadores criaram versões de VIM que os atendem de forma melhor.
\subsubsection{Neovim}
Surgindo como competidor direto, sendo desenvolvido por um número maior de desenvolvedores, e adicionando novas funcionalidades uma atrás da outra, temos o neovim.
Sua proposta é ser um editor com maior capacidade de ser moldado.
A primeira ação foi refatorar a base de código do VIM para manutenção, e extensão.
Também adicionou linguagens de programação exteriores para criação de plugins, sendo muito mais amigável.
\subsubsection{Astrovim}
Como o neovim trouxe um novo patamar para a característica de extensibilidade, logo surgiram projetos que possuem configurações para programação e ferramentas auxiliares para desenvolvimento.
Isto por padrão, ou seja, sem precisar cutucar configurações e ver como reage.
\subsubsection{Lunarvim}
O objetivo destas versões parecem estar se focando para se tornarem IDE's (Integrated Development Environment), para que programadores possam usar ferramentas que funcionam de forma mais responsiva, sem travamentos, sem adição de peso em suas máquinas, e sem perda de tempo desnecessária.
É uma grande frente de desenvolvimento, devendo eu citar aqui, e você devendo explorar e verificar se assim o quiser.

\subsection{Conclusão}
Até então o texto desenvolve algumas noções gerais do que é, como funciona, e quais as variantes do editor de texto baseado em linha de comando.
Claro que os pontos mais atrativos dessa ferramenta só são apreciáveis se você leitor possuir como atividade algo relacionado a escrita, desenvolvimento, configuração de máquinas e servidores.

Caso não o tenha, é aqui que você deve perceber que este livro não te apetece nem engrandece.
Mas caso contrário, então seremos amigos daqui pra frente, e irei te expor à informações caçadas e lapidadas de diversas fontes.
Informações condensadas em uma espécie de guia, que antes da metade, te dará a capacidade de andar com as próprias pernas.
Mas mais que isso, eu gostaria de ter um guia desta natureza, e portanto, é também um guia para mim.
Venha comigo, por hora, eu serei seu guia.

%ADICIONAR AQUI O COMEÇO DO TUTORIAL

\section{Introdução Prática}
\subsection{Salvar e Sair}
Uma espécie de piada interna do vim diz: "Uma vez eu abri o vim, e nunca mais saí... Sério, me ajuda, como sai do vim?".
Existe certa verdade nisso.
Quando abrimos o vim pela primeira vez, nos deparamos com o modo normal, que foi explicado anteriormente.
Ele é feito principalmente para execução de comandos e para movimentação no texto.
Não para inserir, como fazemos quando abrimos um aplicativo qualquer.
A depender da versão, não existe sequer o botão de backspace para apagar caracteres, nem funcionalidades baseadas em mouse.

Para abrir o vim, no terminal, basta escrever \textbf{\textit{\$vim}}

% Inserir imagem da tela inicial do vim
% Arrumar um conteiner basicamente em branco para adicionar as configurações aos poucos do vim.

Mas temos a vantagem de que, se você não clicou em nenhum botão, muito menos os botões \textbf{\textit{a; i; v ou r}}, para sair é simples.
Aperte \textbf{\textit{":"}} para entrar no modo \textbf{\textit{"ex"}}. Em seguida, escreva \textbf{\textit{"quit"}} para sair.
Se entrou em algum modo estranho, aperte \textbf{\textit{esc}} repetidamente até voltar para o modo normal.
Caso tenha inserido algum texto, o programa irá reclamar que não foi especificado um nome de arquivo.
Então deve-se fazer \textbf{\textit{:write (nome do arquivo)}} ou \textbf{\textit{:wq (nome do arquivo)}}.

O vim aceita, e incentiva, abreviações, de forma que praticamente todos os comandos possuem a forma extensa e a forma abreviada para serem chamados.
Além disso, existem teclas de atalho que realizam ações de certos comandos.
Um detalhe a se perceber é que, teclar \textbf{\textit{"a"}} é diferente de teclar \textbf{\textit{"A"}}.
O atalho da vez será o \textbf{\textit{"ZZ"}}, que salva e sai do arquivo.

Uma parte das informações que estarão neste texto podem ser praticadas no \textbf{\textit{\$vimtutor}}, um tutorial interativo com explicações e exemplos para praticar.
Recomanda-se testá-lo, mesmo que deixemos aqui as mesmas informações. A prática fará você criar intimidade com a ferramenta.

\subsection{Movimentando-se}
\subsubsection{No modo normal}
Saber mudar de linha, de coluna, de parágrafo, de página, é essencial.
Todas as movimentações avançadas são feitas no modo normal.
O básico é mover-se um caractere por vez.
A tecla \textbf{\textit{h}} move o cursor para a esquerda.
A tecla \textbf{\textit{j}} move para baixo.
A tecla \textbf{\textit{k}} move para a cima.
A tecla \textbf{\textit{l}} move para a direita.
% Como inserir o diagrama k
%                        h+l
%                         k
% Aqui no LaTeX?

\subsubsection{No modo visual}
\subsubsection{No modo visual-linha}
\subsubsection{No modo visual-bloco}
\subsection{Movendo a Tela}
\subsection{Deletando}
\subsection{Procura em Texto}
\subsection{Copiar, Desfazer e Salvar, folds}
\section{Slot de memória do VIM}
\subsection{Macros}
\subsection{Marks}
\subsection{Registers}
\subsection{Abbreviations}
\section{Alguns truques do modo normal}
\subsection{Command mode}
\subsection{Usando o quickfix como busca de projeto}
\subsection{Inserindo texto de arquivos externos}
\subsection{Compilando programas diretamente do vim}
\subsection{Executando scripts}
\section{Edição em múltiplos arquivos}
\subsection{Buffer}
\subsection{Tabs}
\subsection{Splitted Windows}
\subsection{MultiSplit Manipulation}
\subsection{Usando scrollbind para ver um mesmo arquivo estreito em duas janelas}
\subsection{Sessions}
\subsection{Mark of a Session}
\subsection{View}
\subsection{Diff}
\subsection{Modelines}
\section{Settings}
\subsection{Some nice options}
\subsection{Mapping de botões}
\subsection{Where to put your saved Macros}
\subsection{Installing Plugins}
\subsection{The vim Manual}
\subsection{The index.txt manual}
\section{Plugins}
\subsection{Plugins Built-in}
\subsubsection{netrw e conexão ssh}
\subsection{Recomendação de Plugins}
\subsubsection{Ler com cuidado antes de sair usando}

% Não perder o foco e não encher muita linguiça.
% Vou mesclar o tutorial de quickguide com o uma explicação mais rigorosa. É literalmente escrever um livro, mas não posso ficar enrolando muito nas explicações. É sempre melhor ser direto ao ponto

\end{document}
