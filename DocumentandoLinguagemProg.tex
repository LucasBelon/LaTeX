% Arquivo exemplo para documentação de linguagem de programação.
% Encontrar outros pacotes capazes de fazer o mesmo, mas de forma mais elegante
% Requerindo menos configurações.
\documentclass[a4paper]{article}

\usepackage[utf8]{inputenc}    
%\usepackage[brazil]{babel}
\usepackage{listings}
\usepackage{color}
 
\definecolor{dkgreen}{rgb}{0,0.6,0}
\definecolor{gray}{rgb}{0.5,0.5,0.5}
\definecolor{mauve}{rgb}{0.58,0,0.82}
 
\lstset{
  language=Python,                
  basicstyle=\footnotesize,           
  numbers=left,                   
  numberstyle=\tiny\color{gray},  
  stepnumber=2,                             
  numbersep=5pt,                  
  backgroundcolor=\color{white},    
  showspaces=false,               
  showstringspaces=false,         
  showtabs=false,                 
  frame=single,                   
  rulecolor=\color{black},        
  tabsize=2,                      
  captionpos=b,                   
  breaklines=true,                
  breakatwhitespace=false,        
  title=\lstname,                               
  keywordstyle=\color{blue},          
  commentstyle=\color{dkgreen},       
  stringstyle=\color{mauve},     
}

\begin{document}

    \section*{Como citar  código fonte no \LaTeX}
    Para citar um código fonte geralmente utilizamos o pacote \emph{verbatim}. Mas também existe o pacote                 \emph{listing} que deixa o código fonte citado com a estrutura de um código em um compilador.

        \subsection*{Python}
        Como gerar senhas aleatórias
            \begin{lstlisting}

import random
import string
import time

def mkpass(size=16):
    chars = []
    chars.extend([i for i in string.ascii_letters])
    chars.extend([i for i in string.digits])
    chars.extend([i for i in '\'"!@#$%&*()-_=+[{}]~^,<.>;:/?'])

    passwd = ''

    for i in range(size):
        passwd += chars[random.randint(0,  len(chars) - 1)]

        random.seed = int(time.time())
        random.shuffle(chars)

    return passwd

           \end{lstlisting}
\end{document}
