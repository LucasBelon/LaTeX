\chapter{Tabelas e imagens}
Tabelas e imagens são elementos que não fazem exatamente parte da lógica de um
texto corrido. Não existe parágrafo e linha do jeito que existe no texto. Por
conta disso, devemos encarar como uma inserção de um elemento geralmente
retangular, com certas dimensões, e que deve se comportar de uma ou outra
maneira diferente. 


\section{Tabelas}
Tabelas no \LaTeX\ são entidades que precisam se entendidas em duas etapas.
Existe o formato de tabela, que se refere a um frame, com certo espaçamento
entre as células, com certa disposição no corpo do texto, e seu número de
linhas e colunas. Uma tabela propriamente dita possui contorno entre os
elementos, geralmente um título, e eventualmente uma legenda.

Tabelas são uma forma de se expressar informação. Sua principal virtude é
organizar, categorizar, e exibir, de forma sucinta, dados. Uma boa tabela deve
conter um bom relacionamento lógico entre os itens, em seu ordenamento e sua
apresentação.


\subsection{tabbing}
Para alinhar texto em colunas no tempo das máquinas de escrever era usado o que
se chama de tabstop. Por causa disso, temos no LaTeX um ambiente chamado
"tabbing".

\noindent\textbackslash begin\{tabbing\}\\
	\textbackslash emph\{info: \} \textbackslash = Software \textbackslash = : \textbackslash = \textbackslash LaTeX \textbackslash \textbackslash \\
	\textbackslash $>$ Author \textbackslash $>$ : \textbackslash $>$ Leslie Lamport \textbackslash \textbackslash \\
	\textbackslash $>$ Website \textbackslash $>$ : \textbackslash $>$ www.latex-project.org \textbackslash \textbackslash \\
\textbackslash end\{tabbing\}\\

O Código acima irá gerar o seguinte texto:

\begin{tabbing}
	\emph{info: } \= Software \= : \= \LaTeX \\
	\> Author \> : \> Leslie Lamport \\
	\> Website \> : \> www.latex-project.org \\
\end{tabbing}

Fica visível a organização entre colunas.
Na primeira linha são usados os caracteres "\textbackslash = ", que servem para
definir um tabstop. Basicamente, é criada uma linha vertical imaginária que irá
ditar qual a distância deve ser pulada num pressionar da tecla tab.

Como estamos em \LaTeX, precisamos inserir um caractere como o tab com um
comando. Na segunda e terceira linha de nossa tabela temos o uso do next tab
stop, próximo tabstop, que basicamente avança o texto na distância definida na
primeira linha.

Por fim, pulamos a linha usando \textbackslash \textbackslash. Essa é uma
sintaxe já conhecida. O que criamos ainda não se trata de uma tabela, mas sim
um texto tabularizado. No entanto, a lógica já começa a se expressar. É preciso
pensar em como os espaços são definidos, e como podem ser definidos.

Esse formato de texto nos permite criar colunas com texto alinhado à esquerda.
Mas e se o texto alcançar os limites da página, tanto para a borda externa
quanto para a base da página? É o que veremos, e como consertar caso algo dê
errado.

\noindent\textbackslash newcommand\{\textbackslash head\}[1]\{\textbackslash textbf\{\#1\}\}\\
\\
\textbackslash begin\{tabbing\}\\
	Family \textbackslash = \textbackslash verb$|$\textbackslash textrm\{...\}$|$ \textbackslash = \textbackslash head\{Declaration\} \textbackslash = \textbackslash kill\\
	\textbackslash head\{Type\} \textbackslash $>$ \textbackslash head\{Command\} \textbackslash $>$ \textbackslash head\{Declaration\} \textbackslash $>$ \textbackslash head\{Example\} \textbackslash \textbackslash \\
	Family \textbackslash $>$ \textbackslash verb$|$\textbackslash textrm\{...\}$|$ \textbackslash $>$ \textbackslash verb$|$\textbackslash rmfamily$|$\\
	\textbackslash $>$ \textbackslash rmfamily Example text \textbackslash \textbackslash \\
	\textbackslash $>$ \textbackslash verb$|$\textbackslash textsf\{...\}$|$ \textbackslash $>$ \textbackslash verb$|$\textbackslash sffamily$|$\\
	\textbackslash $>$ \textbackslash sffamily Example text \textbackslash \textbackslash \\
\textbackslash end\{tabbing\}\\

% Novo comando que serve apenas pra melhorar a sintaxe.
\newcommand{\head}[1]{\textbf{#1}}

\begin{tabbing}
	Family \= \verb|\textrm{...}| \= \head{Declaration} \= \kill
	\head{Type}\> \head{Command} \> \head{Declaration} \> \head{Example} \\
	Family \> \verb|\textrm{...}| \> \verb|\rmfamily|
	\> \rmfamily Example text \\
	\> \verb|\textsf{...}| \> \verb|\sffamily|
	\> \sffamily Example text \\
\end{tabbing}

Este é mais difícil de compreender. A primeira linha cria o espaçamento, e o
comando kill apaga a linha. No entanto, o espaçamento segue definido. Na
segunda linha estamos criando já os títulos das colunas. Na terceira e quarta
linha estamos criando a primeira linha da tabela. Na quinta e sexta linha
estamos criando a última. O segredo está em entender o que está acontecendo nas
duas primeiras linhas e como funciona a lógica do espaçamento.

Detalhe: usamos o comando \textbackslash verb$|$\textbackslash textsf\{...\}$|$
que faz com que o LaTeX escreva o que tiver entre os $|$ de forma como é,
desabilitando qualquer função que possa ser escrita ali.
O nome disso é verbatim. Para uso mais extenso, usar o ambiente \emph{verbatim}. 


Outros comandos úteis são:
\begin{itemize}
	\item \textbackslash + no fim da linha faz a linha subsequente começar na primeira tabulação.
	\item \textbackslash - cancela um \textbackslash + anterior. O uso de múltiplos desses é cumulativo.
	\item \textbackslash $<$ no começo da linha cancela o efeito de um \textbackslash + anterior nesta linha.
\end{itemize}


\subsection{tabular}
Para o caso de precisarmos de estruturas com mais opções, como centralizar na
célula, dividir linhas, e outras opções que só imagino, vamos subir o degrau da
complexidade e usar um novo ambiente.

O ambiente tabular permite exibir tabelas propriamente ditas, inclusive podendo
aninhá-las, para obtermos mais liberdade de configuração.

Uma nova sintaxe precisa ser usada, e os comandos possuem uma certa
individualidade que precisa ser respeitada. Ao definir o ambiente usamos um
argumento a mais, para dizer se os elementos das colunas serão alinhados à
esquerda, direita, ou centralizados. E iremos separar os elementos das colunas
com o símbolo \&. Veja o código que gera uma tabela, e seu respectivo
resultado:

\noindent\textbackslash begin\{tabular\}\{ccc\} \\
	\textbackslash hline \%Adiciona linhas horizontais\\
	\textbackslash head\{Command\} \& \textbackslash head\{Declaration\} \& \textbackslash head\{Output\} \textbackslash \textbackslash \\
	\textbackslash hline\\
	\textbackslash verb$|$\textbackslash textrm$|$ \& \textbackslash verb$|$\textbackslash rmfamily$|$ \& \textbackslash rmfamily Example text \textbackslash \textbackslash \\
	\textbackslash verb$|$\textbackslash textsf$|$ \& \textbackslash verb$|$\textbackslash sffamily$|$ \& \textbackslash sffamily Example text \textbackslash \textbackslash \\
	\textbackslash verb$|$\textbackslash texttt$|$ \& \textbackslash verb$|$\textbackslash ttfamily$|$ \& \textbackslash ttfamily Example text \textbackslash \textbackslash \\
	\textbackslash hline\\
\textbackslash end\{tabular\}\\

\begin{tabular}{ccc}
	\hline %Adiciona linhas horizontais
	\head{Command} & \head{Declaration} & \head{Output} \\
	\hline
	\verb|\textrm| & \verb|\rmfamily| & \rmfamily Example text \\
	\verb|\textsf| & \verb|\sffamily| & \sffamily Example text \\
	\verb|\texttt| & \verb|\ttfamily| & \ttfamily Example text \\
	\hline
\end{tabular}

O comando \{ccc\} nos diz que haverão três colunas, todas centralizadas. As
linhas são separadas como no ambiente tabbing. Não devemos criar uma nova linha
no último elemento (hline), isso causa problemas de alinhamento para a coluna.

Assim como o minipage, o tabular possui outro argumento, que diz respeito à
posição da tabela. A definição completa do ambiente é: \newline
\noindent\textbackslash begin\{tabular\}[posição]\{especificadores de coluna\} \\\indent
No argumento de posição, refere-se ao alinhamento em relação ao topo do texto que circunda a tabela
célula, em relação à base, ou centralizadas. O padrão é centralizado. As opções
são, portanto [tbc].

É possível desenhar três tipos de linhas em tabelas:

\begin{itemize}
\item\textbf{\textbackslash hline} - desenha uma linha horizontal.
\item\textbf{\textbackslash cline[m-n]} - desenha uma linha no topo, nas colunas selecionadas.
\item\textbf{\textbackslash vline} - desenha uma linha vertical.
\end{itemize}

Vejamos uma aplicação descuidada dessas linhas:

\noindent\textbackslash begin\{center\}\\
\textbackslash begin\{tabular\}\{ccc\}\\
	\textbackslash hline \\
	\textbackslash vline \textbackslash head\{Animal\}\textbackslash vline \& \textbackslash vline\textbackslash head\{Late?\}\textbackslash vline \& \textbackslash head\{Mia?\} \textbackslash \textbackslash \\
	\textbackslash hline\\
	Cachorro \& sim \&     \textbackslash \textbackslash \\
	Gato     \&     \& não \textbackslash \textbackslash \\
	\textbackslash cline\{2-3\} Peixe    \&     \&     \textbackslash \textbackslash \\
	\textbackslash hline\\
\textbackslash end\{tabular\}\\
\textbackslash end\{center\}\\

\begin{center}
\begin{tabular}{ccc}
	\hline %Adiciona linhas horizontais
	\vline \head{Animal}\vline & \vline\head{Late?}\vline & \head{Mia?} \\
	\hline
	Cachorro & sim &     \\
	Gato     &     & não \\
	\cline{2-3} Peixe    &     &     \\
	\hline
\end{tabular}
\end{center}

Nos especificadores de coluna podemos aplicar um pouco mais de formatação do
que explicamos anteriormente. Podemos desenhar linhas verticais melhores, que
encaixam de forma mais graciosa, criando uma tabela com células bem
delimitadas. Vamos ver essas formatações.

\noindent\textbackslash begin\{center\}\\
\textbackslash begin\{tabular\}\{$|$l$|$c$|$r$|$p\{1.9cm\}$|$\}\\
	\textbackslash hline\\
	Left \& Center \& Right \& A fully Justified Paragraph cell \textbackslash \textbackslash \\
	\textbackslash hline\\
	l \& c \& r \& p \textbackslash \textbackslash \\
	\textbackslash hline\\
\textbackslash end\{tabular\}\\
\textbackslash end\{center\}\\

\begin{center}
\begin{tabular}{|l|c|r|p{1.9cm}|}
	\hline
	Left & Center & Right & A fully Justified Paragraph cell \\
	\hline
	l & c & r & p \\
	\hline
\end{tabular}
\end{center}

As opções que a formatação que os especificadores de coluna aceitam são:

\begin{itemize}
	\item l para alinhamento à esquerda (left).
	\item r para alinhamento à direita (right).
	\item c para alinhamento ao centro (center).
	\item p\{tamanho\} para uma linha formatada como parágrafo, com uma
		certa largura. Se várias forem usadas sequencialmente, estas
		serão alinhadas ao topo. é o equivalente a usar \textbackslash
		parbox[t]\{tamanho\} dentro da célula.  \item @\{código\}
		insere código ao invés de um espaço vazio antes ou depois de
		uma coluna. Pode ser também algum texto, ou deixado vazio para
		evitar esse espaço.
	\item $|$ significa linha vertical.
	\item *\{n\}\{opções\} é o equivalente a n cópias de "opções", em que n
		é um inteiro positivo, e options consiste em uma ou mais
		colunas, incluindo *.
\end{itemize}

Como sempre, podemos extender nossas opções usando pacotes. O pacote array irá
nos dar estas opções extras:

\begin{itemize}
	\item m\{tamanho\} é similar a \textbackslash parbox\{tamanho\}: A linha base fica no meio.
	\item b\{tamanho\} é similar a \textbackslash parbox[b]\{tamanho\}: A linha base fica na base.
	\item !\{Código\} pode ser usado como $|$ mas insere algum código ao
		invés de uma linha vertical. Em contraste com @\{...\}, o
		espaço entre as colunas não é suprimido.
	\item $>$\{Código\} Pode ser usado após um l,c,r,p,m ou b e insere
		código ao começo da entrada desta coluna.
	\item $<$\{Código\} Pode ser usado após um l,c,r,p,m ou b e insere
		código ao fim da entrada desta coluna.
\end{itemize}

Veremos um exemplo com alguns usos do que foi mostrado acima:

\noindent\textbackslash begin\{center\}\\
\textbackslash begin\{tabular\}\{@\{\}lp\{1.2cm\}m\{1.2cm\}b\{1.2cm\}@\{\}\}\\
	\textbackslash hline\\
	baseline \& aligned at the top \& aligned at the middle \& aligned at the bottom \textbackslash \textbackslash \\
	\textbackslash hline\\
\textbackslash end\{tabular\}\\
\textbackslash end\{center\}\\

\begin{center}
\begin{tabular}{@{}lp{1.2cm}m{1.2cm}b{1.2cm}@{}}
	\hline
	baseline & aligned at the top & aligned at the middle & aligned at the bottom \\
	\hline
\end{tabular}
\end{center}

Podemos aumentar a altura das linhas para caso o texto fique verticalmente apertado.

\ttfamily
\indent\textbackslash setlength\{\textbackslash extrarowheight\}\{4pt\}

\textbackslash begin\{tabular\}\{@\{\}$>$\{\textbackslash itshape\}ll!\{:\}l$<$\{.\}@\{\}\}

	\textbackslash hline

	Info: \& Software \& \textbackslash LaTeX \textbackslash \textbackslash 

	      \& Author \& Leslie Lamport\textbackslash \textbackslash 

	      \& Website \& www.latex-project.org\textbackslash \textbackslash 

	\textbackslash hline

\textbackslash end\{tabular\}

\sffamily

\setlength{\extrarowheight}{4pt}
\begin{tabular}{@{}>{\itshape}ll!{:}l<{.}@{}}
	\hline
	Info: & Software & \LaTeX \\
	      & Author & Leslie Lamport\\
	      & Website & www.latex-project.org\\
	\hline
\end{tabular}

\setlength{\extrarowheight}{0pt} % Voltando ao padrão

Existem problemas em usar $>$\{\textbackslash itshape\}. Esta declaração pode
acabar mudando o significado de \textbackslash \textbackslash. Assim, caso seja
necessário, utilize a seguinte declaração:

\textbackslash begin\{tabular\}\{$>$\{\textbackslash centering \textbackslash arraybackslash\}p\{5cm\}\}

Lembrando que pode-se também utilizar o \textbackslash \textbackslash [10pt] como argumento para alterar o tamanho da quebra de linha.

Há uma macro para esticar a tabela inteira \textbackslash arraystretch. Para
usá-la:\\
\textbackslash renewcommand\{\textbackslash arraystretch\}\{1.5\}\\
\indent Este comando irá esticar a tabela por 50 por cento.







\subsection{table e legendas}


\section{Imagens}

\newpage
