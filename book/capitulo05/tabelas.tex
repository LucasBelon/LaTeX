\chapter{Tabelas e imagens}
Tabelas e imagens são elementos que não fazem exatamente parte da lógica de um
texto corrido. Não existe parágrafo e linha do jeito que existe no texto. Por
conta disso, devemos encarar como uma inserção de um elemento geralmente
retangular, com certas dimensões, e que deve se comportar de uma ou outra
maneira diferente. 



\section{Tabelas}
Tabelas no \LaTeX\ são entidades que precisam se entendidas em duas etapas.
Existe o formato de tabela, que se refere a um frame, com certo espaçamento
entre as células, com certa disposição no corpo do texto, e seu número de
linhas e colunas. Uma tabela propriamente dita possui contorno entre os
elementos, geralmente um título, e eventualmente uma legenda.

Tabelas são uma forma de se expressar informação. Sua principal virtude é
organizar, categorizar, e exibir, de forma sucinta, dados. Uma boa tabela deve
conter um bom relacionamento lógico entre os itens, em seu ordenamento e sua
apresentação.

\subsection{tabbing}
Para alinhar texto em colunas no tempo das máquinas de escrever era usado o que
se chama de tabstop. Por causa disso, temos no LaTeX um ambiente chamado
"tabbing".

\noindent\textbackslash begin\{tabbing\}\\
	\textbackslash emph\{info: \} \textbackslash = Software \textbackslash = : \textbackslash = \textbackslash LaTeX \textbackslash \textbackslash \\
	\textbackslash $>$ Author \textbackslash $>$ : \textbackslash $>$ Leslie Lamport \textbackslash \textbackslash \\
	\textbackslash $>$ Website \textbackslash $>$ : \textbackslash $>$ www.latex-project.org \textbackslash \textbackslash \\
\textbackslash end\{tabbing\}\\

O Código acima irá gerar o seguinte texto:

\begin{tabbing}
	\emph{info: } \= Software \= : \= \LaTeX \\
	\> Author \> : \> Leslie Lamport \\
	\> Website \> : \> www.latex-project.org \\
\end{tabbing}

Fica visível a organização entre colunas.
Na primeira linha são usados os caracteres "\textbackslash = ", que servem para
definir um tabstop. Basicamente, é criada uma linha vertical imaginária que irá
ditar qual a distância deve ser pulada num pressionar da tecla tab.

Como estamos em \LaTeX, precisamos inserir um caractere como o tab com um
comando. Na segunda e terceira linha de nossa tabela temos o uso do next tab
stop, próximo tabstop, que basicamente avança o texto na distância definida na
primeira linha.

Por fim, pulamos a linha usando \textbackslash \textbackslash. Essa é uma
sintaxe já conhecida. O que criamos ainda não se trata de uma tabela, mas sim
um texto tabularizado. No entanto, a lógica já começa a se expressar. É preciso
pensar em como os espaços são definidos, e como podem ser definidos.

Esse formato de texto nos permite criar colunas com texto alinhado à esquerda.
Mas e se o texto alcançar os limites da página, tanto para a borda externa
quanto para a base da página? É o que veremos, e como consertar caso algo dê
errado.

\noindent\textbackslash newcommand\{\textbackslash head\}[1]\{\textbackslash textbf\{\#1\}\}\\
\\
\textbackslash begin\{tabbing\}\\
	Family \textbackslash = \textbackslash verb$|$\textbackslash textrm\{...\}$|$ \textbackslash = \textbackslash head\{Declaration\} \textbackslash = \textbackslash kill\\
	\textbackslash head\{Type\} \textbackslash $>$ \textbackslash head\{Command\} \textbackslash $>$ \textbackslash head\{Declaration\} \textbackslash $>$ \textbackslash head\{Example\} \textbackslash \textbackslash \\
	Family \textbackslash $>$ \textbackslash verb$|$\textbackslash textrm\{...\}$|$ \textbackslash $>$ \textbackslash verb$|$\textbackslash rmfamily$|$\\
	\textbackslash $>$ \textbackslash rmfamily Example text \textbackslash \textbackslash \\
	\textbackslash $>$ \textbackslash verb$|$\textbackslash textsf\{...\}$|$ \textbackslash $>$ \textbackslash verb$|$\textbackslash sffamily$|$\\
	\textbackslash $>$ \textbackslash sffamily Example text \textbackslash \textbackslash \\
\textbackslash end\{tabbing\}\\

% Novo comando que serve apenas pra melhorar a sintaxe.
\newcommand{\head}[1]{\textbf{#1}}

\begin{tabbing}
	Family \= \verb|\textrm{...}| \= \head{Declaration} \= \kill
	\head{Type}\> \head{Command} \> \head{Declaration} \> \head{Example} \\
	Family \> \verb|\textrm{...}| \> \verb|\rmfamily|
	\> \rmfamily Example text \\
	\> \verb|\textsf{...}| \> \verb|\sffamily|
	\> \sffamily Example text \\
\end{tabbing}

Este é mais difícil de compreender. A primeira linha cria o espaçamento, e o
comando kill apaga a linha. No entanto, o espaçamento segue definido. Na
segunda linha estamos criando já os títulos das colunas. Na terceira e quarta
linha estamos criando a primeira linha da tabela. Na quinta e sexta linha
estamos criando a última. O segredo está em entender o que está acontecendo nas
duas primeiras linhas e como funciona a lógica do espaçamento.

Detalhe: usamos o comando \textbackslash verb$|$\textbackslash textsf\{...\}$|$
que faz com que o LaTeX escreva o que tiver entre os $|$ de forma como é,
desabilitando qualquer função que possa ser escrita ali.
O nome disso é verbatim. Para uso mais extenso, usar o ambiente \emph{verbatim}. 


Outros comandos úteis são:
\begin{itemize}
	\item \textbackslash + no fim da linha faz a linha subsequente começar na primeira tabulação.
	\item \textbackslash - cancela um \textbackslash + anterior. O uso de múltiplos desses é cumulativo.
	\item \textbackslash $<$ no começo da linha cancela o efeito de um \textbackslash + anterior nesta linha.
\end{itemize}

CONTINUA NA PÁGINA 144!!!!

\subsection{tabular}

\subsection{table e legendas}


\section{Imagens}

\newpage
