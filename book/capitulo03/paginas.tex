\chapter[Design de página]{Design de página - Entendendo estruturas de texto}
Já falamos sobre fontes das letras, já falamos sobre ligaduras entre palavras, e já discorremos sobre espaçamentos entre frases e parágrafos.
Vamos agora subir na hierarquia da estrutura de um texto.
Nos interessa de agora em diante qual o tipo de texto irá ser adotado, para que a devida estrutura seja adotada.

A maior parte dos formatos de textos assumem os comandos \emph{chapter}, \emph{section}, \emph{subsection} e \emph{subsubsection}.
Estas entradas interagem com o sumário, adicionam numeração, e alteram a fonte do título de suas partes.

O comando \textbackslash chapter irá sempre começar uma nova página.
O \textbackslash section será aninhado ao \textbackslash chapter anterior.
Os \textbackslash subsection e \textbackslash subsubsection seguirão a mesma lógica.

A estrela do capítulo será o \textbackslash documentclass[a4paper,12pt]\{book\}.
Este possui a estrutura mais interessante.
Geralmente livros são compostos por páginas e contrapáginas.
Os distanciamentos entre as bordas ficam refletidos entre páginas pares e ímpares.
Por conta disso, adotamos a página esquerda para referenciar estas bordas.

A margem direita é a margem externa, já que capítulos começam sempre na página direita.
A margem esquerda é a margem interna, sempre maior por causa da colagem das folhas do livro impresso.
A margem inferior comumente contém a numeração de páginas e a margem superior possui um espaçamento maior em páginas de início de capítulo.


\begin{landscape}
\section{Pacote geometry}
Para customizar páginas além do que o \LaTeX\ entrega por padrão, usamos o pacote \textbackslash geometry. 
Exemplificamos com 
\textbackslash usepackage[a4paper, inner=1.5cm, outer=3cm, top=2cm, bottom=3cm, bindingoffset=1cm]\{geometry\}

No inicio, realizava-se a diagramação à mão.
O tamanho de margem esquerda + direita + comprimento do texto com frequência era maior do que o tamanho da página.
O pacote geometry resolve este problema, tornando automático o layout do texto.

O pacote geometry nos oferece uma quantidade de opções para o tamanho do papel a ser utilizado.

\begin{multicols}{2}
\begin{itemize}
    \item \emph{paper=name} - letterpaper, executivepaper, legalpaper, a0paper, \ldots ,a6paper, b0paper, \ldots ,b6paper;
    \item \emph{paperwidth} - paperwidth=10in;
    \item \emph{paperheight} - paperheight=7in;
    \item \emph{papersize=\{width, height\}} - papersize=\{7in,10in\};
    \item \emph{landscape} - ;
    \item \emph{portrait} - modo padrão;
\end{itemize}
\end{multicols}

Outras opções como largura, altura, orientação da página, utilização de espaço para cabeçalho e rodapé, são exemplificadas a seguir:
\begin{multicols}{2}
\begin{itemize}
    \item \emph{textwidth} Delimita a largura que o texto ocupará na página, por exemplo \emph{textwidth=140mm};
    \item \emph{textheight} Delimita a altura que o texto ocupará na página, por exemplo \emph{textheight=180mm} ;
    \item \emph{lines} Outra forma de delimitar a altura do texto, por exemplo \emph{lines=25} ;
    \item \emph{includehead} Faz com que o cabeçalho seja incorporado como área de texto. A opção padrão é \emph{false} ;
    \item \emph{includefoot} Faz com que o rodapé seja incorporado como área de texto. A opção padrão é \emph{false} ;
    \item \emph{left ou right} Delimita o tamanho da margem, por exemplo \emph{left=3cm}. Usa-se no caso de documentos de uma página por folha;
    \item \emph{inner ou outer} Delimita o tamanho da margem interna ou externa, por exemplo \emph{inner=2cm}. Usa-se para documentos de duas páginas por folha;
    \item \emph{top, bottom} Especifica o tamanho das margens verticais como em \emph{top=25mm};
    \item \emph{twosides} Força o documento a adotar página esquerda e direita. Nas páginas verso os valores das margens esquerda e direita serão invertidos;
    \item \emph{bindingoffset} Separa um espaço para a união de folhas. Será aplicado à margem interna;
\end{itemize}
\end{multicols}

\end{landscape}

De vez em quando precisamos preparar um documento com uma página que contém uma imagem.
Este tipo de página, geralmente, convém ser posicionada em paisagem.
Infelizmente não é o pacote geometry que resolve esse tipo de problema.
Para alterar para paisagem utilizamos um outro pacote, chamado lscape.
Simplesmente começamos um ambiente novo, com \textbackslash begin\{landscape\} e \textbackslash end\{landscape\}.

O pacote geometry é definido no preâmbulo, mas podemos definir uma nova geometria com \textbackslash newgeometry\{lista de argumentos\}.
Não se engane, só podemos redefinir o layout sem alterar o tamanho do papel, e não podemos alterar de retrato para paisagem.
Um exemplo é alterar a altura e largura do layout, e opções de offset.
Também podem ser alteradas opções referentes ao cabeçalho e rodapé, assim como ao corpo do texto.
Após realizada a alteração, pode-se retornar à geometria original com o comando \textbackslash restoregeometry.

\section{Pacote setspace}
Este pacote se concentra no espaçamento das linhas.
Suas opções de uso são:
\begin{itemize}
    \item \textbackslash usepackage[onehalfspacing]\{setspace\}
    \item \textbackslash usepackage[singlespacing]\{setspace\}- Opção padrão.
    \item \textbackslash usepackage[doublespacing]\{setspace\}
\end{itemize}

\begin{spacing}{2.4}
    Este texto está esticado em uma razão de 2,4. Lorem ipsum dolor sit amet,
    officia excepteur ex fugiat reprehenderit enim labore culpa sint ad nisi
    Lorem pariatur mollit ex esse exercitation amet. Nisi anim cupidatat
    excepteur officia. Reprehenderit nostrud nostrud ipsum Lorem est aliquip
    amet voluptate voluptate dolor minim nulla est proident. Nostrud officia
    pariatur ut officia. Sit irure elit esse ea nulla sunt ex occaecat
    reprehenderit commodo officia dolor Lorem duis laboris cupidatat officia
    voluptate. Culpa proident adipisicing id nulla nisi laboris ex in Lorem
    sunt duis officia eiusmod. Aliqua reprehenderit commodo ex non excepteur
    duis sunt velit enim. Voluptate laboris sint cupidatat ullamco ut ea
    consectetur et est culpa et culpa duis.
\end{spacing}

\section{Especificações do DocumentClass}
As classes base do LaTeX são book, article, report, slides, e letter.
Existe uma classe de documento melhor que letter, scrlttr2.
Slides é velho e ruim, usar beamer ou powerdot.

Vamos verificar as opções de documentclass:
\begin{multicols}{2}
\begin{itemize}
    \item a4paper, b2paper, letterpaper, legalpaper, executivepaper.
    \item 10pt, 11pt ou 12pt. 
    \item landscape.
    \item onecolumn ou twocolumn.
    \item oneside ou twoside. Não se usa twoside com slides e letter.
    \item openright ou openany. Permite capitulos comecarem na direita ou em qualquer página.
    \item titlepage ou notitlepage. Cria uma página de título separada.
    \item final ou draft. Quando draft é selecionado o documento fica com uma caixa preta não sei onde, pra ajudar a revisão
    \item openbib. Faz os arquivos bib serem salvos em formato aberto ao invés de comprimido.
    \item fleqn. Faz fórmulas serem alinhadas à esquerda.
    \item leqno. Para fórmulas numeradas, coloca o número à esquerda.
\end{itemize}
\end{multicols}

Existe uma família de pacotes que chama KOMA-Script.
Ele permite usar classes com maior flexibilidade, com mais opções e recursos.
No entanto, também parece que é uma outra coleção de coisas para aprender.

Pacotes que permitem uso de fontes arbitrárias:
extarticle, extbook, extreport, extletter.

\section{TOC - Table of Contents}
O sumário, ou tabela de conteúdos (traduzindo diretamente), é definida no início do documento.

\noindent \textbackslash begin\{document\}

\noindent \textbackslash tableofcontents

O sumário é gerado a partir do arquivo auxiliar.
Ao utilizar uma ferramenta em que a compilação não é automática é possível que seja necessária a repetição da compilação.
Nestes casos, a primeira compilação irá criar os arquivos auxiliares, e a segunda irá criar corretamente o sumário.

No sumário podemos colocar um nome diferente do colocado no capítulo com \\
\textbackslash chapter[nome sumario]\{Nome capitulo\} \\
A utilidade é adicionar um nome menor no sumário.
O nome deste capítulo é um exemplo.
No sumário aparece apenas uma parte do nome completo.

O comando \textbackslash tableofcontents está vinculado aos comandos que definem camadas de texto.
As "camadas" de níveis que as bases geralmente aceitam são:
\begin{multicols}{2}
\begin{enumerate}
    \item part
    \item chapter
    \item section
    \item subsection
    \item subsubsection
    \item paragraph
    \item subparagraph
\end{enumerate}
\end{multicols}

Todos os comandos que seccionam o texto possuem a forma
com asterisco, como \textbackslash section*, que afeta a exibição
da numeração no corpo, mas não no sumário.

\section{Cabeçalho}
Customizando cabeçalho com pacote fancyhdr (fancy header)

\noindent \textbackslash usepackage\{fancyhdr\}\\
\textbackslash fancyhdr\{\} \\
\textbackslash fancyhead[LE]\{\textbackslash leftmark\} \\
\textbackslash fancyhead[RO]\{\textbackslash nouppercase\{\textbackslash rightmark\}\} \\
\textbackslash fancyfoot[LE,RO]\{\textbackslash thepage\} \\
\textbackslash pagestyle\{fancy\} 

Deve ser aplicado no preambulo.

Fancyhdr limpa os estilos do cabecalho e rodapé.

Leftmark e Rightmark trazem informações sobre o capítulo e seção atuais.

LE se refere ao lado esquerdo de páginas pares. Ou seja, Left Even.
Por padrão são utilizadas letras maiúsculas, por isso
é adicionado o nouppercase.

As funcionalidades do fancyhdr são parte de um estilo de página chamafo fancy.
Portanto, é preciso declarar o estilo de página como pagestyle fancy.

O LaTeX padrão possui quatro estilos de página.
\begin{itemize}
    \item empty
    \item plain
    \item headings
    \item myheadings
\end{itemize}

Para definir o estilo de uma página específica thispagestyle pode ser usado.

O cabeçalho e rodapé podem ser divididos cada um em 3 partes.
left, center, right (l, c, r).
Os comandos para editar essas áreas são:
\textbackslash lhead \textbackslash chead \textbackslash rhead 
\textbackslash lfoot \textbackslash lfoot \textbackslash lfoot 

Cada um desses comandos requer um argumento (um texto) que será adicionado
na localização referida, como por exemplo thepage ou rightmark.

Alternativamente podemos utilizar \textbackslash fancyhead[code]\{text\} e
\textbackslash fancyfoot[code]\{text\}.
Em code, podemos ter a combinação das lerras:
L(eft) R(ight) C(enter) O(dd) E(ven) H(ead) F(oot)

Podemos adicionar linhas decorativas tanto no cabeçalho como no rodapé com\\
\textbackslash renewcommand\{\\headrulewidth\}{width}\\
\textbackslash renewcommand\{\\footrulewidth\}{width}

Com renewcommand estamos sobrescrevendo o valor que o objeto headrulewidth possui.
É semelhante à newcommand.
Aparentemente vamos utilizar bastante de agora em diante.

Quando usamos chapter*, não é produzido cabeçalho.
Por isso, vamos usar \\
markright\{right head\} e markboth\{left head \}\{right head\}

Uma alternativa para o pacote fanyhdr é o scrpage2.
Pertence ao Koma-script.

AQUI TEM ASSUNTOS QUE FAZEM PARTE DE OUTRO CAPÍTULO.
NÃO SEI EXATAMENTE O QUE FAZER, MAS VOU DAR UM JEITO.

\textbackslash pagebreak é diferente de newpage. Ele altera os espacos entre parágrafos.
Existem mais duas variantes: clearpage e cleardoublepage iniciarão
em uma página em branco, mesmo que no modo twosides.

Caso você perceba um espaçamento entre título e parágrafo,
atenção, pode ter sido causado por uma quebra de página ruim.
Para consertar, desligue a justificação com raggedbottom.
Para religar flushbottom.

Podemos aumentar o tamanho de nossa página com 

DESCOBRIR EM QUAL PÁGINA CONTINUAR
% TODO: Adicionar um fbox nos trechos que me refiro ao latex em código.
% Centralizar é uma boa ideia

\newpage
