\begin{landscape}
\chapter{Design de página}
Já falamos sobre fontes das letras, já falamos sobre ligaduras entre palavras, e já discorremos sobre espaçamentos entre frases e parágrafos.
Vamos agora subir na hierarquia da estrutura de um texto.
Nos interessa de agora em diante qual o tipo de texto irá ser adotado, para que a devida estrutura seja adotada.

A maior parte dos formatos de textos assumem os comandos \emph{chapter}, \emph{section}, \emph{subsection} e \emph{subsubsection}.
Estas entradas interagem com o sumário, adicionam numeração, e alteram a fonte do título de suas partes.

O comando \textbackslash chapter irá sempre começar uma nova página. O \textbackslash section será aninhado ao \textbackslash chapter anterior. Os \textbackslash subsection e \textbackslash subsubsection seguirão a mesma lógica.

A estrela do capítulo será o \textbackslash documentclass[a4paper,12pt]\{book\}.
Este possui a estrutura mais interessante. Geralmente livros são compostos por páginas e contrapáginas.
Os distanciamentos entre as bordas ficam refletidos entre páginas pares e ímpares.
Por conta disso, adotamos a página esquerda para referenciar estas bordas.

A margem direita é a margem externa, já que capítulos começam sempre na página direita.
A margem esquerda é a margem interna, sempre maior por causa da colagem das folhas do livro impresso.
A margem inferior comumente contém a numeração de páginas e a margem superior possui um espaçamento maior em páginas de início de capítulo.

Para customizar páginas além do que o \LaTeX\ entrega por padrão, usamos o pacote \textbackslash geometry. 
Exemplificamos com 
\textbackslash usepackage[a4paper, inner=1.5cm, outer=3cm, top=2cm, bottom=3cm, bindingoffset=1cm]\{geometry\}

No inicio, realizava-se a diagramação à mão.
O tamanho de margem esquerda + direita + comprimento do texto com frequência era maior do que o tamanho da página.
O pacote geometry resolve este problema, tornando automático o layout do texto.

O pacote geometry nos oferece uma quantidade de opções para o tamanho do papel a ser utilizado.
\end{landscape}
\begin{itemize}
    \item \emph{paper=name} - letterpaper, executivepaper, legalpaper, a0paper, \ldots ,a6paper, b0paper, \ldots ,b6paper;
    \item \emph{paperwidth} - paperwidth=10in;
    \item \emph{paperheight} - paperheight=7in;
    \item \emph{papersize=\{width, height\}} - papersize=\{7in,10in\};
    \item \emph{landscape} - ;
    \item \emph{portrait} - modo padrão;
\end{itemize}

Outras opções como largura, altura, orientação da página, utilização de espaço para cabeçalho e rodapé, são exemplificadas a seguir:
\begin{itemize}
    \item \emph{textwidth} Delimita a largura que o texto ocupará na página, por exemplo \emph{textwidth=140mm};
    \item \emph{textheight} Delimita a altura que o texto ocupará na página, por exemplo \emph{textheight=180mm} ;
    \item \emph{lines} Outra forma de delimitar a altura do texto, por exemplo \emph{lines=25} ;
    \item \emph{includehead} Faz com que o cabeçalho seja incorporado como área de texto. A opção padrão é \emph{false} ;
    \item \emph{includefoot} Faz com que o rodapé seja incorporado como área de texto. A opção padrão é \emph{false} ;
    \item \emph{left ou right} Delimita o tamanho da margem, por exemplo \emph{left=3cm}. Usa-se no caso de documentos de uma página por folha;
    \item \emph{inner ou outer} Delimita o tamanho da margem interna ou externa, por exemplo \emph{inner=2cm}. Usa-se para documentos de duas páginas por folha;
    \item \emph{top, bottom} Especifica o tamanho das margens verticais como em \emph{top=25mm};
    \item \emph{twosides} Força o documento a adotar página esquerda e direita. Nas páginas verso os valores das margens esquerda e direita serão invertidos;
    \item \emph{bindingoffset} Separa um espaço para a união de folhas. Será aplicado à margem interna;
\end{itemize}

De vez em quando precisamos preparar um documento com uma página que contém uma imagem.
Este tipo de página, geralmente, convém ser posicionada em paisagem.
O pacote é definido no preâmbulo, mas podemos definir uma nova geometria com \textbackslash newgeometry\{lista de argumentos\}.
Não se engane, só podemos redefinir o layout sem alterar o tamanho do papel, e não podemos alterar de retrato para paisagem.

Para alterar para paisagem utilizamos um outro pacote, chamado lscape.

Após realizada a alteração, pode-se retornar à geometria original com o comando \textbackslash restoregeometry.

\newpage
