%---------Preâmbulo-----------------+
% Define as características gerais do documento
% e adiciona pacotes para nos dar mais funcionalidades
% para escrita e expressão.

% DocumentClass, define o tipo de papel, e tipo
% de texto que será elaborado.
\documentclass[a4paper, 12pt, oneside, final]{book}

% Pacotes de matemática, pacotes matemáticos,
% ambientes, funções e declarações, símbolos,
% e fontes.
\usepackage{
	amsmath,
	amssymb,
	amsfonts,
	latexsym,
	mathrsfs,
	amsthm,
	amstext,
	bezier,
	amscd
    }

% Pacote Geometry, definindo espaçamento
% do corpo do texto. Deixar depois um
% uso genérico, com as opções.
\usepackage[top=2cm,
            bottom=2cm,
            left=2.5cm,
            right=2.5cm]
            {geometry}

% Pacote Landscape, responsável pela
% vista em paisagem.
\usepackage{lscape}

% Pacote Setspace lida com o espaçamento
% entre as linhas do texto. Dá um efeito
% Interessante
\usepackage{setspace}

% Pacote X color lida com certas cores
\usepackage{xcolor}

% Pacote color lida com colorização, age
% junto dos hiperlinks.
\usepackage{color}

% Pacote Hyperref que transforma os
% títulos do sumário em links que nos
% jogam para o capítulo clicado.
\usepackage{hyperref}
% Configurando o hyperref
\hypersetup{
    colorlinks=true,
    linktoc=all,
    linkcolor=black,
}

% Mais um pacote relacionado à cores.
% Se não me engano, este faz algo em
% relação à cor de fundo.
\usepackage{soul}

% Pacote Inputenc, lida com codificação
% do input, geralmente tornando em utf8
\usepackage[utf8]{inputenc}

% Pacote que faz com que o começo de
% parágrafo seja indentado.
\usepackage{indentfirst}

% Pacote que permite inserção de imagens
% no texto
\usepackage{graphicx}

% Pacote que adiciona manipulações relacionadas
% à língua, como hifenização e talvez certos acentos.
\usepackage[brazilian]{babel}

% Embelezando o cabeçalho e o rodapé
\usepackage{fancyhdr}
\setlength{\headheight}{16pt}
\addtolength{\topmargin}{-4pt}
\fancyhf{}
\fancyhead[L]{\nouppercase{\leftmark}}
\fancyhead[R]{\nouppercase{\rightmark}}
\fancyfoot[C]{\thepage}
\pagestyle{fancy}

% Configurando fbox para usar de moldura
% para imagens.
\setlength{\fboxrule}{1.2pt}

% Pacote Multicol adiciona a capacidade de
% mostrar listas com duas colunas
\usepackage{multicol}

% Pacote para termos um texto qualquer
% adicionado para preenchimento de página.
\usepackage{blindtext}

% Pacote que adiciona opções de listas.
\usepackage{paralist}

% Pacote que adiciona opções para os marcadores das listas.
\usepackage{enumitem}

% Pacote que adiciona capacidades de alteração de comprimentos diversos.
\usepackage{layouts}

% Pacote que adiciona opções de formatação de tabelas.
\usepackage{array}
%\setlength{\extrarowheight}{4pt} % Comando para aumentar o tamanho das linhas em tabelas.

% Pacote para adicionar ferramentas para embelezar tabelas.
\usepackage{booktabs}

% Pacote Multirow adiciona a capacidade de configurar tabelas
\usepackage{multirow}

