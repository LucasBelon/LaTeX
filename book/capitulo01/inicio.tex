\chapter{Início}
Este texto se trata de um resumo e tradução de partes do livro Latex Begginer's Guide de Stefan Kottwitz.

\section{Onde encontrar recursos}
Um dos melhores lugares para encontrar conteúdo sobre \LaTeX\ é no site CTAN:\\
\indent \url{https://www.ctan.org}

É interessante buscar por templates em\\
\indent \url{https://www.latextemplates.com}

Também é interessante buscar no github em\\
\indent \url{https://github.com/topics/latex-template}

Além de o projeto overleaf possuir alguns templates\\
\indent \url{https://www.overleaf.com/latex/templates}

Para encontrar documentação por tópicos, caso nem o nome se saiba:
\indent \url{http://texcatalogue.sarovar.org/bytopic.html}

\section{Preâmbulo}
No preambulo, trecho encontrado antes do \textbackslash begin\{document\},
é onde declaramos qual o tipo de texto, o formato, e qual o tamanho da fonte.
Utilizamos para isso a função \textbackslash documentclass.
Além disso, é onde deixamos as declarações \textbackslash usepackage.

Temos pacotes para adição de controle fino da geometria do documento,
para adição de símbolos e funções matemáticas,
para controle de cores, cabeçalhos e rodapés, entre outras funcionalidades.

O formato mínimo para um texto \LaTeX\ é:

\indent \textbackslash documentclass[a4paper]\{article\}

\textbackslash usepackage[utf8]\{inputenc\}

\textbackslash begin\{document\}

texto texto texto.

Outro parágrafo com texto.
Lorem ipsum dolor sit amet, qui minim labore adipisicing minim sint cillum sint consectetur cupidatat.

\textbackslash end\{document\}

\section{Caracteres Especiais}
Existem muitos caracteres que são usados na sintaxe do \LaTeX.
Por conta disso, existem formas específicas de escrevê-los:
\begin{itemize}
\item \textbackslash \# para obter \# ;
\item \textbackslash \% para obter \% ;
\item \textbackslash \$ para obter \$ ;
\item \textbackslash \_ para obter \_ ;
\item \textbackslash \& para obter \& ;
\item \textbackslash \{ para obter \{ ;
\item \textbackslash \} para obter \} ;
\item \textbackslash textbackslash para obter \textbackslash\ ;
\end{itemize}

\section{Acentuando sem pacotes}
Quando não adicionamos o pacote que adiciona utf-8,
nem um idioma que naturalmente utiliza acentuações,
ficamos um pouco à mercê.

Digamos que misteriosamente uma função não aceita acentuação,
por algum tipo de problema com ambientes diferentes.
Para acentuar usamos a sintaxe \textbackslash [acento] \{alvo\}.

Vejamos com o exemplo \textbackslash 'a: \'a.
Agora com o exemplo \textbackslash \"o: \"{o}.
Um último exemplo \textbackslash \~\ a: \~{a}.

\section{Itálico, negrito e outros}
Existem algumas formas de alterar o texto.
Escrever \textbackslash textit\{texto italizado\} é um tanto ruim,
mas depois de feito, ao menos é mais fácil encontrar o texto enfatizado.

\textit{italizado: abcdefghijklmnopqrstuvwxyz}

\textit{ITALIZADO: ABCDEFGHIJKLMNOPQRSTUVWXYZ}

Inclusive, podemos enfatizar com \textbackslash emph\{texto enfatizado\}.
O efeito se parece com \emph{Texto enfatizado}.
A depender de certas configurações, o texto enfatizado é idêntico ao italizado.

\emph{enfatizado: abcdefghijklmnopqrstuvwxyz}

\emph{ENFATIZADO: ABCDEFGHIJKLMNOPQRSTUVWXYZ}

Também temos com \textbackslash textbf\{texto em negrito\} como deixar em negrito.
\textbf{Texto em negrito}.

\textbf{negrito: abcdefghijklmnopqrstuvwxyz}

\textbf{NEGRITO: ABCDEFGHIJKLMNOPQRSTUVWXYZ}

Temos por fim um efeito chamado slanted. \textbackslash textsl\{texto slanted\}.
\textsl{Texto slanted}. É como um itálico, mas com outra fonte:

\textsl{abcdefghijklmnopqrstuvwxyz}

\textsl{ABCDEFGHIJKLMNOPQRSTUVWXYZ}


\section{Ambientes, fontes e famílias}
Dois conceitos importantes no \LaTeX\ são os environments,
em que se aplicam efeitos, e a dupla, declarações e funções.
Ambientes delimitam os efeitos de declarações.
Quando escrevemos \{ \}, estamos criando um ambiente.
Temos ambientes pré-definidos com \textbackslash begin\{efeito\} terminando com \textbackslash end.

Declarações são efeitos que queremos aplicar no texto, podendo ter um início mas não obrigatoriamente tendo um fim.
Funções geralmente definem um ambiente para seus efeitos, já ditando o fim na aplicação.

Uma aplicabilidade desses conceitos é a mudança de fonte.
Podemos mudar a fonte de um pequeno trecho de texto, assim como podemos alterá-la sem previsão de voltar à fonte padrão.

Para usarmos uma família de fontes usando uma declaração num ambiente, utilizamos a seguinte sintaxe:
\{\textbackslash itshape \textbackslash bfseries Texto de exemplo\}.
O efeito é de {\itshape \bfseries Texto de exemplo}.
Este tipo de aplicação é útil quando desejamos utilizar uma declaração que não possui função associada,
ou mesmo quando desejamos misturar e aninhar efeitos, ficando geralmente mais claro quais são os efeitos aplicados.
Como o texto se mistura com pequenas palavras chave, como em código, é bom manter uma conduta de escrita que auxilie a legibilidade.

Eu não sabia antes de procurar sobre tópicos de \LaTeX\, mas existem famílias de fontes.
As mais comuns são a família roman, a família sans-serif, e tipewriter.

Abaixo segue uma tabela de quais os comandos e declarações para diferentes fontes e famílias.

\begin{center}
\begin{tabular}{ c c c }
Comando & Declaração & Significado \\
\textbackslash textrm\{...\}&\textbackslash rmfamily&roman family\\
\textbackslash textsf\{...\}&\textbackslash sffamily&sans-serif family\\
\textbackslash texttt\{...\}&\textbackslash ttfamily&tipewriter family\\
\textbackslash textbf\{...\}&\textbackslash bfseries&bold-face\\
\textbackslash textmd\{...\}&\textbackslash mdseries&medium\\
\textbackslash textit\{...\}&\textbackslash itshape&italic shape\\
\textbackslash textsl\{...\}&\textbackslash slshape&slanted shape\\
\textbackslash textsc\{...\}&\textbackslash scshape&Small Caps Shape\\
\textbackslash textup\{...\}&\textbackslash upshape&Upright Shape\\
\textbackslash textnormal\{...\} & \textbackslash normalfont & Default font\\
\end{tabular}
\end{center}

\vspace{10pt}

Agora, experimentando cada tipo de fonte:

\subsection{Roman Family}
\textrm{abcdefghijklmnopqrstuvwxyz}

\textrm{ABCDEFGHIJKLMNOPQRSTUVWXYZ}

\subsection{Sans-serif Family}
\textsf{abcdefghijklmnopqrstuvwxyz}

\textsf{ABCDEFGHIJKLMNOPQRSTUVWXYZ}

\subsection{Tipewriter Family}
\textrm{abcdefghijklmnopqrstuvwxyz}

\textrm{ABCDEFGHIJKLMNOPQRSTUVWXYZ}

\subsection{Bold-Face}
\textbf{abcdefghijklmnopqrstuvwxyz}

\textbf{ABCDEFGHIJKLMNOPQRSTUVWXYZ}

\subsection{Medium}
\textmd{abcdefghijklmnopqrstuvwxyz}

\textmd{ABCDEFGHIJKLMNOPQRSTUVWXYZ}

\subsection{Italic Shape}
\textit{abcdefghijklmnopqrstuvwxyz}

\textit{ABCDEFGHIJKLMNOPQRSTUVWXYZ}

\subsection{slanted shape}
\textsl{abcdefghijklmnopqrstuvwxyz}

\textsl{ABCDEFGHIJKLMNOPQRSTUVWXYZ}

\subsection{Small Caps Shape}
\textsc{abcdefghijklmnopqrstuvwxyz}

\textsc{ABCDEFGHIJKLMNOPQRSTUVWXYZ}

\subsection{Upright Shape}
\textup{abcdefghijklmnopqrstuvwxyz}

\textup{ABCDEFGHIJKLMNOPQRSTUVWXYZ}

\subsection{Default Font}
\textnormal{abcdefghijklmnopqrstuvwxyz}

\textnormal{ABCDEFGHIJKLMNOPQRSTUVWXYZ}

\section{Tamanho de fontes}
O \LaTeX\ possui uma forma de trabalhar que praticamente te força a ignorar os detalhes menores da formatação.
O tamanho da fonte é um desses.
Você só precisa focar em decidir se a fonte é normal, grande, muito grande, e assim em diante.
É possível controlar diretamente o tamanho usando a notação de 10 pontos, 12 pontos, mas não é o fluxo que o programa te oferece.

Temos as seguintes declarações para alterar o tamanho da fonte:
\begin{multicols}{2}
\begin{itemize}
	\item \textbackslash tiny
	\item \textbackslash scriptsize
	\item \textbackslash footnotesize
	\item \textbackslash small
	\item \textbackslash normalsize
	\item \textbackslash large
	\item \textbackslash Large
	\item \textbackslash LARGE
	\item \textbackslash huge
	\item \textbackslash Huge
\end{itemize}
\end{multicols}

Exemplificando:

{\tiny Texto usando a declaração \textbackslash tiny}

{\scriptsize Texto usando a declaração \textbackslash scriptsize}

{\footnotesize Texto usando a declaração \textbackslash footnotesize}

{\small Texto usando a declaração \textbackslash small}

{\normalsize Texto usando a declaração \textbackslash normalsize}

{\large Texto usando a declaração \textbackslash large}

{\Large Texto usando a declaração \textbackslash Large}

{\LARGE Texto usando a declaração \textbackslash LARGE}

{\huge Texto usando a declaração \textbackslash huge}

{\Huge Texto usando a declaração \textbackslash Huge}

Algo que deve sempre ser levado em consideração é o tamanho de linhas vazias, saltos de linha, quebras de linha.
Quando alteramos o tamanho da fonte, também alteramos o tamanho destes elementos, que podem acabar se tornando maiores do que o desejado por descuido.
Não seja descuidado.

\section{Condensando em comandos}
Existem comandos que imprimem símbolos.
Esse é o caso do comando \textbackslash LaTeX, que gera  \LaTeX, \textbackslash TeX, gerador de \TeX\ e \$\textbackslash pi\$, $\pi$.
Ainda não foi discutido, mas uma das maiores potencialidades do tex é seu ambiente de fórmulas.
O caso de \$\textbackslash pi\$ é um exemplo da aplicação do ambiente de fórmulas.
Nesse ambiente, certos símbolos e comandos funcionam de forma diferente.

Até mesmo a fonte das letras é diferentes.
O uso do ambiente matemático será visto mais à frente, mas por enquanto, perceba o que \$\$ cos ; \textbackslash cos \$\$ gera:

$$cos ; \cos$$

Indo ao que interessa neste capítulo.
É possível estilizar estes comandos.

Utilizando:\\
\indent \textbackslash newcommand\{\textbackslash keyword\}[1]\{\textbackslash textbf\{\#1\}\}\\
\indent Geramos um novo comando, chamado \textbackslash keyword.
Em resumo, este comando cria um comando que irá mimetizar o comando \textbackslash textbf\{\}.

Podemos assim, por exemplo, criar um comando que irá tornar itálico e negrito ao mesmo tempo.
\textbackslash newcommand\{\textbackslash italicBold\}[1]\{\textbackslash textit \textbackslash textbf\{\#1\}\}.
A sintaxe é ruim de entender, mas o número 1 se refere ao número de argumentos.
Um comando com muitos argumentos seria usado da seguinte forma: \textbackslash keyword\{arg1\}\{arg2\}\{arg3\}\ldots

\newcommand{\italicBold}[1]{\textit{\textbf{#1}}}
Ilustrando, temos o efeito \italicBold{Negrito e Itálico aplicado}

\newpage
