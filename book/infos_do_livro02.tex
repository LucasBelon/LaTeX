\documentclass{article}
\usepackage{xspace}
\frenchspacing
\begin{document}


Usando um ambiente chamado minipage para obter um efeito semelhante ao parbox:

\begin{minipage}{3cm}
    \hyphenation{acro-nym}

    TUG is an acronym. It means \TeX\ Users Group. Largura de 3cm.

\end{minipage}

O comando hyphenation serve para, manualmente, indicar qual parte da palavra pode ser separada quando encontrado o fim do espaço de escrita.

Podemos usar o pacote hyphenat para aumentar as possibilidades.
Podemos prevenir hiphenização com
\textbackslash usepackage[none]\{hyphenat\}
E permitir hiphenização de texto da fonte typewriter com
\textbackslash usepackage[htt]\{hyphenat\}

Para pequenos trechos, podemos inibir a hyphenação com \textbackslash nohyphens\{pequeno trecho de texto\}

O pacote microtype mexe no espaçamento para obter o efeito de texto justificado.
No exemplo dado, mexeria no microtype para tentar eliminar a necessidade de hiphens.

Vamos quebrar linhas manualmente, usando de exemplo um poema do Allan Poe.

\emph{Annabel Lee} \\[3mm]
It was many and many years ago, \\
In a kingdom by the sea. \\
That a maiden there lived whom you may know \\
By the name of Annabel Lee

O comando que possui o mesmo efeito de \\
é o comando \newline.

O comando que indica que deve ser adicionada uma nova linha,
mas mantendo a justificação do texto \linebreak, pode causar
alongamento demasiado na distância entre palavras.
Por isso, é raramente usado.

Podemos usar \\[3mm] para adicionar espaçamento vertical.
\\*[5mm] faz o mesmo, mas previne a quebra de página antes da
próxima linha de texto.

\textbackslash linebreak[number] pode ser usado para influenciar a quebra de linha de forma sutil ou evidente.
Se number for zero, a quebra de linha é permitida, 1 significa que é desejada, 2 e 3 marcam outras requisições de insistência, e 4 irá forcar a quebra de linha.
Esta última é a opção padrão se não for especificado número.

Prevenindo quebras de linha.
\nolinebreak Faz o serviço.
Possui opções assim como o linebreak, sendo as requisições 
se tornando mais insistentes de 0 (recomenda não quebrar)
à 4 (proíbe de quebrar).

O comando \mbox[texto] desabilita a hyphenaçâo \textbf{e} a 
a quebra de linha do texto.

Quando duas palavras devem permanecer unidas na mesma linha utilizamos o símbolo \~.
Exemplo:
Dr. ~Watson previne que Dr. apareça sozinho no fim de uma linha

O comando para nota de rodapé é \footnote{Adicionei uma nota de rodapé}

% Explorando ligaduras -----------

ff fi fl flffifl -- ---

f\/f f\/i f\/l f\/l\/f\/f\/l\/f\/l -\/- -\/-\/-

O comando \/ está impedindo ligaduras de acontecerem.

O mesmo efeito pode ser alcançado usando {},
pois os espaços devem respeitar os limites do ambiente.

Para desligar as ligaduras de letras, pode-se passar o argumento noligature para o pacote microtype
% \usepackage[noligatures]{microtype}

Cada quantidade de hiphens utilizados servem a um propósito.

Um único hiphen serve para dividir palavras ou para palavras compostas.

Dois hiphens seguidos servem como separador de intervalo, como
"Das 9--10hrs da manhã".
A largura deste hiphen é equivalente a um dígito.

Três hiphens são convertidos para o travessão.

Algo semelhante ocorre com pontos.
reticências são menos espaçadas entre si do que um ponto 
e o caractere subsequente.

usar \@ indica ao LaTeX que o ponto se situa no fim de uma frase.

Para inserir reticências, use \ldots.

Pode-se alterar o espaçamento do texto adicionando ao preambulo:
\textbackslash frenchspacing
O padrão é \textbackslash ofrenchspacing

Vou deixar anotado qual a forma de se forçar acentos (já precisei uma vez).

\~{a}

Basicamente, backslash, acento, letra alvo entre chaves.
Mais de uma letra pode ser o alvo ao mesmo tempo.

No windows, se utf8 não funcionar, tentar latin1
Talvez mudar o compilador para LuaLaTeX ou XeLaTeX ajude a solucionar
problemas de fonte e de idioma.
Mudar apenas se houverem problemas irresolvíveis no LaTeX normal.

Continuar na página 81. vamos ver como controlar a justificação do texto.
Estamos pra terminar esse pedaço do livro. O primeiro terço chegará ao fim.

% O texto será jogado para a esquerda, e não haverá hiphenização.
\parbox{3cm}{\raggedright
    TUG is an acronym. It means \TeX\ user's group.
}
% Eu usava outro comando para enviar uma linha para a esquerda ou direita.
% Verificar o Kit Belon para Física B.
Se quiséssemos que o texto todo seja justificado à esquerda, poderíamos
adicionar \raggedright no preâmbulo. 
Testar depois.
Justificar à esquerda também é possível.

Centralizando um trecho de texto
{\centering
\huge\bfseries Centered Text \\
\Large \normalfont written by me \\
\normalsize \today \\
}


\noindent This is the beginning of a poem
by Edgar Allan Poe:
\begin{center}
    \emph{Annabel Lee}
\end{center}
\begin{center}
    It was many and many a year ago,\\
    In a kingdom by the sea,\\
    That a maiden there lived whom you may know\\
    By the name of Annabel Lee
\end{center}

The complete poem can be read on
\textbackslash url\{http://www.online-literature.com/poe/576/\}.

Para alinhar à esquerda
\begin{flushleft}
Alinhado à esquerda não justificado
\end{flushleft}
{\raggedright
Mesmo efeito, não justificado
}

Para alinhar à direita:
\begin{flushright}
    Alinhado à direita
\end{flushright}
{\raggedleft
Mesmo efeito, não justificado
}

Exemplo de citação:

Niels Bohr said: ``An expert is a person who has made
all the mistakes that can be made in a very narrow field.''

Albert Einstein said:

\begin{quote}
    Anyone who has never made a mistake has never tried anything new.
\end{quote}
Errors are inevitable. So, let's be brave trying something new.


Exemplo de citação longa:

The authors of the CTAN team listed ten good reasons
for using \TeX. Among them are:
\begin{quotation}
 \TeX\ has the best output. What you end with,
the symbols on the page, is as useable, and beautiful,
as a non-professional can produce.
 \TeX\ knows typesetting. As those plain text samples
show, \TeX's has more sophisticated typographical algorithms
such as those for making paragraphs and for hyphenating.
 \TeX\ is fast. On today's machines \TeX\ is very fast.
 It is easy on memory and disk space, too.
 \TeX\ is stable. It is in wide use, with a long history.
 It has been tested by millions of users, on demanding input.
 It will never eat your document. Never.
\end{quotation}
The original text can be found on
\textbackslash url\{http://www.ctan.org/what\_is\_tex.html\}.

Adicionar o pacote parskip irá adicionar um espaço entre os parágrafos.
Existem duas formas de se distinguir parágrafos.
Uma, a padrão, é indentar o começo de cada parágrafo.
A segunda é adicionando um espaço vertical entre parágrafos.
pode ser útil no caso de colunas estreitas, em que a indentação seria custosa ou anti-estilística.

\end{document}
