\chapter{Listas}
A partir de agora não teremos arquivos de rascunho.
Mas a partir de agora, o grosso das funcionalidades devem ser familiares.
O que significa que será muito mais expositivo do que explicativo.

Iremos introduzir um tipo de listas e explicar um pouco da mecãnica.
Esta mecânica servirá para outros tipos de lista, mostradas mais ao fim.

\section{Listas simples}
Em inglês, bulleted lists, são listas compostas de pontos.
Geralmente indicam itens que não possuem exatamente uma ordem expressa.
Uma lista de compras, uma lista de tarefas sem prioridade definida,
uma lista de desejos de natal.

A sintaxe é a partir de um ambiente pré-definido:\newline
\textbackslash begin\{itemize\}\newline
\indent \textbackslash item texto do item quadrado\newline
\indent \textbackslash item texto do item círculo\newline
\indent \textbackslash item texto do item triângulo\newline
\textbackslash end\{itemize\}\newline

O efeito é:
\begin{itemize}
	\item texto do item quadrado
	\item texto do item círculo
	\item texto do item triângulo
\end{itemize}

\subsection{Aninhando listas}
Podemos criar itens que agregam subitens.

\noindent\textbackslash begin\{itemize\}\newline
\textbackslash item texto do item quadrado\newline
\textbackslash item texto do item círculo\newline
\textbf{
\textbackslash begin\{itemize\}\newline
\textbackslash item texto do item semicírculo\newline
\textbackslash item texto do item $\pi$\newline
\textbackslash item texto do item circunferência\newline
\textbackslash end\{itemize\}\newline
}
\textbackslash item texto do item triângulo\newline
\textbackslash end\{itemize\}\newline

E o efeito será:

\begin{itemize}
	\item texto do item quadrado
	\item texto do item círculo
	\begin{itemize}
		\item texto do item semicírculo
		\item texto do item $\pi$
		\begin{itemize}
			\item números irracionais
			\begin{itemize}
				\item números infinitos
			\end{itemize}
		\end{itemize}
		\item texto do item circunferência
	\end{itemize}
	\item texto do item triângulo
\end{itemize}

O exemplo mostra uma menor quantidade de listas aninhadas, pelo bem da legibilidade.
Acima de 4 níveis neste aninhamento, não são geradas sublistas.
Raramente se utiliza uma sublista com ícones que seja tão profunda.
Quando se alcança tamanha profundidade, altera-se para numeração do tipo, 1.1, 1.2, 1.2.1, e assim em diante.

\section{Listas numeradas}
Podemos aninhar listas simples em listas numeradas. Veremos o exemplo:\newline

\noindent\textbackslash begin\{enumerate\}\newline
\textbackslash item item um\newline
\textbackslash item item dois\newline
\textbackslash begin\{enumerate\}\newline
\textbackslash item item a\newline
\textbackslash end\{enumerate\}\newline
\textbackslash item item um\newline
\textbackslash end\{enumerate\}\newline

\noindent\begin{enumerate}
\item item um
\item item dois
\begin{enumerate}
\item item a
\item item b
\item item c
\begin{enumerate}
\item item i
\begin{enumerate}
\item item A
\end{enumerate}
\end{enumerate}
\end{enumerate}
\item item um
\end{enumerate}

Temos mais uma vez o efeito de quatro níveis de aninhamento.
O comando \textbackslash item aceita o formato \textbackslash item[text], que
irá fazer com que qualquer coisa apareça como marcador.

\section{Customizando listas}
Vamos ver outros ambientes de listagem.
Estes podem ser utilizados carregando um pacote chamado \emph{paralist}.

\noindent \emph{No preâmbulo adicionar:\newline
\noindent\textbackslash usepackage\{paralist\}\newline}
\textbackslash begin\{compactenum\}\newline
\textbackslash item Abra o terminal.\newline
\textbackslash item Invoque o vim.\newline
\textbackslash begin\{compactitem\}\newline
\textbackslash item Salve seu arquivo com o nome e extensão corretos.\newline
\textbackslash item Escreva seu texto.\newline
\textbackslash end\{compactitem\}\newline
\textbackslash item Salve o arquivo e saia.\newline
\textbackslash end\{compactenum\}\newline

Perceba no resultado o espaçamento menor, graças ao estilo compacto que o
pacote nos traz.

\begin{compactenum}
\item Abra o terminal.
\item Invoque o vim.
\begin{compactitem}
\item Salve seu arquivo com o nome e extensão corretos.
\item Escreva seu texto.
\end{compactitem}
\item Salve o arquivo e saia.
\end{compactenum}

\subsection{Listas alinhadas}
Ao invés de serem itens colocados um abaixo do outro, serão itens escritos na
mesma linha.

\noindent \emph{No preâmbulo adicionar:\newline
\noindent\textbackslash usepackage\{paralist\}\newline}
\textbackslash begin\{compactenum\}\newline
\textbackslash item Abra o terminal.\newline
\textbackslash item Invoque o vim:\newline
\textbackslash begin\{\textbf{inparaenum}\}\newline
\textbackslash item Salve seu arquivo com o nome e extensão corretos.\newline
\textbackslash item Escreva seu texto.\newline
\textbackslash item Refatore seu texto.\newline
\textbackslash end\{\textbf{inparaenum}\}\newline
\textbackslash item Salve o arquivo e saia.\newline
\textbackslash end\{compactenum\}\newline

Desta vez estamos colocando uma lista que reside em um parágrafo ao invés de uma sucessão de linhas.

\begin{compactenum}
\item Abra o terminal.
\item Invoque o vim:
\begin{inparaenum}
\item Salve seu arquivo com o nome e extensão corretos.
\item Escreva seu texto.
\item Refatore seu texto.
\end{inparaenum}
\item Salve o arquivo e saia.
\end{compactenum}

\begin{compactenum}
\item Abra o terminal.
\item Invoque o vim:
\begin{asparaenum}
\item Salve seu arquivo com o nome e extensão corretos.
\item Escreva seu texto.
\item Refatore seu texto.
\end{asparaenum}
\item Salve o arquivo e saia.
\end{compactenum}

Para cada ambiente padrão, o pacote paralist adiciona três ambientes correspondentes:

Listas numeradas:
\begin{itemize}
	\item compactenum: Já visto
	\item inparaenum: Já visto
	\item asparaenum: Supostamente cada item é tratado como um parágrafo do LaTeX.
\end{itemize}

Listas simples:
\begin{itemize}
	\item compactitem
	\item inparaitem
	\item asparaitem
\end{itemize}


\newpage
