\chapter{Listas}
A partir de agora não teremos arquivos de rascunho.
Mas a partir de agora, o grosso das funcionalidades devem ser familiares.
O que significa que será muito mais expositivo do que explicativo.

Iremos introduzir um tipo de listas e explicar um pouco da mecãnica.
Esta mecânica servirá para outros tipos de lista, mostradas mais ao fim.

\section{Listas simples}
Em inglês, bulleted lists, são listas compostas de pontos.
Geralmente indicam itens que não possuem exatamente uma ordem expressa.
Uma lista de compras, uma lista de tarefas sem prioridade definida,
uma lista de desejos de natal.

A sintaxe é a partir de um ambiente pré-definido:\newline
\textbackslash begin\{itemize\}\newline
\indent \textbackslash item texto do item quadrado\newline
\indent \textbackslash item texto do item círculo\newline
\indent \textbackslash item texto do item triângulo\newline
\textbackslash end\{itemize\}\newline

O efeito é:
\begin{itemize}
	\item texto do item quadrado
	\item texto do item círculo
	\item texto do item triângulo
\end{itemize}

\subsection{Aninhando listas}
Podemos criar itens que agregam subitens.

\noindent\textbackslash begin\{itemize\}\newline
\textbackslash item texto do item quadrado\newline
\textbackslash item texto do item círculo\newline
\textbf{
\textbackslash begin\{itemize\}\newline
\textbackslash item texto do item semicírculo\newline
\textbackslash item texto do item $\pi$\newline
\textbackslash item texto do item circunferência\newline
\textbackslash end\{itemize\}\newline
}
\textbackslash item texto do item triângulo\newline
\textbackslash end\{itemize\}\newline

E o efeito será:

\begin{itemize}
	\item texto do item quadrado
	\item texto do item círculo
	\begin{itemize}
		\item texto do item semicírculo
		\item texto do item $\pi$
		\begin{itemize}
			\item números irracionais
			\begin{itemize}
			\item números infinitos
			\end{itemize}
		\end{itemize}
		\item texto do item circunferência
	\end{itemize}
	\item texto do item triângulo
\end{itemize}

O exemplo mostra uma menor quantidade de listas aninhadas, pelo bem da legibilidade.
Acima de 4 níveis neste aninhamento, não são geradas sublistas.
Raramente se utiliza uma sublista com ícones que seja tão profunda.
Quando se alcança tamanha profundidade, altera-se para numeração do tipo, 1.1, 1.2, 1.2.1, e assim em diante.

% TODO: Descobrir como usar esse tipo de lista de numeração infinita.

\section{Listas numeradas}
Podemos aninhar listas simples em listas numeradas. Veremos o exemplo:\newline

\noindent\textbackslash begin\{enumerate\}\newline
\textbackslash item item um\newline
\textbackslash item item dois\newline
\textbackslash begin\{enumerate\}\newline
\textbackslash item item a\newline
\textbackslash end\{enumerate\}\newline
\textbackslash item item um\newline
\textbackslash end\{enumerate\}\newline

\noindent\begin{enumerate}
\item item um
\item item dois
\begin{enumerate}
\item item a
\item item b
\item item c
\begin{enumerate}
\item item i
\begin{enumerate}
\item item A
\end{enumerate}
\end{enumerate}
\end{enumerate}
\item item um
\end{enumerate}

Temos mais uma vez o efeito de quatro níveis de aninhamento.
O comando \textbackslash item aceita o formato \textbackslash item[text], que
irá fazer com que qualquer coisa apareça como marcador.

\section{Customizando listas}
Vamos ver outros ambientes de listagem.
Estes podem ser utilizados carregando um pacote chamado \emph{paralist}.

\noindent \emph{No preâmbulo adicionar:\newline
\noindent\textbackslash usepackage\{paralist\}\newline}
\textbackslash begin\{compactenum\}\newline
\textbackslash item Abra o terminal.\newline
\textbackslash item Invoque o vim.\newline
\textbackslash begin\{compactitem\}\newline
\textbackslash item Salve seu arquivo com o nome e extensão corretos.\newline
\textbackslash item Escreva seu texto.\newline
\textbackslash end\{compactitem\}\newline
\textbackslash item Salve o arquivo e saia.\newline
\textbackslash end\{compactenum\}\newline

Perceba no resultado o espaçamento menor, graças ao estilo compacto que o
pacote nos traz.

\begin{compactenum}
\item Abra o terminal.
\item Invoque o vim.
\begin{compactitem}
\item Salve seu arquivo com o nome e extensão corretos.
\item Escreva seu texto.
\end{compactitem}
\item Salve o arquivo e saia.
\end{compactenum}

\subsection{Listas alinhadas}
Ao invés de serem itens colocados um abaixo do outro, serão itens escritos na
mesma linha.

\noindent \emph{No preâmbulo adicionar:\newline
\noindent\textbackslash usepackage\{paralist\}\newline}
\textbackslash begin\{compactenum\}\newline
\textbackslash item Abra o terminal.\newline
\textbackslash item Invoque o vim:\newline
\textbackslash begin\{\textbf{inparaenum}\}\newline
\textbackslash item Salve seu arquivo com o nome e extensão corretos.\newline
\textbackslash item Escreva seu texto.\newline
\textbackslash item Refatore seu texto.\newline
\textbackslash end\{\textbf{inparaenum}\}\newline
\textbackslash item Salve o arquivo e saia.\newline
\textbackslash end\{compactenum\}\newline

Desta vez estamos colocando uma lista que reside em um parágrafo ao invés de uma sucessão de linhas.

\begin{compactenum}
\item Abra o terminal.
\item Invoque o vim:
\begin{inparaenum}
\item Salve seu arquivo com o nome e extensão corretos.
\item Escreva seu texto.
\item Refatore seu texto.
\end{inparaenum}
\item Salve o arquivo e saia.
\end{compactenum}

Para cada ambiente padrão, o pacote paralist adiciona três ambientes correspondentes:

Listas numeradas:
\begin{itemize}
	\item compactenum: Já visto.
	\item inparaenum: Já visto.
	\item asparaenum: Supostamente cada item é tratado como um parágrafo do LaTeX.
\end{itemize}

Listas simples:
\begin{itemize}
	\item compactitem: Já visto.
	\item inparaitem: Fácil de entender.
	\item asparaitem: Supostamente o mesmo que asparaenum.
\end{itemize}

\section{Escolhendo estilos de números e itens}
Vamos trocar essas bolinhas chatas e esses números sem personalidade.
Vamos usar o pacote enumitem.

No sumário precisamos adicionar:
\textbackslash usepackage\{enumitem\}

Então, escrevemos um pouco de configurações que podem ficar em qualquer lugar. Deixar no preâmbulo é bom para manter a organização.

\noindent\textbackslash setlist\{nolistsep\}\newline
\textbackslash setitemize[1]\{label=---\}\newline
\textbackslash setitemize[2]\{label=1.1\}\newline
\textbackslash setenumerate[1]\{label=\textbackslash textcircled\{\textbackslash scriptsize\textbackslash Alph*\},font=\textbackslash sffamily\}\newline
\newline
Por fim, adicionamos nossa lista normalmente.\newline
\textbackslash begin\{enumerate\}\newline
\textbackslash item Item um, item dois.\newline
\textbackslash item Item três e quatro.\newline
\textbackslash begin\{itemize\}\newline
\textbackslash item geometria.\newline
\textbackslash item matemática.\newline
\textbackslash begin\{itemize\}\newline
\textbackslash item alguma coisa\newline
\textbackslash end\{itemize\}\newline
\textbackslash end\{itemize\}\newline
\textbackslash item Item cinco e seis.\newline
\textbackslash end\{enumerate\}\newline


\setlist{nolistsep}
\setitemize[1]{label=---}
\setitemize[2]{label=1.1}
\setenumerate[1]{label=\textcircled{\scriptsize\Alph*},font=\sffamily}

\begin{enumerate}
\item Item um, item dois.
\item Item três e quatro.
\begin{itemize}
\item geometria.
\item matemática.
	\begin{itemize}
		\item alguma coisa
	\end{itemize}
\end{itemize}
\item Item cinco e seis.
\end{enumerate}

Para ser 100\% sincero, não espero dominar esse tipo de customização, apenas
saber que existe para poder pesquisar depois à medida que preciso.

Se ao terminar uma listagem, chamarmos novamente o \textbackslash begin\{item\}[resume*], continuamos a partir da contagem da última lista do mesmo tipo iniciada.

Vimos nas configurações que o \textbackslash setenumerate aceita argumentos no formato chave=valor. 

EU AINDA NÃO SEI USAR TABELAS DIREITO, MAS QUE EU SAIBA O FORMATO É MAIS OU MENOS ESSE.
NÃO ESTÁ PRONTA AINDA, MAS PELO MENOS ESTÁ MEIO FORMATADO.

%\begin{tabular}
%	Parametro & Significado & Valores & Exemplo \\
%	font & Modifica o rótulo da fonte & Qualquer comando de fonte & font=\textbackslash bfseries \\
%	label & Configura o rótulo para o nível corrente & Pode conter: \textbf{\textbackslash arabic*, \textbackslash alph*, \textbackslash Alph*,  \textbackslash roman, \textbackslash Roman*} & label=\textbackslash emph \textbackslash alph* \\
%	label* & Como label normal, mas adicionado ao nível corrente. & Semelhante ao label & label*=\textbackslash arabic \\
%	align & Alinhamento do rótulo & left ou right & align=right \\
%	start & Número do primeiro item & Integer & start=10 \\
%	resume, resume* & Permite que o contador siga a partir da última contagem &  & resume \\
%	noitemsep & Sem espaço extra entre os itens e parágrafos &  & noitemsep \\
%	nolistsep & Sem espaço extra em qualquer lugar &  & nolistsep \\
%\end{tabular}

Essas opções podem ser configuradas globalmente com \textbackslash
setenumerate[level]{chave=valor lista}. Se a variável level não for definida,
será aplicado a todos os níveis.

Para retomar uma contagem basta usar \textbackslash begin\{enumerate\}[resume*].

\section{Listas de definição}
Listas de definição são listas com o primeiro termo sendo um título.

\noindent\textbackslash begin\{description\textbackslash \}\\
	\textbackslash item[paralist] provém listas compactas e versões de listas que podem ser utilizadas em parágrafos, além de ajudar a customizar rótulos e layout.\\
	\textbackslash item[enumitem] dá controle sobre rótulos e comprimentos de todos os tipos de lista.\\
	\textbackslash item[mdwlist] é útil para customizar listas de descrição. Permite até mesmo rótulos multi-linha. Entrega listas compactas e a capacidade de suspender e retomar a contagem.\\
	\textbackslash item[desclist] oferece mais flexibilidade à listas de definição.\\
	\textbackslash item[multenum] produz uma lista enumerada vertical em múltiplas colunas.\\
\textbackslash end\{description\textbackslash \}\\

\begin{description}
	\item[paralist] provém listas compactas e versões de listas que podem ser utilizadas em parágrafos, além de ajudar a customizar rótulos e layout.
	\item[enumitem] dá controle sobre rótulos e comprimentos de todos os tipos de lista.
	\item[mdwlist] é útil para customizar listas de descrição. Permite até mesmo rótulos multi-linha. Entrega listas compactas e a capacidade de suspender e retomar a contagem.
	\item[desclist] oferece mais flexibilidade à listas de definição.
	\item[multenum] produz uma lista enumerada vertical em múltiplas colunas.
\end{description}

O pacote paralist também admite listas de descrições compactas com compactdesc e outras versões.

\section{Alterando algumas dimensões}
Podemos adicionar o pacote layouts, e utilizar um comando para visualizar o
layout de páginas. Este pacote é capaz de mais coisas além de controlar listas,
mas por enquanto vamos nos manter no tópico. 
O comando é \textbackslash listdiagram.

\listdiagram

\newpage
