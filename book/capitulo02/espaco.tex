\chapter{Manipulando mais o espaço}
No capítulo anterior vimos como manipular o formato, tamanho, fonte e família do texto,
além de vermos usabilidade e sintaxe da linguagem de marcação.
Ao final, vimos manipulações de espaço em que o texto é confinado.
Pense no espaço como sendo uma sentença, linha, parágrafo, seção,
capítulo, título, ou livro.
Temos muitos e muitos exemplos de espaços.
Imagine também uma revista com dupla coluna,
ou um jornal que possui caixas de texto menores.

Com isso em mente, pense também no espaço entre as linhas,
e como as palavras são quebradas ao fim das linhas, com hífens.
Reflita sobre o espaçamento entre uma linha e outra.

CONTINUAR AQUI
