%TexFile
\chapter{Manipulando o espaço}
No capítulo anterior vimos como manipular o formato, tamanho, fonte e família do texto,
além de vermos usabilidade e sintaxe da linguagem de marcação.
Ao final, vimos manipulações de espaço em que o texto é confinado.
Pense no espaço como sendo uma sentença, linha, parágrafo, seção,
capítulo, título, ou livro.
Temos muitos e muitos exemplos de espaços.
Imagine também uma revista com dupla coluna,
ou um jornal que possui caixas de texto menores.

Com isso em mente, pense também no espaço entre as linhas,
e como as palavras são quebradas ao fim das linhas, com hífens.
Reflita sobre o espaçamento entre uma linha e outra.

\section{Estilizando espaços}
Vamos iniciar o assunto.
Existe um comando chamado parbox.
Este cria uma caixa que abriga um parágrafo.
O uso é dado da seguinte forma:\\
\textbackslash parbox [alinhamento]\{largura\}\{Texto.\}
\\

Temos aqui uma linha.
\quad\parbox[b]{2.5cm}{Este parbox está alinhado à base}
\quad\parbox{2.5cm}{Este está alinhado ao centro}
\quad\parbox[t]{2.5cm}{Alinhado ao topo da linha}

\fbox{\parbox[b]{6cm}{Parbox padrão, dentro de uma framebox} }

O \textbackslash parbox delimita e altera o espaço que um parágrafo é confinado.
Vimos nas linhas acima o uso de quatro \textbackslash parbox diferentes.
Nos primeiros três foram exemplificados os diferentes alinhamentos que a uma \textbackslash parbox pode ter em relação à linha guia.

No mais abaixo, foi adicionado o comando \textbackslash fbox\{\}
para poder visualizar os limites que \textbackslash parbox\{\} cria.
Não é exatamente um uso que se vê uma utilidade imediata,
mas com certeza traz mais liberdade e possibilidade de expressão.

\section{Terminando o assunto}
Quando usamos o \textbackslash parbox\{\}, usamos algo que cria uma espécie de contêiner para um parágrafo.
Vejamos um tipo de ambiente com declaração, um ambiente pré-definido interessante.
Nosso exemplo, dentro do arquivo tex é:

\textbackslash begin\{minipage\}\{3cm\}

\textbackslash hyphenation\{Signi-fica\}

TUG é um acrônimo. Significa \textbackslash TeX\textbackslash Users Grupo. Largura de 3cm.

\textbackslash end\{minipage\}

\vspace{0.3cm}
Ilustrando o exemplo temos:
\vspace{0.3cm}

\begin{minipage}{3cm}
\hyphenation{Signi-fica}
TUG é um acrônimo. Significa \TeX\ Users Grupo. Largura de 3cm.
\end{minipage}

\vspace{0.3cm}

O ambiente minipage causa um efeito semelhante ao parbox, no entanto, este possui sintaxe que é aplicável à páginas.
Isso significa que esta poderia, em tese, possuir um cabeçalho, rodapé, e até mesmo algum tipo de título de capítulo.
No entanto, não é exatamente muito útil querer desenhar em uma minipage tantas coisas.
O mais eficiente é aprender a mexer em páginas normais e eventualmente se aprofundar em tópicos que interessem.

\section{Unindo e fragmentando palavras}
No exemplo anterior vimos o uso de uma função \textbackslash hyphenation\{\}.
Em palavras simples, ela força que o hífen, separador quando uma palavra atinge o limite da página,
apareça dividindo a palavra em um certo trecho.
Normalmente damos preferência para separação silábica, mas não precisamos nos limitar a isso.

Esta discussão sobre separação silábica nos dá uma reflexão importante.
Só faz sentido falar sobre palavras hifenizadas quando estamos no formato justificado de texto.
Esse formato nasceu da necessidade de se utilizar ao máximo o espaço escaço do papel.
Mesmo papel sendo barato, uma gráfica de jornais não se pode dar ao luxo de gastar várias linhas de espaço,
quando pode condensar mais o texto.

Na visualização em internet, como não temos a limitação do papel, a palavra inteira pode ser jogada para a linha debaixo.
Com isso temos o efeito de que cada linha possui um comprimento diferente, e em certos casos, a leitura fica mais fluida.

Para inibirmos a hifenização de pequenos trechos podemos aplicar \textbackslash nohyphens\{\}.
Enfim, para aumentarmos as possibilidades de customização, podemos importar e ler a documentação do pacote hyphenat.

\section{Espaçando palavras}
Deixando apenas como citação, para futura busca e pequisa.
O pacote microtype trata dos espaçamentos entre letras.
Por isso, também pode ser capaz de solucionar um problema envolvendo palavras quebrando ao chegar no fim do espaço da linha.

\section{Quebras de linha}
Quebras de linha são especialmente delicadas.
Estão presentes quando uma frase continua para a linha debaixo,
se apresentam quando estamos indo de um parágrafo a outro,
e quando estamos declarando o título de um capítulo.
O espaço vertical entre uma linha e outra pode nos trazer facilidade visual na leitura,
como pode transformar esse exercício em tortura.

Vejamos um espaçamento manual de um poema de Poe.

CONTINUA NA LINHA 29 DO INFOS\_DO\_LIVRO02.

\newpage
