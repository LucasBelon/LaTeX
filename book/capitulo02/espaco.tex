%TexFile
\chapter{Manipulando o espaço}
No capítulo anterior vimos como manipular o formato, tamanho, fonte e família do texto,
além de vermos usabilidade e sintaxe da linguagem de marcação.
Ao final, vimos manipulações de espaço em que o texto é confinado.
Pense no espaço como sendo uma sentença, linha, parágrafo, seção,
capítulo, título, ou livro.
Temos muitos e muitos exemplos de espaços.
Imagine também uma revista com dupla coluna,
ou um jornal que possui caixas de texto menores.

Com isso em mente, pense também no espaço entre as linhas,
e como as palavras são quebradas ao fim das linhas, com hifens.
Reflita sobre o espaçamento entre uma linha e outra.

\section{Estilizando espaços}
Vamos iniciar o assunto. Existe um comando chamado parbox. Este cria uma caixa que abriga um parágrafo. O uso é dado da seguinte forma:
\newline
\noindent \textbackslash parbox [alinhamento]\{largura\}\{Texto.\}
\newline
Temos aqui uma linha.
\fbox{
\quad\parbox[b]{2.5cm}{Este parbox está alinhado à base}}
\fbox{\quad\parbox{2.5cm}{Este está alinhado ao centro}}
\fbox{\quad\parbox[t]{2.5cm}{Alinhado ao topo da linha}}
que em seguida volta ao normal. Perceba o espaçamento acima e abaixo.

O \textbackslash parbox delimita e altera o espaço que um parágrafo é confinado.
Vimos nas linhas acima o uso de quatro \textbackslash parbox diferentes.
Nos primeiros três foram exemplificados os diferentes alinhamentos que a uma \textbackslash parbox pode ter em relação à linha guia.

Em todos foi adicionado o comando \textbackslash fbox\{\}
para poder visualizar os limites que \textbackslash parbox\{\} cria.
Não é exatamente um uso que se vê uma utilidade imediata,
mas com certeza traz mais liberdade e possibilidade de expressão.

\section{Terminando o assunto}
Quando usamos o \textbackslash parbox\{\}, usamos algo que cria uma espécie de contêiner para um parágrafo.
Vejamos um tipo de ambiente com declaração, um ambiente pré-definido interessante.
Nosso exemplo, dentro do arquivo tex é:

\textbackslash begin\{minipage\}\{3cm\}

\textbackslash hyphenation\{Signi-fica\}

TUG é um acrônimo. Significa \textbackslash TeX\textbackslash Users Grupo. Largura de 3cm.

\textbackslash end\{minipage\}

\vspace{0.3cm}
Ilustrando o exemplo temos:
\vspace{0.3cm}

\begin{minipage}{3cm}
\hyphenation{Signi-fica}
TUG é um acrônimo. Significa \TeX\ Users Grupo. Largura de 3cm.
\end{minipage}

\vspace{0.3cm}

O ambiente minipage causa um efeito semelhante ao parbox, no entanto, este possui sintaxe que é aplicável à páginas.
Isso significa que esta poderia, em tese, possuir um cabeçalho, rodapé, e até mesmo algum tipo de título de capítulo.
No entanto, não é exatamente muito útil querer desenhar em uma minipage tantas coisas.
O mais eficiente é aprender a mexer em páginas normais e eventualmente se aprofundar em tópicos que interessem.

\section{Unindo e fragmentando palavras}
No exemplo anterior vimos o uso de uma função \textbackslash hyphenation\{\}.
Em palavras simples, ela força que o hífen, separador quando uma palavra atinge o limite da página,
apareça dividindo a palavra em um certo trecho.
Normalmente damos preferência para separação silábica, mas não precisamos nos limitar a isso.

Esta discussão sobre separação silábica nos dá uma reflexão importante.
Só faz sentido falar sobre palavras hifenizadas quando estamos no formato justificado de texto.
Esse formato nasceu da necessidade de se utilizar ao máximo o espaço escaço do papel.
Mesmo papel sendo barato, uma gráfica de jornais não se pode dar ao luxo de gastar várias linhas de espaço,
quando pode condensar mais o texto.

Na visualização em internet, como não temos a limitação do papel, a palavra inteira pode ser jogada para a linha debaixo.
Com isso temos o efeito de que cada linha possui um comprimento diferente, e em certos casos, a leitura fica mais fluida.

Para inibirmos a hifenização de pequenos trechos podemos aplicar \textbackslash nohyphens\{\}.
Enfim, para aumentarmos as possibilidades de customização, podemos importar e ler a documentação do pacote hyphenat.

\section{Espaçando palavras}
Deixando apenas como citação, para futura busca e pequisa.
O pacote microtype trata dos espaçamentos entre letras.
Por isso, também pode ser capaz de solucionar um problema envolvendo palavras quebrando ao chegar no fim do espaço da linha.

\section{Quebras de linha}
Quebras de linha são especialmente delicadas.
Estão presentes quando uma frase continua para a linha debaixo,
se apresentam quando estamos indo de um parágrafo a outro,
e quando estamos declarando o título de um capítulo.
O espaço vertical entre uma linha e outra pode nos trazer facilidade visual na leitura,
como pode transformar esse exercício em tortura.

Vejamos um espaçamento manual de um poema de Poe.
\newline

\emph{Annabel Lee} \\[3mm]
It was many and many years ago, \\
In a kingdom by the sea. \\
That a maiden there lived whom you may know \\
By the name of Annabel Lee
\newline
\newline
Foi utilizado um argumento após \textbackslash \textbackslash\ (para quebra de linha), da seguinte forma [3mm].
Testando com outro espaçamento:\newline
\newline
\emph{Annabel Lee} \\[5mm]
It was many and many years ago, \\[3mm]
In a kingdom by the sea. \\[10mm]
That a maiden there lived whom you may know \\[15mm]
By the name of Annabel Lee
\newline
\newline
\textbackslash \textbackslash\ Possui o mesmo efeito de \textbackslash newline.
No entanto, apenas o primeiro aceita o argumento para personalização de comprimento vertical.

O comando para nova linha, mantendo a justificação do texto é \textbackslash linebreak.
Pode causar alongamento na distância entre palavras, por isso, raramente é usado.

Utilizar \textbackslash\textbackslash*[5mm] por exemplo, impede a quebra de página antes da próxima linha.

Em \textbackslash linebreak[number], o parâmetro number indica a prioridade quanto à intenção de quebra de linha.
Quando o número é zero, a quebra de linha é permitida.
Número 1 indica que a quebra é desejada.
2 ou 3 indicam requisição de insistência, e 4 força a quebra de linha.
Quando usado sem o parâmetro, o valor 4 é utilizado por padrão.

Podemos prevenir quebras de linha com \textbackslash nolinebreak[number].
A lógica do parâmetro number é idêntica ao \textbackslash linebreak, mas indicando o inverso.

Caso seja nossa intenção manter o texto sem hifens e quebra de linha, podemos nos utilizar do comando \textbackslash mbox[texto].

\section{Espaço entre letras e palavras}
Em títulos pessoais, desejamos que estes sejam mantidos na mesma linha, em "Dr. Watson" vemos um exemplo claro.

Em títulos pessoais, desejamos que estes sejam mantidos na mesma linha, em "Dr.~Watson" vemos um exemplo claro.

Aplicamos no exemplo acima "Dr.\~\ Watson"
\footnote{O til deve ficar imediatamente entre o . e o W para surtir efeito.}, desta forma, o começo do nome se vinculou ao caractere anterior.

\subsection{Explorando Ligaduras}
Ligaduras são um elemento da caligrafia, referentes à como duas letras se encontram.
No exemplo abaixo percebemos as ligaduras entre as letras f, i e l.

ff fi fl flffifl -- ---

f\/f f\/i f\/l f\/l\/f\/f\/l\/f\/l -\/- -\/-\/-

f\textbackslash/f f\textbackslash/i f\textbackslash/l
f\textbackslash/l\textbackslash/f\textbackslash/f\textbackslash/l\textbackslash/f\textbackslash/l
-\textbackslash/-
-\textbackslash/-\textbackslash/-

O comando \textbackslash/ está impedindo ligaduras de acontecerem.

O mesmo efeito pode ser alcançado usando \{\},
pois os espaços devem respeitar os limites do ambiente.

Para desligar as ligaduras de letras, pode-se passar o argumento noligature para o pacote microtype.
\textbackslash usepackage[noligatures]\{microtype\}

\subsection{Expressando hifens}
Cada quantidade de hifens utilizados servem a um propósito.

Um único hífen serve para dividir palavras ou para palavras compostas.
Vejamos o exemplo guarda-chuva, para-raios.

Dois hifens seguidos servem como separador de intervalo, como
"Das 9--10hrs da manhã".
A largura deste hífen é equivalente a um dígito.


Três hifens são convertidos para o travessão, usados principalmente no início de falas de personagens em livros.

--- Vim aqui falar sobre minhas intenções.

\subsection{Expressando pontos}
Algo semelhante ocorre com pontos.
Reticências são menos espaçadas entre si do que um ponto 
e o caractere subsequente.

Usar \textbackslash @. indica ao LaTeX que o ponto se situa no fim da frase.
Vamos ilustrar com \@.

Donald E. Knuth foi o criador do \TeX. Consegue perceber a diferença?

Donald E. Knuth foi o criador do \TeX\@. Consegue perceber a diferença?

Para inserir reticências, use \textbackslash ldots \ldots

Veja a diferença entre três pontos seguidos: ...

\subsection{Espaçamento por pacote}
Outra forma de se alterar os espaçamentos é com a utilização de pacotes.
Como já explicado, adiciona-se ao preâmbulo, e utiliza-se parâmetros para definição.

Não é exatamente o assunto mais emocionante, e geralmente o espaçamento padrão é suficientemente bom para utilização em textos.
Por conta disso, utilize as fontes citadas no começo para fazer uma pesquisa.


\section{Alinhado à esquerda, direita, centro}
Em textos de direito, citações ficam alinhadas à direita.
Em sites de internet, o texto se alinha à esquerda, nunca quebrando palavras.
Em trabalhos maiores como livros, publicações em revistas,
jornais e afins, procura-se usar o espaço da maneira mais eficiente possível.

Existem variações de sintaxe para justificação de texto.
Geralmente adiciona-se esse detalhe de justificação no preâmbulo.
Vejamos cada um deles.

\fbox{\parbox{3cm}{\raggedright
	TUG é um acrônimo. Significa \TeX\ user's group.
}}
\fbox{\parbox{3cm}{\raggedleft
	TUG é um acrônimo. Significa \TeX\ user's group.
}}
\fbox{\parbox{3cm}{\centering
	TUG é um acrônimo. Significa \TeX\ user's group.
}}

Criamos uma parbox, com um framebox para percebermos onde o texto se alinha.
Outro exemplo pode ser entendido forçando um ambiente a ser centralizado com diferentes linhas.

{\centering
	texto \\
	que eu mesmo\\
	escrevi\\
	e espero\\
	que fique\\
	centralizado\\
}

{\raggedright
	texto \\
	que eu mesmo\\
	escrevi\\
	e espero\\
	que fique\\
	Alinhado à esquerda\\
}

{\raggedleft
	texto \\
	que eu mesmo\\
	escrevi\\
	e espero\\
	que fique\\
	Alinhado à direita\\
}

Podemos também utilizar declarações de ambiente com \textbackslash begin e \textbackslash end.
Testamos os seguintes ambientes: flushright, center, flushleft, respectivamente.

\begin{flushright}
\fbox{Mais textos experimentando com alinhamentos}
\end{flushright}

\begin{center}
\fbox{Experimentando mais uma vez}
\end{center}

\begin{flushleft}
\fbox{Última tentativa}
\end{flushleft}

A diferença entre usar os ragged/centering,
e usar flush/center não está muito clara,
mas ter opção é melhor do que não ter.

\section{Citações}
% AINDA PRECISO CORRIGIR ESTE TRECHO. NÃO ESTÁ COMPLETO, E ESTÁ FEIO.
% TODO: Corrigir aqui
Citações geralmente alteram fonte, tamanho, às vezes família, e disposição do texto.

Exemplo de citação:

Niels Bohr said: ``An expert is a person who has made
all the mistakes that can be made in a very narrow field.''

Albert Einstein said:

\begin{quote}
    Anyone who has never made a mistake has never tried anything new.
\end{quote}
Errors are inevitable. So, let's be brave trying something new.


Exemplo de citação longa:

The authors of the CTAN team listed ten good reasons
for using \TeX. Among them are:
\begin{quotation}
 \TeX\ has the best output. What you end with,
the symbols on the page, is as useable, and beautiful,
as a non-professional can produce.
 \TeX\ knows typesetting. As those plain text samples
show, \TeX's has more sophisticated typographical algorithms
such as those for making paragraphs and for hyphenating.
 \TeX\ is fast. On today's machines \TeX\ is very fast.
 It is easy on memory and disk space, too.
 \TeX\ is stable. It is in wide use, with a long history.
 It has been tested by millions of users, on demanding input.
 It will never eat your document. Never.
\end{quotation}
The original text can be found on
\textbackslash url\{http://www.ctan.org/what\_is\_tex.html\}.

Adicionar o pacote parskip irá adicionar um espaço entre os parágrafos.
Existem duas formas de se distinguir parágrafos.
Uma, a padrão, é indentar o começo de cada parágrafo.
A segunda é adicionando um espaço vertical entre parágrafos.
pode ser útil no caso de colunas estreitas, em que a indentação seria custosa ou anti-estilística.


\newpage
