\documentclass[a4paper, 12pt, oneside]{book}
%---------Pacotes Usados-----------------+
\usepackage{amsmath,                    %|
            amssymb,                    %|
            amsfonts,                   %|
            latexsym,                   %|
            mathrsfs,                   %|
            amsthm,                     %|
            amstext,                    %|
            bezier,                     %|
            amscd}                      %|
                                        %|
\usepackage[top=2cm,                    %|
            bottom=2cm,                 %|
            left=2.5cm,                 %|
            right=2.5cm]                %|
            {geometry}                  %|
                                        %|
\usepackage{xcolor}                     %|
                                        %|
\usepackage{soul}                       %|
                                        %|
\usepackage[utf8]{inputenc}             %|
                                        %|
\usepackage{indentfirst}                %|
                                        %|
\usepackage{graphicx}                   %|
%----------------------------------------+


\begin{document}
%---------Infos MakeTitle----------------+
\title{%                                %|
    \textbf{                            %|
    Anotações Definitivas               %|
    }%                                  %|
    \break                              %|
    Aprendendo a usar VIM               %|
}                                       %|
                                        %|
\author{                                %|
    Lucas Gouveia Belon                 %|
    \\                                  %|
    lucasgouveiabelon@gmail.com         %|
}                                       %|
\date{\vspace{2.2cm}São Paulo\\2023}    %|
                                        %|
\maketitle                              %|
\newpage                                %|
%----------------------------------------+


%---------Sumário------------------------+
\tableofcontents                        %|
\thispagestyle{empty}
\newpage                                %|
%----------------------------------------+


% Escrita começa de fato aqui ------------------------
\chapter{Início}
Este texto se trata de um resumo e tradução do livro Latex begginer's guide de Stefan Kottwitz.

\section{Caracteres Especiais}
Existem muitos caracteres que são usados na sintaxe do \LaTeX.
Por conta disso, existem formas específicas de escrevê-los:
\begin{itemize}
	\item \textbackslash \# para obter \# ;
	\item \textbackslash \% para obter \% ;
	\item \textbackslash \$ para obter \$ ;
	\item \textbackslash \_ para obter \_ ;
	\item \textbackslash \& para obter \& ;
	\item \textbackslash \{ para obter \{ ;
	\item \textbackslash \} para obter \} ;
	\item \textbackslash textbackslash para obter \textbackslash\ ;
\end{itemize}

\section{Itálico, negrito e outros}
Existem algumas formas de alterar o texto.
Escrever \textbackslash textit\{texto italizado\} é um tanto ruim,
mas depois de feito, ao menos é mais fácil encontrar o texto enfatizado.

\textit{italizado: abcdefghijklmnopqrstuvwxyz}

\textit{ITALIZADO: ABCDEFGHIJKLMNOPQRSTUVWXYZ}

Inclusive, podemos enfatizar com \textbackslash emph\{texto enfatizado\}.
O efeito se parece com \emph{Texto enfatizado}.
A depender de certas configurações, o texto enfatizado é idêntico ao italizado.

\emph{enfatizado: abcdefghijklmnopqrstuvwxyz}

\emph{ENFATIZADO: ABCDEFGHIJKLMNOPQRSTUVWXYZ}

Também temos com \textbackslash textbf\{texto em negrito\} como deixar em negrito.
\textbf{Texto em negrito}.

\textbf{negrito: abcdefghijklmnopqrstuvwxyz}

\textbf{NEGRITO: ABCDEFGHIJKLMNOPQRSTUVWXYZ}

Temos por fim um efeito chamado slanted. \textbackslash textsl\{texto slanted\}.
\textsl{Texto slanted}. É como um itálico, mas com outra fonte:

\textsl{abcdefghijklmnopqrstuvwxyz}

\textsl{ABCDEFGHIJKLMNOPQRSTUVWXYZ}


\section{Ambientes, fontes e famílias}
Dois conceitos importantes no \LaTeX\ são os environments,
em que se aplicam efeitos, e a dupla, declarações e funções.
Ambientes delimitam os efeitos de declarações.
Quando escrevemos \{ \}, estamos criando um ambiente.
Temos ambientes pré-definidos com \textbackslash begin\{efeito\} terminando com \textbackslash end.

Declarações são efeitos que queremos aplicar no texto, podendo ter um início mas não obrigatoriamente tendo um fim.
Funções geralmente definem um ambiente para seus efeitos, já ditando o fim na aplicação.

Uma aplicabilidade desses conceitos é a mudança de fonte.
Podemos mudar a fonte de um pequeno trecho de texto, assim como podemos alterá-la sem previsão de voltar à fonte padrão.

Para usarmos uma família de fontes usando uma declaração num ambiente, utilizamos a seguinte sintaxe:
\{ \textbackslash itshape \textbackslash bfseries Texto de exemplo\}.
O efeito é de {\itshape \bfseries Texto de exemplo}.
Este tipo de aplicação é útil quando desejamos utilizar uma declaração que não possui função associada,
ou mesmo quando desejamos misturar e aninhar efeitos, ficando geralmente mais claro quais são os efeitos aplicados.
Como o texto se mistura com pequenas palavras chave, como em código, é bom manter uma conduta de escrita que auxilie a legibilidade.

Eu não sabia antes de procurar sobre tópicos de \LaTeX\, mas existem famílias de fontes.
As mais comuns são a família roman, a família sans-serif, e tipewriter.

Abaixo segue uma tabela de quais os comandos e declarações para diferentes fontes e famílias.

\begin{center}
\begin{tabular}{ c c c }
Comando & Declaração & Significado \\
\textbackslash textrm\{...\}&\textbackslash rmfamily&roman family\\
\textbackslash textsf\{...\}&\textbackslash sffamily&sans-serif family\\
\textbackslash texttt\{...\}&\textbackslash ttfamily&tipewriter family\\
\textbackslash textbf\{...\}&\textbackslash bfseries&bold-face\\
\textbackslash textmd\{...\}&\textbackslash mdseries&medium\\
\textbackslash textit\{...\}&\textbackslash itshape&italic shape\\
\textbackslash textsl\{...\}&\textbackslash slshape&slanted shape\\
\textbackslash textsc\{...\}&\textbackslash scshape&Small Caps Shape\\
\textbackslash textup\{...\}&\textbackslash upshape&Upright Shape\\
\textbackslash textnormal\{...\} & \textbackslash normalfont & Default font\\
\end{tabular}
\end{center}

\vspace{10pt}

Agora, experimentando cada tipo de fonte:

\subsection{Roman Family}
\textrm{abcdefghijklmnopqrstuvwxyz}

\textrm{ABCDEFGHIJKLMNOPQRSTUVWXYZ}

\subsection{Sans-serif Family}
\textsf{abcdefghijklmnopqrstuvwxyz}

\textsf{ABCDEFGHIJKLMNOPQRSTUVWXYZ}

\subsection{Tipewriter Family}
\textrm{abcdefghijklmnopqrstuvwxyz}

\textrm{ABCDEFGHIJKLMNOPQRSTUVWXYZ}

\subsection{Bold-Face}
\textbf{abcdefghijklmnopqrstuvwxyz}

\textbf{ABCDEFGHIJKLMNOPQRSTUVWXYZ}

\subsection{Medium}
\textmd{abcdefghijklmnopqrstuvwxyz}

\textmd{ABCDEFGHIJKLMNOPQRSTUVWXYZ}

\subsection{Italic Shape}
\textit{abcdefghijklmnopqrstuvwxyz}

\textit{ABCDEFGHIJKLMNOPQRSTUVWXYZ}

\subsection{slanted shape}
\textsl{abcdefghijklmnopqrstuvwxyz}

\textsl{ABCDEFGHIJKLMNOPQRSTUVWXYZ}

\subsection{Small Caps Shape}
\textsc{abcdefghijklmnopqrstuvwxyz}

\textsc{ABCDEFGHIJKLMNOPQRSTUVWXYZ}

\subsection{Upright Shape}
\textup{abcdefghijklmnopqrstuvwxyz}

\textup{ABCDEFGHIJKLMNOPQRSTUVWXYZ}

\subsection{Default Font}
\textnormal{abcdefghijklmnopqrstuvwxyz}

\textnormal{ABCDEFGHIJKLMNOPQRSTUVWXYZ}



\end{document}
