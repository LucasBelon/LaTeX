\documentclass[a4paper, 12pt, oneside]{book}
%---------Pacotes Usados-----------------+
\usepackage{amsmath,                    %|
            amssymb,                    %|
            amsfonts,                   %|
            latexsym,                   %|
            mathrsfs,                   %|
            amsthm,                     %|
            amstext,                    %|
            bezier,                     %|
            amscd}                      %|
                                        %|
\usepackage[top=2cm,                    %|
            bottom=2cm,                 %|
            left=2.5cm,                 %|
            right=2.5cm]                %|
            {geometry}                  %|
                                        %|
\usepackage{xcolor}                     %|
                                        %|
\usepackage{soul}                       %|
                                        %|
\usepackage[utf8]{inputenc}             %|
                                        %|
\usepackage{indentfirst}                %|
                                        %|
\usepackage{graphicx}                   %|
%----------------------------------------+


\begin{document}
%---------Infos MakeTitle----------------+
\title{%                                %|
    \textbf{                            %|
    Anotações Definitivas               %|
    }%                                  %|
    \break                              %|
    Aprendendo a usar VIM               %|
}                                       %|
                                        %|
\author{                                %|
    Lucas Gouveia Belon                 %|
    \\                                  %|
    lucasgouveiabelon@gmail.com         %|
}                                       %|
\date{\vspace{2.2cm}São Paulo\\2023}    %|
                                        %|
\maketitle                              %|
\newpage                                %|
%----------------------------------------+


%---------Sumário------------------------+
\tableofcontents                        %|
\thispagestyle{empty}
\newpage                                %|
%----------------------------------------+


% Escrita começa de fato aqui ------------------------
\chapter{Início}
Existem muitos caracteres que são usados na sintaxe do \LaTeX.
Por conta disso, existem formas específicas de escrevê-los:
\begin{itemize}
	\item \textbackslash \# para obter \# ;
	\item \textbackslash \% para obter \% ;
	\item \textbackslash \$ para obter \$ ;
	\item \textbackslash \_ para obter \_ ;
	\item \textbackslash \& para obter \& ;
	\item \textbackslash \{ para obter \{ ;
	\item \textbackslash \} para obter \} ;
	\item \textbackslash textbackslash para obter \textbackslash ;
\end{itemize}

Existem algumas formas de alterar o texto.
É um pouco ruim escrever \textbackslash textit\{texto_italizado\},
mas depois de feito, ao menos é mais fácil encontrar o texto enfatizado.
\textit{Texto em itálico}.

Inclusive, podemos enfatizar com \textbackslash emph\{texto_enfatizado\}.
O efeito se parece com \emph{Texto enfatizado}.

Também temos com \textbackslash emph\{texto_enfatizado\} como deixar em negrito.


\end{document}


