\documentclass{article}
\usepackage[utf8]{inputenc}

\begin{document}

Escrevendo caracteres especiais;
\begin{itemize}
        \item \#
        \item \%
        \item \$
        \item \_
        \item \&
        \item \{
        \item \}
        \textbackslash

\end{itemize}

texto pode ser \emph{enfatizado}
Pode também ser \textit{italico} ou \textbf{Negrito}
E mesmo ser encadeado \textbf{\textit{Italico e Negrito}}
\emph{Algumas formatações possuem \emph{comportamento específico} quando encadeados}
Os comandos de formatação de texto obedecem a sintaxe: 
\backslash text**. Os dois asteriscos são referentes aos efeitos:
\begin{intemize}
\item{text it: italic - itálico}
    \textit{Exemplo}
\item{text bf: bold face - negrito}
    \textbf{Exemplo}
\item{text sl: slanted - xxxxxx}
    \textsl{Exemplo}
\end{intemize}

Verificar depois a diferença entre vimtex e vim-latex.

\section{\textsf{Fontes sobre \latex\ na internet.}}
% \textsf Sans-seriff
Um dos melhores lugares para encontrar conteúdo sobre LaTeX é no site CTAN:
\texttt{https://www.ctan.org}
%\texttt{} typewriter text
%\textrm{} texto romano = texto com serifa. Fonte padrão.



Mudando a fonte e família durante longoa trechos:

\section{\sffamily\LaTeX\ resources in the internet}

The best place for downloading LaTeX related software is CTAN.

Its address is \ttfamily http://www.ctan.org\rmfamily.



\end{document}
