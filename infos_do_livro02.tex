\documentclass{article}
\usepackage{xspace}
\begin{document}


Usando um ambiente chamado minipage para obter um efeito semelhante ao parbox:

\begin{minipage}{3cm}
    \hyphenation{acro-nym}

    TUG is an acronym. It means \TeX\ Users Group. Largura de 3cm.

\end{minipage}

O comando hyphenation serve para, manualmente, indicar qual parte da palavra pode ser separada quando encontrado o fim do espaço de escrita.

Podemos usar o pacote hyphenat para aumentar as possibilidades.
Podemos prevenir hiphenização com
\usepackage[none]{hyphenat}
E permitir hiphenização de texto da fonte typewriter com
\usepackage[htt]{hyphenat}

Para pequenos trechos, podemos inibir a hyphenação com \nohyphens{pequeno trecho de texto}

O pacote microtype mexe no espaçamento para obter o efeito de texto justificado.
No exemplo dado, mexeria no microtype para tentar eliminar a necessidade de hiphens.

Vamos quebrar linhas manualmente, usando de exemplo um poema do Allan Poe.

\emph{Annabel Lee} \\[3mm]
It was many and many years ago, \\
In a kingdom by the sea. \\
That a maiden there lived whom you may know \\
By the name of Annabel Lee

O comando que possui o mesmo efeito de \\
é o comando \newline.

O comando que indica que deve ser adicionada uma nova linha,
mas mantendo a justificação do texto \linebreak, pode causar
alongamento demasiado na distância entre palavras.
Por isso, é raramente usado.

Podemos usar \\[3mm] para adicionar espaçamento vertical.
\\*[5mm] faz o mesmo, mas previne a quebra de página antes da
próxima linha de texto.

\linebreak[number] pode ser usado para influenciar a quebra de linha de forma sutil ou evidente.
Se number for zero, a quebra de linha é permitida, 1 significa que é desejada, 2 e 3 marcam outras requisições de insistência, e 4 irá forcar a quebra de linha.
Esta última é a opção padrão se não for especificado número.

Prevenindo quebras de linha.
\nolinebreak Faz o serviço.
Possui opções assim como o linebreak, sendo as requisições 
se tornando mais insistentes de 0 (recomenda não quebrar)
à 4 (proíbe de quebrar).

O comando \mbox[texto] desabilita a hyphenaçâo \textbf{e} a 
a quebra de linha do texto.

Quando duas palavras devem permanecer unidas na mesma linha utilizamos o símbolo \~.
Exemplo:
Dr. ~Watson previne que Dr. apareça sozinho no fim de uma linha

O comando para nota de rodapé é \footnote{Adicionei uma nota de rodapé}

% Explorando ligaduras -----------

ff fi fl flffifl -- ---

f\/f f\/i f\/l f\/l\/f\/f\/l\/f\/l -\/- -\/-\/-

O comando \/ está impedindo ligaduras de acontecerem.

O mesmo efeito pode ser alcançado usando {},
pois os espaços devem respeitar os limites do ambiente.

Para desligar as ligaduras de letras, pode-se passar o argumento noligature para o pacote microtype
% \usepackage[noligatures]{microtype}

Cada quantidade de hiphens utilizados servem a um propósito.

Um único hiphen serve para dividir palavras ou para palavras compostas.

Dois hiphens seguidos servem como separador de intervalo, como
"Das 9--10hrs da manhã".
A largura deste hiphen é equivalente a um dígito.

Três hiphens são convertidos para o travessão.

Algo semelhante ocorre com pontos.
reticências são menos espaçadas entre si do que um ponto 
e o caractere subsequente.

usar \@ indica ao LaTeX que o ponto se situa no fim de uma frase.

Para inserir reticências, use \ldots.

Pode-se alterar o espaçamento do texto adicionando ao preambulo:
\frenchspacing
O padrão é \nofrenchspacing

Vou deixar anotado qual a forma de se forçar acentos (já precisei uma vez).

\~{a}

Basicamente, backslash, acento, letra alvo entre chaves.
Mais de uma letra pode ser o alvo ao mesmo tempo.

No windows, se utf8 não funcionar, tentar latin1

Continuar na página 81. vamos ver como controlar a justificação do texto.
Estamos pra terminar esse pedaço do livro. O primeiro terço chegará ao fim.

\end{document}
