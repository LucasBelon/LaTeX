\documentclass{article}
\begin{document}

Página 60 do livro.
Vamos verificar um pouco do uso de pacotes.

\usepackage{xspace}
\section{The \TUG}

The \TUG is an organization for people
who are interested in \TeX or \LaTeX.

Teoricamente a adição do pacote xspace consertaria o problema dos espaços.

Vamos ver comandos com argumentos.

\newcommand{\keyword}[1]{\textbf{#1}}

O número entre colchetes é o número de argumentos.
\#1 é o primeiro parâmetro. Poderia haver um \#2 e assim
em diante.


\keyword{Grouping} by curly braces limits the

\keyword{scope} of \keyword{declarations}

Exemplo 2:

\newcommand{\keyword}[2][\bfseries]{{#1#2}}
\begin{document}
\keyword{Grouping} by curly braces limits the
\keyword{scope} of \keyword[\itshape]{declarations}.

Modo Genérico:
% \newcommand{command}[arguments][optional]{definition}

Existe um pacote chamado url para formatação de tipos para sites.

Podemos definir caixas para limitar verticalmente o texto.
Isso é particularmente útil no caso de procurarmos criar paragrafos semelhantes em tamanho.

\parbox{3cm}{TUG is an acronym. It means \TeX\ Users Group.}

Uso genérico do parbox \parbox[alignment]{width}{text}
\begin{itemize}
        \item Alinhamento: Argumento opcional para comprimento vertical.
        t alinha para a linha superior da caixa.
        b alinha para a linha inferior.
        \item Largura: Largira cuja caixa irá tomar. Pode ser dado em cm, em mm ou em in.
        \item Texto: Texto a ser adicionado dentro da caixa.
        Para textos mais complicados veremos outros métodos.
\end{itemize}

text line
\quad\parbox[b]{1.8cm}{this parbox is aligned at its bottom line}
\quad\parbox{1.5cm}{center-aligned parbox}
\quad\parbox[t]{2cm}{another parbox aligned at its top line}

\fbox{\parbox{Default parbox within fbox}}

Usando um ambiente chamado minipage para obter um efeito semelhante ao parbox:

\begin{minipage}{3cm}
    \hyphenation{acro-nym}

    TUG is an acronym. It means \TeX\ Users Group. Largura de 3cm.

\end{minipage}

O comando para nota de rodapé é \footnote{Adicionei uma nota de rodapé}





\end{document}
