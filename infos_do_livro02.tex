\documentclass{article}
\usepackage{xspace}
\begin{document}

Página 60 do livro.
Vamos verificar um pouco do uso de pacotes.

\section{The \TUG}

The \TUG is an organization for people
who are interested in \TeX or \LaTeX.

Teoricamente a adição do pacote xspace consertaria o problema dos espaços.

Vamos ver comandos com argumentos.

\newcommand{\keyword}[1]{\textbf{#1}}

O número entre colchetes é o número de argumentos.
\#1 é o primeiro parâmetro. Poderia haver um \#2 e assim
em diante.


\keyword{Grouping} by curly braces limits the

\keyword{scope} of \keyword{declarations}

Exemplo 2:

\renewcommand{\keyword}[2][\bfseries]{{#1#2}}
\keyword{Grouping} by curly braces limits the
\keyword{scope} of \keyword[\itshape]{declarations}.

Modo Genérico:
% \newcommand{command}[arguments][optional]{definition}

Existe um pacote chamado url para formatação de tipos para sites.

Podemos definir caixas para limitar verticalmente o texto.
Isso é particularmente útil no caso de procurarmos criar paragrafos semelhantes em tamanho.

\parbox{3cm}{TUG is an acronym. It means \TeX\ Users Group.}

Uso genérico do parbox \parbox[alignment]{width}{text}
\begin{itemize}
        \item Alinhamento: Argumento opcional para comprimento vertical.
        t alinha para a linha superior da caixa.
        b alinha para a linha inferior.
        \item Largura: Largira cuja caixa irá tomar. Pode ser dado em cm, em mm ou em in.
        \item Texto: Texto a ser adicionado dentro da caixa.
        Para textos mais complicados veremos outros métodos.
\end{itemize}

text line
\quad\parbox[b]{1.8cm}{this parbox is aligned at its bottom line}
\quad\parbox{1.5cm}{center-aligned parbox}
\quad\parbox[t]{2cm}{another parbox aligned at its top line}

\fbox{\parbox{Default parbox within fbox}}

Usando um ambiente chamado minipage para obter um efeito semelhante ao parbox:

\begin{minipage}{3cm}
    \hyphenation{acro-nym}

    TUG is an acronym. It means \TeX\ Users Group. Largura de 3cm.

\end{minipage}

O comando hyphenation serve para, manualmente, indicar qual parte da palavra pode ser separada quando encontrado o fim do espaço de escrita.

Podemos usar o pacote hyphenat para aumentar as possibilidades.
Podemos prevenir hiphenização com
\usepackage[none]{hyphenat}
E permitir hiphenização de texto da fonte typewriter com
\usepackage[htt]{hyphenat}

Para pequenos trechos, podemos inibir a hyphenação com \nohyphens{pequeno trecho de texto}

O pacote microtype mexe no espaçamento para obter o efeito de texto justificado.
No exemplo dado, mexeria no microtype para tentar eliminar a necessidade de hiphens.

Vamos quebrar linhas manualmente, usando de exemplo um poema do Allan Poe.

\emph{Annabel Lee} \\[3mm]
It was many and many years ago, \\
In a kingdom by the sea. \\
That a maiden there lived whom you may know \\
By the name of Annabel Lee

O comando que possui o mesmo efeito de \\
é o comando \newline.

O comando que indica que deve ser adicionada uma nova linha,
mas mantendo a justificação do texto \linebreak, pode causar
alongamento demasiado na distância entre palavras.
Por isso, é raramente usado.

Podemos usar \\[3mm] para adicionar espaçamento vertical.
\\*[5mm] faz o mesmo, mas previne a quebra de página antes da
próxima linha de texto.

\linebreak[number] pode ser usado para influenciar a quebra de linha de forma sutil ou evidente.
Se number for zero, a quebra de linha é permitida, 1 significa que é desejada, 2 e 3 marcam outras requisições de insistência, e 4 irá forcar a quebra de linha.
Esta última é a opção padrão se não for especificado número.

Prevenindo quebras de linha.
\nolinebreak Faz o serviço.
Possui opções assim como o linebreak, sendo as requisições 
se tornando mais insistentes de 0 (recomenda não quebrar)
à 4 (proíbe de quebrar).

O comando \mbox[texto] desabilita a hyphenaçâo \textbf{e} a 
a quebra de linha do texto.

Quando duas palavras devem permanecer unidas na mesma linha utilizamos o símbolo \~.
Exemplo:
Dr. ~Watson previne que Dr. apareça sozinho no fim de uma linha

O comando para nota de rodapé é \footnote{Adicionei uma nota de rodapé}

% Explorando ligaduras -----------

ff fi fl flffifl -- ---

f\/f f\/i f\/l f\/l\/f\/f\/l\/f\/l -\/- -\/-\/-

O comando \/ está impedindo ligaduras de acontecerem.

O mesmo efeito pode ser alcançado usando {},
pois os espaços devem respeitar os limites do ambiente.

Cada quantidade de hiphens utilizados servem a um propósito.

Um único hiphen serve para dividir palavras ou para palavras compostas.

Dois hiphens seguidos servem como separador de intervalo, como
"Das 9--10hrs da manhã".
A largura deste hiphen é equivalente a um dígito.

Três hiphens são convertidos para o travessão.

Algo semelhante ocorre com pontos.
reticências são menos espaçadas entre si do que um ponto 
e o caractere subsequente.




\end{document}
