\documentclass{article}
\begin{document}

Página 60 do livro.
Vamos verificar um pouco do uso de pacotes.

\usepackage{xspace}
\section{The \TUG}

The \TUG is an organization for people
who are interested in \TeX or \LaTeX.

Teoricamente a adição do pacote xspace consertaria o problema dos espaços.

Vamos ver comandos com argumentos.

\newcommand{\keyword}[1]{\textbf{#1}}

O número entre colchetes é o número de argumentos.
\#1 é o primeiro parâmetro. Poderia haver um \#2 e assim
em diante.


\keyword{Grouping} by curly braces limits the

\keyword{scope} of \keyword{declarations}

Exemplo 2:

\newcommand{\keyword}[2][\bfseries]{{#1#2}}
\begin{document}
\keyword{Grouping} by curly braces limits the
\keyword{scope} of \keyword[\itshape]{declarations}.

Modo Genérico:
% \newcommand{command}[arguments][optional]{definition}

\end{document}
