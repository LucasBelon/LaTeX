\documentclass[a4paper, 12pt]{article}
\usepackage[utf8]{inputenc}
\usepackage{amsmath,
            amssymb,
            amsfonts,
            latexsym,
            mathrsfs,
            amsthm,
            amstext,
            bezier,
            amscd}
\usepackage[top=2cm,
            bottom=2cm,
            left=2.5cm,
            right=2.5cm]
            {geometry}
\usepackage{indentfirst}
\usepackage{graphicx}

\begin{document}

\title{
    \textbf{
    Anotações Definitivas
    }
    \break
    Aprendendo a usar vimtex
}
\author{
    Lucas Gouveia Belon
    \\
    lucasgouveiabelon@usp.br
}

\date{\vspace{2.2cm}São Paulo\\31/08/2020}

\maketitle
%\thispagestyle{empty}
%\setcounter{page}{-1}
\newpage
%\thispagestyle{empty}
\tableofcontents
\newpage
\section{Instalação com Vundle}
\subsection{Edições no vimrc e a atualização de plugins}

Para aprender a usar o latex com vim não fica muito claro se é necessário saber antes vim ou tex. Então eu sugiro aprender vim antes. Eu mesmo estou aprendendo a mexer no vimtex para poder criar este documento, e não é tarefa exatamente muito fácil. O uso de uma IDE com certeza simplifica muito as coisas, mas não é sempre que temos máquinas potentes o suficiente para isso. O pacote de ferramentas que vem com o \LaTeX \hspace{0.5cm} é gigantesco em variedade e em peso. São alguns gigabytes de arquivos, muitas fontes, e diferentes compiladores. Até assusta, mas vamos com calma que chegamos lá. Este pequeno projeto é uma forma de eu me ambientar com o vimtex, e se ajudar alguém, é um ponto extra.

Eu perdi todos os itens para sumário e vários itens que já estavam feitos.
Vou ter que refazer mais tarde


Teste
\end{document}
